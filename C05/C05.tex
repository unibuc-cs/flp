\documentclass[xcolor=pdftex,romanian,colorlinks]{beamer}
%\documentclass[xcolor=pdftex,handout,romanian,colorlinks]{beamer}

\usepackage[export]{adjustbox}
\usepackage{../sty/tslides}
\usepackage[all]{xy}
\usepackage{pgfplots}
\usepackage{flowchart}
\usepackage{todonotes}
\usetikzlibrary{arrows,positioning,calc}
\lstset{language=Haskell}
\lstset{escapeinside={(*@}{@*)}}
\PrerenderUnicode{ăĂîÎȘșȚțâÂ}
\usepackage{amsmath}

%\usepackage{xcolor}
%\definecolor{IntensColor}{HTML}{2E86C1}
%\definecolor{StateTransition}{HTML}{D6EAF8}
%\definecolor{MedianLightOrange}{RGB}{216,178,92}
%\definecolor{Orchid}{HTML}{8E44AD}
%\definecolor{True}{HTML}{229954}
%\definecolor{False}{HTML}{CB4335}


\usepackage{proof}
\usepackage{multirow}
\usepackage{alltt}
\usepackage{mathpartir}
\usepackage{ulem}

\newcommand{\structured}[1]{#1}

\definecolor{IntensColor}{HTML}{2E86C1}
\definecolor{StateTransition}{HTML}{D6EAF8}
\definecolor{MedianLightOrange}{RGB}{216,178,92}
\definecolor{Orchid}{HTML}{8E44AD}
\definecolor{True}{HTML}{229954}
\definecolor{False}{HTML}{CB4335}

\newcommand{\cin}[1]{{\color{cobalt} #1}}
\newcommand{\sel}[1]{{\color{Orchid} #1}}

\newcommand{\intens}[1] {{\color{IntensColor} #1}}
\newcommand{\exe}[1] {{\color{True} #1}}

\newcommand{\la}{\lambda}

\setlength{\leftmargini}{0pt}

\newcommand{\app}[2]{#1\, #2}
\newcommand{\abs}[2]{\lambda #1.\,#2}

\newcommand{\type}[2]{{\color{True}#1\hspace{-.05cm}:}\,{\color{Orchid}#2}}

\newcommand{\sub}[3]{#1\langle#2/#3\rangle}
\newcommand{\subt}[3]{#1[#2/#3]}
\newcommand{\equiva}{=_\alpha}

\newcommand{\trueL}{\mathbf{T}}
\newcommand{\falseL}{\mathbf{F}}
\newcommand{\notL}{\mathbf{not}}
\newcommand{\andL}{\mathbf{and}}
\newcommand{\orL}{\mathbf{or}}
\newcommand{\ifL}{\mathbf{if}}
\newcommand{\boolL}{\mathbf{bool}}

\newcommand{\maybeL}{\mathbf{maybe}}
\newcommand{\nothingL}{\mathbf{Nothing}}
\newcommand{\justL}{\mathbf{Just}}
\newcommand{\Maybe}[1]{\mathop{\mathbf{Maybe}}{#1}}

\newcommand{\foldrL}{\mathbf{foldr}}
\newcommand{\nilL}{\mathbf{Nil}}
\newcommand{\consL}{\mathbf{Cons}}
\newcommand{\ListL}[1]{\mathop{\mathbf{List}}{#1}}

\newcommand{\unpairL}{\mathbf{uncons}}
\newcommand{\pairL}{\mathbf{Pair}}
\newcommand{\firstL}{\mathbf{first}}
\newcommand{\secondL}{\mathbf{second}}
\newcommand{\Pair}[2]{\mathop{\mathop{\mathbf{Pair}}{#1}}{#2}}

\newcommand{\succL}{\mathbf{Succ}}
\newcommand{\zeroL}{\mathbf{Zero}}
\newcommand{\iterateL}{\mathbf{iterate}}
\newcommand{\addL}{\mathbf{add}}
\newcommand{\mulL}{\mathbf{mul}}
\newcommand{\expL}{\mathbf{exp}}
\newcommand{\isZero}{\mathbf{isZero}}
\newcommand{\pred}{\mathbf{pred}}
\newcommand{\factL}{\mathbf{fact}}

\newcommand{\BoolT}{\ensuremath{\texttt{Bool}}}
%\newcommand{\BoolT}{\ensuremath{\texttt{Bool}}}
\newcommand{\ifT}[3]{\mathrm{if}\ #1\ \mathrm{then}\ #2\ \mathrm{else}\ #3}

\newcommand{\UnitT}{\ensuremath{\texttt{Unit}}}
\newcommand{\unit}{\mathrm{unit}}

\newcommand{\VoidT}{\ensuremath{\texttt{Void}}}
\newcommand{\void}{\mathrm{void}}


\newcommand{\ProductT}[2]{#1 \times #2}
\newcommand{\PairL}[2]{\langle #1,#2\rangle}
\newcommand{\ProjOne}[1]{fst\ #1}
\newcommand{\ProjTwo}[1]{snd\ #1}

\newcommand{\SumT}[2]{#1 + #2}
\newcommand{\Left}[1]{\mathrm{Left}\ #1}
\newcommand{\Right}[1]{\mathrm{Right}\ #1}
\newcommand{\Case}[3]{\mathrm{case}\ #1\ \mathrm{of}\ #2\ ;\ #3}

%----------------------------------------------

\newcommand{\SSnot}{\terminal{not}}

\newcommand{\Sand}{\terminal{and}}
\newcommand{\Sor}{\terminal{or}}
\newcommand{\Splus}{\terminal{+}}
\newcommand{\Smul}{\terminal{*}}
\newcommand{\Ssucc}{\terminal{S}}
\newcommand{\Spow}{\terminal{pow}}
\newcommand{\Spred}{\terminal{pred}}
\newcommand{\Seq}{\terminal{eq}}
\newcommand{\Sneq}{\terminal{neq}}

\newcommand{\SisZero}{\terminal{isZero}}
\newcommand{\Slte}{\terminal{<=}}
\newcommand{\Sgte}{\terminal{>=}}
\newcommand{\Slt}{\terminal{<}}
\newcommand{\Sgt}{\terminal{>}}
\newcommand{\Spair}{\terminal{pair}}
\newcommand{\Sfst}{\terminal{fst}}
\newcommand{\Ssnd}{\terminal{snd}}
\newcommand{\Sminus}{\terminal{-}}

\newcommand{\Snull}{\terminal{null}}
\newcommand{\Scons}{\terminal{cons}}
%\newcommand{\c sead}{\terminal{head}}
\newcommand{\SisNull}{\terminal{?null}}
\newcommand{\Stail}{\terminal{tail}}
\newcommand{\Ssum}{\terminal{sum}}
\newcommand{\Sfoldr}{\terminal{foldr}}
\newcommand{\Smap}{\terminal{map}}
\newcommand{\Sfilter}{\terminal{filter}}

\newcommand{\const}[1]{\triangleright {\color{False} #1}}

\newcommand{\egf}[1]{\stackrel{\cdot}{=}_{#1}}

\newcommand{\Conf}[2]{\ensuremath{\langle #1\ ,\ #2\rangle}}
\newcommand{\plus}[1] {{\color{True} #1}}
\newcommand{\te}[1]{\mbox{\texttt{#1}}}

\newcommand{\vexp}{\ensuremath{\mathbb{E}}}
\newcommand{\bexp}{\ensuremath{\mathbb{B}}}
\newcommand{\cmd}{\ensuremath{\mathbb{C}}}

\definecolor{section-color}{HTML}{23373b} %mDarkTeal
%\AtBeginSection[]{
%  \begin{frame}
%  \vfill
%  \centering
%  \begin{beamercolorbox}[sep=8pt,center,shadow=true,rounded=true]{title}
%    \usebeamerfont{title}\insertsectionhead\par%
%  \end{beamercolorbox}
%  \vfill
%  \end{frame}
%}



\title[FLP]{Fundamentele limbajelor de programare}
\subtitle{C05}
\date{}


\begin{document}
\begin{frame}
  \titlepage
\end{frame}

\setlength{\leftmargini}{12pt}


%================================================
\section{\color{section-color} Lambda calcul cu tipuri simple}
%================================================

%------------------------------------------------
\begin{frame}{Probleme cu lambda calculul fără tipuri}

\intens{Proprietăți negative ale lambda calculului fără tipuri:}

\hspace{.4cm}{\color{False} - } Aplicări de forma $\app{x}{x}$ sau $\app{M}{M}$ sunt pemise, \\ \hspace{.7cm}deși sunt contraintuitive.

\hspace{.4cm}{\color{False} - } Existența formelor normale pentru $\lambda$-termeni nu este garantată \\ \hspace{.7cm}și putem avea "calcule infinite" nedorite

\hspace{.4cm}{\color{False} - } Orice $\lambda$-termen are un punct fix ceea ce nu este în armonie cu \\ \hspace{.7cm}ceea ce știam despre funcții oarecare

Vrem să eliminăm aceste proprietăți negative, păstrându-le pe cele pozitive.

Proprietățile negative sunt eliminate prin adăugarea de \intens{tipuri} ceea ce induce restricțiile dorite pe termeni.

\end{frame}

%------------------------------------------------
\begin{frame}{Tipuri simple}

Fie \intens{$\mathbb{V} = \{\alpha, \beta, \gamma, \ldots\}$} o mulțime infinită de \alert{tipuri variabilă}. \\
Mulțimea tuturor \alert{tipurilor simple} $\mathbb{T}$ este definită prin
\vspace{-.3cm}
\begin{center}
\intens{$\mathbb{T} = \mathbb{V}\ |\ \mathbb{T} \rightarrow \mathbb{T}$}
\end{center}
\vspace{-.5cm}
\begin{itemize}
	\item \alert{(Tipul variabilă)} Dacă $\alpha \in \mathbb{V}$, atunci $\alpha \in \mathbb{T}$.
	\item \alert{(Tipul săgeată)} Dacă $\sigma,\tau \in \mathbb{T}$, atunci  $(\sigma \rightarrow \tau) \in \mathbb{T}$.
\end{itemize}
\vspace{-.2cm}
Câteodată vom nota tipurile simple și cu litere $A,B,\ldots$.

%\medskip
%{\color{True} Exemple de tipuri simple:}  $\gamma$, $(\beta \to \gamma)$, $((\gamma \to \alpha) \to (\alpha \to (\beta \to \gamma)))$.

\smallskip
Tipurile variabilă sunt reprezentări abstracte pentru \intens{tipuri de bază} cum ar fi $Nat$ pentru numere naturale, $List$ pentru liste etc. 

\smallskip
Tipurile săgeată reprezintă \intens{tipuri pentru funcții}  cum ar fi 
\vspace{-.2cm}
\begin{itemize}
	\item $Nat \to Real$, mulțimea tuturor funcțiilor de la numere naturale la numere reale
	\item $(Nat \to Int) \to (Int \to Nat)$, mulțimea tuturor funcțiilor care au ca intrare o funcție de la numere naturale la întregi și produce o funcție de la întregi la numere naturale.
\end{itemize}
\end{frame}

%------------------------------------------------
\begin{frame}{Tipuri simple}

\begin{center}
 \alert{Mulțimea tipurilor simple} \hspace{.2cm} \intens{$\mathbb{T} = \mathbb{V}\ |\ \mathbb{T} \rightarrow \mathbb{T}$}
\end{center}

{\color{True} Exemple de tipuri simple:}  
\vspace{-.2cm}
\begin{itemize}
	\item $\gamma$
	\item $(\beta \to \gamma)$
	\item $((\gamma \to \alpha) \to (\alpha \to (\beta \to \gamma)))$
\end{itemize}

\medskip
În tipurile săgeată, parantezele exterioare pot fi omise.

\medskip
\intens{Parantezele în tipurile săgeată sunt asociative la dreapta.} \\
De exemplu,
\vspace{-.2cm}
\begin{itemize}
	\item $\alpha_1 \to \alpha_2 \to \alpha_3 \to \alpha_4$ este abreviere pentru $(\alpha_1 \to (\alpha_2 \to (\alpha_3 \to \alpha_4)))$
	\item $\app{\app{\app{x_1}{x_2}}{x_3}}{x_4}$ este abreviere pentru $(\app{(\app{(\app{x_1}{x_2})}{x_3})}{x_4})$
\end{itemize}
\end{frame}

%------------------------------------------------
\begin{frame}{Termeni și tipuri}

\intens{Ce înseamnă că un termen $M$ are un tip $\sigma$?} \\
Vom nota acest lucru cu \intens{$\type{M}{\sigma}$}.

\pause
\smallskip
\intens{Variabilă.} Dacă o variabilă $x$ are un tip $\sigma$, notăm cu \alert{$\type{x}{\sigma}$}. \\
	\alert{Convenția Barendregt:} variabilele legate sunt distincte.\\
	Presupunem că \alert{orice variabilă din $M$ are un unic tip}. \\
	Dacă $\type{x}{\sigma}$ și $\type{x}{\tau}$, atunci $\sigma \equiv \tau$.

\pause
\smallskip
\intens{Aplicare.} Pentru $\app{M}{N}$ este clar că vrem să știm tipurile lui $M$ și $N$. \\
Intuitiv, $\app{M}{N}$ înseamnă că ("funcția") $M$ este aplicată ("intrării") $N$. \\
Atunci $M$ trebuie să aibă un tip funcție, adică $\type{M}{\sigma \to \tau}$, iar $N$ trebuie să fie "adecvat" pentru această funcție, adică $\type{N}{\sigma}$.
\vspace{-.2cm}
\begin{center}
Dacă $\type{M}{\sigma \to \tau}$ și $\type{N}{\sigma}$, atunci $\type{\app{M}{N}}{\tau}$.
\end{center}

\pause
\smallskip
\intens{Abstractizare.} Dacă $\type{M}{\tau}$, ce tip trebuie sa aibă $\abs{x}{M}$?\\
\vspace{-.2cm}
\begin{center}
Dacă $\type{x}{\sigma}$ și $\type{M}{\tau}$, atunci $\type{\abs{x}{M}}{\sigma \to \tau}$.
\end{center}


\end{frame}

%------------------------------------------------
\begin{frame}{Termeni și tipuri}

\intens{Variabilă.} $\type{x}{\sigma}$. \\
\intens{Aplicare.} Dacă $\type{M}{\sigma \to \tau}$ și $\type{N}{\sigma}$, atunci $\type{\app{M}{N}}{\tau}$. \\
\intens{Abstractizare.} Dacă $\type{x}{\sigma}$ și $\type{M}{\tau}$, atunci $\type{\abs{x}{M}}{\sigma \to \tau}$.

$M$ \intens{\textit{are tip}} (este \intens{\textit{typeable}}) dacă există un tip $\sigma$ astfel încât $\type{M}{\sigma}$.

\pause
{\color{True} Exemple.}
\vspace{-.2cm}
\begin{itemize}
	\item Dacă {\color{True}$\type{x}{\sigma}$}, atunci funcția identitate are tipul {\color{True}$\type{\abs{x}{x}}{\sigma \to \sigma}$}.
	\smallskip
	\pause
	\item Conform convențiilor de la aplicare, {\color{True}$\app{y}{x}$} poate avea un tip doar dacă $y$ are un tip săgeată de forma $\sigma \to \tau$ și tipul lui $x$ se potrivește cu tipul domeniu $\sigma$. În acest caz, tipul lui {\color{True}$\type{\app{y}{x}}{\tau}$}.
	\smallskip
	\pause
	\item {\color{False}Termenul $\app{x}{x}$ nu poate avea nici un tip}  (nu este typeable). \\ 
	Pe de o parte, $x$ ar trebui să aibă tipul $\sigma \to \tau$ (pentru prima apariție), pe de altă ar trebui să aibă tipul $\sigma$ (pentru a doua apariție). Cum am stabilit că orice variabilă are un unic tip, obținem $\sigma \to \tau \equiv \sigma$, ceea ce este imposibil.
\end{itemize} 
\end{frame}


%------------------------------------------------
\begin{frame}{Discuție despre asocitativitate}

\intens{Asociativitatea la dreapta pentru tipurile săgeată vs. asociativitatea la stânga pentru aplicare:}
\begin{itemize}
	\item Să presupunem că {\color{True}$\type{f}{\rho \to (\sigma \to \tau)}$}, {\color{True}$\type{x}{\rho}$} și {\color{True}$\type{y}{\sigma}$}.
	\item Atunci {\color{True}$\type{\app{f}{x}}{\sigma \to \tau}$} și {\color{True}$\type{\app{(\app{f}{x})}{y}}{\tau}$}.
	\pause
	\item Folosind ambele convenții pentru asociativitate pentru a elimina parantezele, avem
	\begin{center}
	\begin{tabular}{l}
	{\color{True}$\type{f}{\rho \to \sigma \to \tau}$} \\
	{\color{True}$\type{\app{\app{f}{x}}{y}}{\tau}$}
	\end{tabular}
	\end{center}
	Convențiile pentru asociativitate sunt în armonie una cu cealaltă.
\end{itemize}
\end{frame}

%------------------------------------------------
\begin{frame}{Church-typing vs. Curry-typing}

A găsi tipul unui termen începe cu a găsi tipurile pentru variabile. \\
Există două metode prin care putem asocia tipuri variabilelor.

\intens{Asociere explicită \textit{(Church-typing)}.}
\vspace{-.2cm}
\begin{itemize}
	\item Constă în prescrierea unui unic tip pentru fiecare variabilă, la introducerea acesteia.
	\item Presupune că tipurile variabilelor sunt explicit stabilite.
	\item Tipurile termenilor mai complecși se obțin natural, ținând cont de convențiile pentru aplicare și abstractizare.
\end{itemize}

\intens{Asociere implicită \textit{(Curry-typing)}.}
\vspace{-.2cm}
\begin{itemize}
	\item Constă în a nu prescriere un tip pentru fiecare variabilă, ci în a le lăsa "deschise" (implicite).
	\item În acest caz, termenii \textit{typeable} sunt descoperiți printr-un proces de căutare, care poate presupune "ghicirea" anumitor tipuri.
\end{itemize}

\end{frame}

%------------------------------------------------
\begin{frame}{Church-typing vs. Curry-typing}

{\color{True} Exemplu.}
\intens{Asociere explicită \textit{(Church-typing)}.}

Vrem să calculăm tipul expresiei $\app{(\abs{zu}{z})}{(\app{y}{x})}$ știind că
\begin{minipage}{.45\columnwidth}
\begin{enumerate}
\item {\color{True}$\type{x}{\alpha \to \alpha}$}
\item {\color{True}$\type{y}{(\alpha \to \alpha) \to \beta}$}
\item ${\color{True}\type{z}{\beta}}$
\item ${\color{True}\type{u}{\gamma}}$
\end{enumerate}
\end{minipage} \hfill
\begin{minipage}{.54\columnwidth}
\intens{Aplicare.} \hfill Dacă $\type{M}{\sigma \to \tau}$ și $\type{N}{\sigma}$,

\hfill atunci $\type{\app{M}{N}}{\tau}$.

\intens{Abstractizare.} \hfill Dacă $\type{x}{\sigma}$ și $\type{M}{\tau}$,

\hfill atunci $\type{\abs{x}{M}}{\sigma \to \tau}$.
\end{minipage}
\pause

Din (2) și (1), prin aplicare obținem (5): ${\color{True}\type{\app{y}{x}}{\beta}}$.

Din (4) și (3), prin abstractizare obținem (6): ${\color{True}\type{\abs{u}{z}}{\gamma \to \beta}}$.

Din (3) și (6), prin abstractizare obținem (7): ${\color{True}\type{\abs{zu}{z}}{\beta \to \gamma \to \beta}}$. \\
Nu uitați că $\beta \to \gamma \to \beta$ înseamnă $\beta \to (\gamma \to \beta)$.

Atunci, din (7) și (5), prin aplicare, avem {\color{True}$\type{\app{(\abs{zu}{z})}{(\app{y}{x})}}{\gamma \to \beta}$}.
\end{frame}

%------------------------------------------------
\begin{frame}{Church-typing vs. Curry-typing}

{\color{True} Exemplu.}
\intens{Asociere implicită \textit{(Curry-typing)}.}

Considerăm termenul de mai devreme \intens{$M = \app{(\abs{zu}{z})}{(\app{y}{x})}$}.

Putem să "ghicim" tipurile variabilelor astfel încât $M$ să \textit{aibă tip}?

\intens{Aplicare.} \hfill Dacă $\type{M}{\sigma \to \tau}$ și $\type{N}{\sigma}$,
\hfill atunci $\type{\app{M}{N}}{\tau}$.

\intens{Abstractizare.} \hfill Dacă $\type{x}{\sigma}$ și $\type{M}{\tau}$,
\hfill atunci $\type{\abs{x}{M}}{\sigma \to \tau}$.

\pause\vspace{-.2cm}
\begin{itemize}
	\item Observăm că $M$ este o aplicare a lui {$\abs{zu}{z}$} termenului {$\app{y}{x}$}. 
	\item Atunci {$\abs{zu}{z}$} trebuie să aibă un tip săgeată, de exemplu {\color{True}$\type{\abs{zu}{z}}{A \to B}$}, și {$\app{y}{x}$} să se potrivească, adică {\color{True}$\type{\app{y}{x}}{A}$}. 
	\item În acest caz, avem {\color{True}$\type{M}{B}$}.
\end{itemize}
\end{frame}

%------------------------------------------------
\begin{frame}{Church-typing vs. Curry-typing}

{\color{True} Exemplu.}
\intens{Asociere implicită \textit{(Curry-typing)} (cont.)}

Știm \intens{$M = \app{(\abs{zu}{z})}{(\app{y}{x})}$} și am dedus până acum:
\vspace{-.3cm}
\begin{center}
{\color{True}$\type{\abs{zu}{z}}{A \to B}$} \hspace{.4cm} 
{\color{True}$\type{\app{y}{x}}{A}$} \hspace{.4cm} 
{\color{True}$\type{M}{B}$}  \hspace{.4cm} 
%{\color{True}$\type{z}{A}$} \hspace{.4cm} 
%{\color{True}$\type{\abs{u}{z}}{B}$}
\end{center}
\vspace{-.3cm}

\begin{itemize}
	\item Faptul că {\color{True}$\type{\abs{zu}{z}}{A \to B}$} implică că {\color{True}$\type{z}{A}$} și {\color{True}$\type{\abs{u}{z}}{B}$}.
	\item Deducem că $B$ este tipul unei abstractizări, deci $B \equiv C \to D$, și obținem că {\color{True}$\type{u}{C}$} și {\color{True}$\type{z}{D}$}.
	\item Pe de altă parte, {$\app{y}{x}$} este o aplicare, deci trebuie să existe $E$ și $F$ astfel încât {\color{True}$\type{y}{E \to F}$} și {\color{True}$\type{x}{E}$}. Atunci {\color{True}$\type{\app{y}{x}}{F}$}.
\end{itemize}
\end{frame}

%------------------------------------------------
\begin{frame}{Church-typing vs. Curry-typing}

{\color{True} Exemplu.}
\intens{Asociere implicită \textit{(Curry-typing)} (cont.)}

Știm \intens{$M = \app{(\abs{zu}{z})}{(\app{y}{x})}$}. Am dedus următoarele:
\vspace{-.2cm}
\begin{itemize}
	\item {\color{True}$\type{x}{E}$}
	\item {\color{True}$\type{y}{E \to F}$}
	\item {\color{True}$\type{z}{A}$} și {\color{True}$\type{z}{D}$}, deci $A \equiv D$
	\item {\color{True}$\type{u}{C}$} 
	\item $B \equiv C \to D$ 
	\item {\color{True}$\type{\app{y}{x}}{A}$} și {\color{True}$\type{\app{y}{x}}{F}$}, deci $A \equiv F$.
\end{itemize}
În concluzie, $A \equiv D \equiv F$, și eliminând redundanțele obținem
\vspace{-.2cm}
\begin{center}
$(*)$ \hspace{.3cm}
{\color{True}$\type{x}{E}$} \hspace{.3cm}
{\color{True}$\type{y}{E \to A}$}  \hspace{.3cm}
{\color{True}$\type{z}{A}$}  \hspace{.3cm}
{\color{True}$\type{u}{C}$} 
\end{center}
\vspace{-.2cm}
Reamintim că aveam {\color{True}$\type{M}{B}$}, adică {\color{True}$\type{M}{C \to A}$}.

Am obținut o schemă generală $(*)$ pentru tipurile lui $x,y,z,u$ care induc un tip pentru $M$.
\end{frame}

%------------------------------------------------
\begin{frame}{Church-typing vs. Curry-typing}

{\color{True} Exemplu.}
\intens{Asociere implicită \textit{(Curry-typing)} (cont.)}

Știm \intens{$M = \app{(\abs{zu}{z})}{(\app{y}{x})}$}. Am obținut schema generală
\vspace{-.2cm}
\begin{center}
$(*)$ \hspace{.3cm}
{\color{True}$\type{x}{E}$} \hspace{.3cm}
{\color{True}$\type{y}{E \to A}$}  \hspace{.3cm}
{\color{True}$\type{z}{A}$}  \hspace{.3cm}
{\color{True}$\type{u}{C}$} \hspace{.3cm}
{\color{True}$\type{M}{C \to A}$}
\end{center}
\vspace{-.2cm}

În schema de mai sus, putem considera tipuri "reale":
\begin{itemize}
	\item {\color{True}$\type{x}{\beta}$}, \hspace{.1cm}
{\color{True}$\type{y}{\beta \to \alpha}$},  \hspace{.1cm}
{\color{True}$\type{z}{\alpha}$},  \hspace{.1cm}
{\color{True}$\type{u}{\delta}$}, \hspace{.1cm}
{\color{True}$\type{M}{\delta \to \alpha}$}
\pause
	\item {\color{True}$\type{x}{\alpha \to \alpha}$}, \hspace{.1cm}
{\color{True}$\type{y}{(\alpha \to \alpha) \to \beta}$},  \hspace{.1cm}
{\color{True}$\type{z}{\beta}$},  \hspace{.1cm}
{\color{True}$\type{u}{\gamma}$}, \hspace{.1cm}
{\color{True}$\type{M}{\gamma \to \beta}$} \\
(soluția discutată la Church-typing)
\pause
	\item {\color{True}$\type{x}{\alpha}$}, \hspace{.1cm}
{\color{True}$\type{y}{\alpha \to \alpha \to \beta}$},  \hspace{.1cm}
{\color{True}$\type{z}{\alpha \to \beta}$},  \hspace{.1cm}
{\color{True}$\type{u}{\alpha \to \alpha}$}, \hspace{.1cm}
{\color{True}$\type{M}{(\alpha \to \alpha) \to \alpha \to \beta}$} \\
\end{itemize}
\end{frame}

%------------------------------------------------
\begin{frame}{Church-typing vs. Curry-typing}

Asocierea implicită de tipuri \textit{(Curry-typing)} are proprietăți interesante, cum am văzut în exemplul anterior.

\medskip
Totuși, în continuare vom folosi asocierea explicită \textit{(Church-typing)} deoarece de obicei tipurile sunt cunoscute dinainte (și declararea tipurilor pentru argumentele unei funcții este o bună-practică).

\medskip
Marcăm tipurile \alert{variabilelor legate} imediat după introducerea lor cu o abstractizare.
Tipurile \alert{variabilelor libere} sunt date de un \alert{context}.


\end{frame}

%------------------------------------------------
\begin{frame}{Church-typing}


{\color{True}Exemplu.} Să considerăm exemplul anterior \intens{$\app{(\abs{zu}{z})}{(\app{y}{x})}$}.

Observați că $z$ și $u$ sunt legate, iar $x$ și $y$ sunt libere. 

Presupunând că $\type{z}{\beta}$ și $\type{u}{\gamma}$, scriem termenul astfel
\vspace{-.3cm}
\begin{center}
$\app{(\abs{\type{z}{\beta}}{\abs{\type{u}{\gamma}}{z}})}{(\app{y}{x})}$
\end{center}
\vspace{-.3cm}

\pause
Dacă presupunem un context în care despre variabilele libere știm, de exemplu, că $\type{x}{\alpha \to \alpha}$ și $\type{y}{(\alpha \to \alpha) \to \beta}$, atunci folosim notația:
\vspace{-.3cm}
\begin{center}
$\type{x}{\alpha \to \alpha}, \type{y}{(\alpha \to \alpha) \to \beta} \vdash \app{(\abs{\type{z}{\beta}}{\abs{\type{u}{\gamma}}{z}})}{(\app{y}{x})}$
\end{center}
\vspace{-.3cm}

\pause
Încă nu avem o noțiune de $\beta$-reducție pentru termeni cu tipuri, dar ne-am putea gândi că am avea:
\vspace{-.3cm}
\begin{center}
$\app{(\abs{\type{z}{\beta}}{\abs{\type{u}{\gamma}}{z}})}{(\app{y}{x})} \to_\beta \abs{\type{u}{\gamma}}{\app{y}{x}}$.
\end{center}
\vspace{-.3cm}

Observați că am dori să deducem că $\type{(\abs{\type{u}{\gamma}}{\app{y}{x}})}{\gamma \to \beta}$.

\end{frame}

%------------------------------------------------
\begin{frame}{Sistem de deducție pentru Church $\lambda\hspace{-.1cm}\to$}

Deoarece am convenit cum să decorăm cu informații despre tipuri variabilele legate, trebuie să actualizăm definiția $\lambda$-termenilor.

\medskip
Mulțimea \alert{$\lambda$-termenilor cu pre-tipuri $\Lambda_\mathbb{T}$} este 
\vspace{-.3cm}
\begin{center}
\intens{$\Lambda_\mathbb{T} = x\ |\ \app{\Lambda_\mathbb{T}}{\Lambda_\mathbb{T}}\ |\ \abs{\type{x}{\mathbb{T}}}{\Lambda_\mathbb{T}}$}
\end{center}

O \alert{afirmație} este o expresie de forma $\type{M}{\sigma}$, unde $M \in \Lambda_\mathbb{T}$ și $\sigma\in \mathbb{T}$.\\
Într-o astfel de afirmație, $M$ se numește \alert{subiect} și $\sigma$ \alert{tip}.

\medskip
O \alert{declarație} este o afirmație în care subiectul este o variabilă ($\type{x}{\sigma}$).

\medskip
Un \alert{context} este o listă de declarații cu subiecți diferiți.

\medskip
O \alert{judecată} este o expresie de forma $\Gamma \vdash \type{M}{\sigma}$, unde $\Gamma$ este context și $\type{M}{\sigma}$ este o afirmație.

\end{frame}

%------------------------------------------------
\begin{frame}{Sistem de deducție pentru Church $\lambda\hspace{-.1cm}\to$}

Deoarece suntem în general interesați de termeni \textit{typeable}, am dori să avem o metodă prin care să putem stabili dacă un termen $t \in \Lambda_\mathbb{T}$ este \textit{typeable} și dacă da, să calculăm un tip pentru $t$.

Vom da niște reguli care să ne permită să stabilim dacă o judecată $\Gamma \vdash \type{M}{\sigma}$ poate fi dedusă, adică dacă $M$ are tipul $\sigma$ în contextul $\Gamma$.
\end{frame}

%------------------------------------------------
\begin{frame}{Sistem de deducție pentru calculul Church $\lambda\hspace{-.1cm}\to$}

\begin{center}
\begin{tabular}{c}
\infer[\mbox{dacă $\type{x}{\sigma}\in \Gamma$ }(var)]
	{\Gamma \vdash \type{x}{\sigma}}
	{}
\\[1cm] 
%------------------
\infer[(app)]
	{\Gamma \vdash \type{\app{M}{N}}{\tau}}
	 {\Gamma \vdash \type{M}{\sigma \to \tau} \hspace{.5cm} \Gamma \vdash \type{N}{\sigma}}
 \\[1cm] 
%------------------
\infer[(abs)]
	{\Gamma \vdash \type{(\abs{\type{x}{\sigma}}{M})}{\sigma \to \tau} }
	 {\Gamma, \type{x}{\sigma} \vdash \type{M}{\tau}}
\end{tabular}
\end{center}

Un termen $M$  în calculul $\lambda\hspace{-.1cm}\to$ este \alert{legal} dacă există un context $\Gamma$ și un tip $\rho$ astfel încât $\Gamma \vdash \type{M}{\rho}$.
\end{frame}

%------------------------------------------------
\begin{frame}{Sistem de deducție pentru calculul Church $\lambda\hspace{-.1cm}\to$}
\begin{center}
{\footnotesize
\begin{tabular}{lll}
\infer[(var)]
	{\Gamma \vdash \type{x}{\sigma}}
	{}
%------------------
&
\infer[(app)]
	{\Gamma \vdash \type{\app{M}{N}}{\tau}}
	 {\Gamma \vdash \type{M}{\sigma \to \tau} \hspace{.5cm} \Gamma \vdash \type{N}{\sigma}}
%------------------
&
\infer[(abs)]
	{\Gamma \vdash \type{(\abs{\type{x}{\sigma}}{M})}{\sigma \to \tau} }
	 {\Gamma, \type{x}{\sigma} \vdash \type{M}{\tau}}\\
 dacă $\type{x}{\sigma}\in \Gamma$ 	 & &\\	 
\end{tabular}
}	 
\end{center}

{\color{True} Exemplu.} Să arătăm că termenul $\abs{\type{y}{\alpha \to \beta}}{\abs{\type{z}{\alpha}}{\app{yz}}}$ are tipul $(\alpha \to \beta)\to \alpha \to \beta$ în contextul vid.
\[\emptyset \vdash \type{(\abs{\type{y}{\alpha \to \beta}}{\abs{\type{z}{\alpha}}{\app{yz}}})}{(\alpha \to \beta)\to \alpha \to \beta}\]

\vspace{-.4cm}\pause
\[
\infer[(abs)]
{
\emptyset \vdash \type{(\abs{\type{y}{\alpha \to \beta}}{\abs{\type{z}{\alpha}}{\app{yz}}})}{(\alpha \to \beta)\to \alpha \to \beta}
}
{
\infer[(abs)]
{\type{y}{\alpha \to \beta} \vdash \type{(\abs{\type{z}{\alpha}}{\app{yz}})}{\alpha \to \beta}}
{
{
\infer[(app)]
{\type{y}{\alpha \to \beta}, \type{z}{\alpha} \vdash \type{(\app{yz})}{\beta}}
{
\infer[(var)]
{\type{y}{\alpha \to \beta}, \type{z}{\alpha} \vdash \type{y}{\alpha\to \beta}}{} \hspace{.5cm} 
\infer[(var)]{\type{y}{\alpha \to \beta}, \type{z}{\alpha} \vdash \type{z}{\alpha}}{}
}
}
}
}
\]
\end{frame}

%------------------------------------------------
\begin{frame}{Diferite stiluri pentru a scrie deducții}

În exemplul anterior, am scris derivarea în \alert{stilul arbore}.

În \alert{stilul liniar}, derivarea precedentă ar arăta astfel:
\vspace{-.6cm}
\begin{center}
\begin{tabular}{rll}
1. & $\type{y}{\alpha \to \beta}, \type{z}{\alpha} \vdash \type{y}{\alpha\to \beta}$ & (var) \\
2. & $\type{y}{\alpha \to \beta}, \type{z}{\alpha} \vdash \type{z}{\alpha}$ & (var) \\
3. & $\type{y}{\alpha \to \beta}, \type{z}{\alpha} \vdash \type{(\app{yz})}{\beta}$ & (app) cu 1 și 2 \\
4. & $\type{y}{\alpha \to \beta} \vdash \type{(\abs{\type{z}{\alpha}}{\app{yz}})}{\alpha \to \beta}$ & (abs) cu 3 \\
5. & $\emptyset \vdash \type{(\abs{\type{y}{\alpha \to \beta}}{\abs{\type{z}{\alpha}}{\app{yz}}})}{(\alpha \to \beta)\to \alpha \to \beta}$ & (abs) cu 4 \\
\end{tabular}
\end{center}

\end{frame}

%------------------------------------------------
\begin{frame}{Diferite stiluri pentru a scrie deducții}

În \alert{stilul cu cutii}, afișăm fiecare declarație la începutul unei cutii  și considerăm că declarația respectivă face parte din contextul pentru toate afirmațiile din cutia respectivă.

Când închidem o cutie, abstractizăm după variabila din declarația de la începutul cutiei.

\vspace{-.3cm}
\[
\onslide<4>{\begin{array}{r}
1. \\
2. \\
3. \\
4. \\
5. \\
\end{array}}
\begin{array}{ll}
\only<2->{
\begin{array}{|ll}
\hline
\type{y}{\alpha\to \beta} & (context) \\
\only<3->{
\begin{array}{|ll}
\hline
\type{z}{\alpha} & (context) \\
\type{(\app{yz})}{\beta} & \only<4->{(app) \mbox{ cu }1 \mbox{ și } 2} \\
\hline
\end{array} \\
}
\type{(\abs{\type{z}{\alpha}}{\app{yz}})}{\alpha \to \beta} & \only<3->{(abs)\only<4>{\mbox{ cu } 3}} \\
\hline
\end{array} & \\
}
\type{(\abs{\type{y}{\alpha \to \beta}}{\abs{\type{z}{\alpha}}{\app{yz}}})}{(\alpha \to \beta)\to \alpha \to \beta} & \only<2->{(abs) \only<4>{\mbox{ cu  } 4}} \\
\end{array}
\]

%\begin{center}
%\begin{tabular}{rll}
%1. & $\type{y}{\alpha\to \beta}$ & (var) \\
%2. & $\type{z}{\alpha}$ & (var) \\
%3. & $\type{(\app{yz})}{\beta}$ & (app) cu 1 și 2 \\
%4. & $\type{(\abs{\type{z}{\alpha}}{\app{yz}})}{\alpha \to \beta}$ & (abs) cu 3 \\
%5. & $\type{(\abs{\type{y}{\alpha \to \beta}}{\abs{\type{z}{\alpha}}{\app{yz}}})}{(\alpha \to \beta)\to \alpha \to \beta}$ & (abs) cu 4 \\
%\end{tabular}
%\end{center}
%
\end{frame}

%------------------------------------------------
\begin{frame}{Sistem de deducție pentru Church $\lambda\hspace{-.1cm}\to$}

{\color{True} Exercițiu.} Arătăți că termenul $\abs{\type{x}{((\alpha \to \beta)\to \alpha)}}{\app{x}{({\abs{\type{z}{\alpha}}{y})}}}$ are tipul $((\alpha \to \beta)\to \alpha) \to \alpha$ în contextul $\type{y}{\beta}$.


\end{frame}
%%------------------------------------------------
%\begin{frame}{Ce probleme putem să rezolvăm în teoria tipurilor?}
%
%\intens{1. \textit{Well-typedness (Typability)}}\\
%Se reduce la a verifica că un termen este \alert{legal}.
%Concret, trebuie să găsim un context și un tip dacă termenul este legal, altfel să arătăm de ce nu se poate.
%\vspace{-.3cm}
%\begin{center}
%$? \vdash \type{term}{?}$
%\end{center}
%O variațiune a problemei este \intens{\textit{Type Assignment}} în care contextul este dat și trebuie să găsim tipul.
%\vspace{-.3cm}
%\begin{center}
%$context \vdash \type{term}{?}$
%\end{center}
%
%\end{frame}
%
%%------------------------------------------------
%\begin{frame}{Ce probleme putem să rezolvăm în teoria tipurilor?}
%
%\intens{2. \textit{Type Checking}} \\
%Se reduce la a verifica că putem găsi o derivare pentru
%
%\vspace{-.3cm}
%\begin{center}
%$context \vdash \type{term}{type}$
%\end{center}
%
%
%\intens{3. \textit{Term Finding (Inhabitation)}} \\
%Dându-se un context și un tip, să stabilim dacă există un termen cu acel tip, în contextul dat.
%\vspace{-.3cm}
%\begin{center}
%$context \vdash \type{?}{type}$
%\end{center}
%
%\begin{center}
%\alert{Toate aceste probleme sunt decidabile pentru calculul $\lambda\hspace{-.1cm}\to$!}
%\end{center}
%\end{frame}
%
%%------------------------------------------------
%\begin{frame}[fragile]{Limitări ale lambda-calculului cu tipuri simple}
%
%\begin{block}{Nu mai avem recursie nelimitată}
%\begin{itemize}
%\item Deoarece combinatorii de punct fix nu sunt \textit{typeable}
%
%e.g., \intens{$\mathbf{Y} \triangleq \abs{y}{\app{(\abs{x}{\app{y}{(\app{x}{x})}})}{(\abs{x}{\app{y}{(\app{x}{x})}})}}$}
%\item Dar avem recursie primitivă (limitată, ca aceea cu care definim operații pe numere naturale)
%
%e.g., 
%\intens{$\expL \triangleq \abs{mn}{\app{\app{m}{(\app{\mulL}{n})}}{\overline{1}}}$}  
%\item Faptul că orice evaluare se termină este important pentru implementări ale logicilor folosind lambda-calculul
%\end{itemize}
%\end{block}
%
%\end{frame}
%
%%------------------------------------------------
%\begin{frame}[fragile]{Limitări ale lambda-calculului cu tipuri simple}
%
%\begin{block}{Tipurile pot fi prea restrictive}
%\[\app{(\abs{f}{\app{\app{\app{\ifL}{(\app{f}{\trueL})}}{(\app{f}{3})}}{(\app{f}{5})}})}{(\abs{x}{x})}\]
%am putea zice că expresia de sus ar trebui să aibă un tip.
%
%Soluții posibile:
%\begin{itemize}
%\item Let-polymorphism 
%\(\begin{array}[t]{l}
%	\mathop{\mathbf{let}} f \mathrel{\mathbf{=}} \abs{x}{x} \mathrel{\mathbf{in}} \\
%	\app{\app{\app{\ifL}{(\app{f}{\trueL})}}{(\app{f}{3})}}{(\app{f}{5})}
%\end{array}\)
%
%variabilele libere din tipul lui $f$ se redenumesc la fiecare folosire
%\item Cuantificatori de tipuri
%\[\abs{x}{x} : \Pi\alpha\mathrel{.}\alpha \to \alpha\]
%Asemănător ca mai sus. Operatorul de legare $\Pi$ face explicit faptul că
%variabila de tip $\alpha$ nu este rigidă.
%\end{itemize}
%\end{block}
%
%\end{frame}

%------------------------------------------------
\begin{frame}
  \vfill
  \centering

\textbf{\large \alert{Quiz time!}}

\includegraphics[scale=.35]{../Quiz/C05-Q1.png}

 \url{https://tinyurl.com/2p9xf67e}
  \vfill
\end{frame}

%---------------------------------------------
\begin{frame}
  \vfill
  \centering

\textbf{Pe săptămâna viitoare!}

  \vfill
\end{frame}
\end{document}







