\documentclass[xcolor=pdftex,romanian,colorlinks]{beamer}
%\documentclass[xcolor=pdftex,handout,romanian,colorlinks]{beamer}

\usepackage[export]{adjustbox}
\usepackage{../sty/tslides}
\usepackage[all]{xy}
\usepackage{pgfplots}
\usepackage{flowchart}
\usepackage{todonotes}
\usetikzlibrary{arrows,positioning,calc}
\lstset{language=Haskell}
\lstset{escapeinside={(*@}{@*)}}
\PrerenderUnicode{ăĂîÎȘșȚțâÂ}
\usepackage{amsmath}

%\usepackage{xcolor}
%\definecolor{IntensColor}{HTML}{2E86C1}
%\definecolor{StateTransition}{HTML}{D6EAF8}
%\definecolor{MedianLightOrange}{RGB}{216,178,92}
%\definecolor{Orchid}{HTML}{8E44AD}
%\definecolor{True}{HTML}{229954}
%\definecolor{False}{HTML}{CB4335}


\usepackage{proof}
\usepackage{multirow}
\usepackage{alltt}
\usepackage{mathpartir}
\usepackage{ulem}

\newcommand{\structured}[1]{#1}

\definecolor{IntensColor}{HTML}{2E86C1}
\definecolor{StateTransition}{HTML}{D6EAF8}
\definecolor{MedianLightOrange}{RGB}{216,178,92}
\definecolor{Orchid}{HTML}{8E44AD}
\definecolor{True}{HTML}{229954}
\definecolor{False}{HTML}{CB4335}

\newcommand{\cin}[1]{{\color{cobalt} #1}}
\newcommand{\sel}[1]{{\color{Orchid} #1}}

\newcommand{\intens}[1] {{\color{IntensColor} #1}}
\newcommand{\exe}[1] {{\color{True} #1}}

\newcommand{\la}{\lambda}

\setlength{\leftmargini}{0pt}

\newcommand{\app}[2]{#1\, #2}
\newcommand{\abs}[2]{\lambda #1.\,#2}

\newcommand{\type}[2]{{\color{True}#1\hspace{-.05cm}:}\,{\color{Orchid}#2}}

\newcommand{\sub}[3]{#1\langle#2/#3\rangle}
\newcommand{\subt}[3]{#1[#2/#3]}
\newcommand{\equiva}{=_\alpha}

\newcommand{\trueL}{\mathbf{T}}
\newcommand{\falseL}{\mathbf{F}}
\newcommand{\notL}{\mathbf{not}}
\newcommand{\andL}{\mathbf{and}}
\newcommand{\orL}{\mathbf{or}}
\newcommand{\ifL}{\mathbf{if}}
\newcommand{\boolL}{\mathbf{bool}}

\newcommand{\maybeL}{\mathbf{maybe}}
\newcommand{\nothingL}{\mathbf{Nothing}}
\newcommand{\justL}{\mathbf{Just}}
\newcommand{\Maybe}[1]{\mathop{\mathbf{Maybe}}{#1}}

\newcommand{\foldrL}{\mathbf{foldr}}
\newcommand{\nilL}{\mathbf{Nil}}
\newcommand{\consL}{\mathbf{Cons}}
\newcommand{\ListL}[1]{\mathop{\mathbf{List}}{#1}}

\newcommand{\unpairL}{\mathbf{uncons}}
\newcommand{\pairL}{\mathbf{Pair}}
\newcommand{\firstL}{\mathbf{first}}
\newcommand{\secondL}{\mathbf{second}}
\newcommand{\Pair}[2]{\mathop{\mathop{\mathbf{Pair}}{#1}}{#2}}

\newcommand{\succL}{\mathbf{Succ}}
\newcommand{\zeroL}{\mathbf{Zero}}
\newcommand{\iterateL}{\mathbf{iterate}}
\newcommand{\addL}{\mathbf{add}}
\newcommand{\mulL}{\mathbf{mul}}
\newcommand{\expL}{\mathbf{exp}}
\newcommand{\isZero}{\mathbf{isZero}}
\newcommand{\pred}{\mathbf{pred}}
\newcommand{\factL}{\mathbf{fact}}

\newcommand{\BoolT}{\ensuremath{\texttt{Bool}}}
%\newcommand{\BoolT}{\ensuremath{\texttt{Bool}}}
\newcommand{\ifT}[3]{\mathrm{if}\ #1\ \mathrm{then}\ #2\ \mathrm{else}\ #3}

\newcommand{\UnitT}{\ensuremath{\texttt{Unit}}}
\newcommand{\unit}{\mathrm{unit}}

\newcommand{\VoidT}{\ensuremath{\texttt{Void}}}
\newcommand{\void}{\mathrm{void}}


\newcommand{\ProductT}[2]{#1 \times #2}
\newcommand{\PairL}[2]{\langle #1,#2\rangle}
\newcommand{\ProjOne}[1]{fst\ #1}
\newcommand{\ProjTwo}[1]{snd\ #1}

\newcommand{\SumT}[2]{#1 + #2}
\newcommand{\Left}[1]{\mathrm{Left}\ #1}
\newcommand{\Right}[1]{\mathrm{Right}\ #1}
\newcommand{\Case}[3]{\mathrm{case}\ #1\ \mathrm{of}\ #2\ ;\ #3}

%----------------------------------------------

\newcommand{\SSnot}{\terminal{not}}

\newcommand{\Sand}{\terminal{and}}
\newcommand{\Sor}{\terminal{or}}
\newcommand{\Splus}{\terminal{+}}
\newcommand{\Smul}{\terminal{*}}
\newcommand{\Ssucc}{\terminal{S}}
\newcommand{\Spow}{\terminal{pow}}
\newcommand{\Spred}{\terminal{pred}}
\newcommand{\Seq}{\terminal{eq}}
\newcommand{\Sneq}{\terminal{neq}}

\newcommand{\SisZero}{\terminal{isZero}}
\newcommand{\Slte}{\terminal{<=}}
\newcommand{\Sgte}{\terminal{>=}}
\newcommand{\Slt}{\terminal{<}}
\newcommand{\Sgt}{\terminal{>}}
\newcommand{\Spair}{\terminal{pair}}
\newcommand{\Sfst}{\terminal{fst}}
\newcommand{\Ssnd}{\terminal{snd}}
\newcommand{\Sminus}{\terminal{-}}

\newcommand{\Snull}{\terminal{null}}
\newcommand{\Scons}{\terminal{cons}}
%\newcommand{\c sead}{\terminal{head}}
\newcommand{\SisNull}{\terminal{?null}}
\newcommand{\Stail}{\terminal{tail}}
\newcommand{\Ssum}{\terminal{sum}}
\newcommand{\Sfoldr}{\terminal{foldr}}
\newcommand{\Smap}{\terminal{map}}
\newcommand{\Sfilter}{\terminal{filter}}

\newcommand{\const}[1]{\triangleright {\color{False} #1}}

\newcommand{\egf}[1]{\stackrel{\cdot}{=}_{#1}}

\newcommand{\Conf}[2]{\ensuremath{\langle #1\ ,\ #2\rangle}}
\newcommand{\plus}[1] {{\color{True} #1}}
\newcommand{\te}[1]{\mbox{\texttt{#1}}}

\newcommand{\vexp}{\ensuremath{\mathbb{E}}}
\newcommand{\bexp}{\ensuremath{\mathbb{B}}}
\newcommand{\cmd}{\ensuremath{\mathbb{C}}}

\definecolor{section-color}{HTML}{23373b} %mDarkTeal
%\AtBeginSection[]{
%  \begin{frame}
%  \vfill
%  \centering
%  \begin{beamercolorbox}[sep=8pt,center,shadow=true,rounded=true]{title}
%    \usebeamerfont{title}\insertsectionhead\par%
%  \end{beamercolorbox}
%  \vfill
%  \end{frame}
%}



\title[FLP]{Fundamentele limbajelor de programare}
\subtitle{C04}
\date{}


\begin{document}
\begin{frame}
  \titlepage
\end{frame}

\setlength{\leftmargini}{12pt}


%================================================
\section{\color{section-color} Expresivitatea $\lambda$-calculului}
%================================================

\begin{frame}{Expresivitatea $\lambda$-calculului}

Deși lambda calculul constă doar în $\lambda$-termeni, putem reprezenta și manipula tipuri de date comune.

Vom vedea cum putem reprezenta:
\begin{itemize}
	\item valori booleene
	\item numere naturale
%	\item liste
%	\item maybe 
\end{itemize}
\end{frame}

%================================================
\section{\color{section-color} \large Booleeni}
%================================================

%------------------------------------------------
\begin{frame}{Booleeni}

Vrem să definim $\lambda$-termeni care să reprezinte constantele booleene.

Sunt mai multe modalități, una dintre ele fiind:

\begin{itemize}
	\item \intens{$\trueL \triangleq \abs{xy}{x}$}  \hfill (dintre cele două alternative o alege pe prima)
	\item \intens{$\falseL \triangleq \abs{xy}{y}$} \hfill (dintre cele două alternative o alege pe a doua)
\end{itemize}

\end{frame}

%------------------------------------------------
\begin{frame}{Booleeni}

\begin{center}
 \intens{$\trueL \triangleq \abs{xy}{x}$} \hspace{1cm} \intens{$\falseL \triangleq \abs{xy}{y}$}
\end{center}

\vspace{-.2cm}
Acum am vrea să definim un test condiționat $\ifL$. 

Am vrea ca $\ifL$ să ia trei argumente $b,t,f$, unde $b$ este o valoare booleană, iar $t,f$ sunt $\lambda$-termeni oarecare. 

Funcția ar trebui să returneze $t$ dacă $b = true$ și $f$ dacă $b = false$
\begin{center}
$\ifL = \abs{btf}{\left\{ \begin{array}{ll}
         t, & \mbox{if $b = true$},\\
        f, & \mbox{if $b = false$}.\end{array} \right.}$
\end{center}

Deoarece $\app{\app{\trueL}{t}}{f} \twoheadrightarrow_\beta t$ și $\app{\app{\falseL}{t}}{f} \twoheadrightarrow_\beta f$, $\ifL$ trebuie doar să aplice argumentul său boolean celorlalte argumente:
\begin{center}
\intens{$\ifL \triangleq \abs{btf}{\app{\app{b}{t}}{f}}$}
\end{center}
\end{frame}

%------------------------------------------------
\begin{frame}{Booleeni}

\vspace{-.2cm}
\begin{center}
 \intens{$\trueL \triangleq \abs{xy}{x}$} \hspace{1cm} \intens{$\falseL \triangleq \abs{xy}{y}$ \hspace{1cm} $\ifL \triangleq \abs{btf}{\app{\app{b}{t}}{f}}$}
\end{center}

\vspace{-.2cm}
Celelalte operații booleene pot fi definite folosind $\ifL$:

\vspace{-.2cm}
\begin{center}
\begin{tabular}{rcl}
\intens{$\andL$} & \hspace{-.3cm} \intens{$\triangleq$} & \hspace{-.3cm} \intens{$\abs{b_1b_2}{\app{\app{\app{\ifL}{b_1}}{b_2}}{\falseL}}$} \\
\intens{$\orL$} & \hspace{-.3cm} \intens{$\triangleq$} & \hspace{-.3cm} \intens{$\abs{b_1b_2}{\app{\app{\app{\ifL}{b_1}}{\trueL}}{b_2}}$} \\
\intens{$\notL$} & \hspace{-.3cm} \intens{$\triangleq$} & \hspace{-.3cm} \intens{$\abs{b_1}{\app{\app{\app{\ifL}{b_1}}{\falseL}}{\trueL}}$} \\
\end{tabular}
\end{center}

\vspace{-.2cm}
Observați că aceste operații lucrează corect doar dacă primesc ca argumente valori booleene. 

Nu există nicio garanție să se comporte rezonabil pe orice alți $\lambda$-termeni.

Folosind lambda calcul fără tipuri, avem \intens{\textit{garbage in, garbage out}}.

Codările nu sunt unice. De exemplu, pentru $\andL$ am fi putut folosi codarile {$\abs{b_1b_2}{\app{\app{b_2}{b_1}}{b_2}}$} sau  {$\abs{b_1b_2}{\app{\app{b_1}{b_2}}{\falseL}}$}.
\end{frame}

%------------------------------------------------
\begin{frame}{Booleeni}

\begin{center}
 \intens{$\trueL \triangleq \abs{xy}{x}$} \hspace{.4cm} \intens{$\falseL \triangleq \abs{xy}{y}$ \hspace{.4cm} $\ifL \triangleq \abs{btf}{\app{\app{b}{t}}{f}}$}
 
 \begin{tabular}{rcl}
\intens{$\andL$} & \hspace{-.3cm} \intens{$\triangleq$} & \hspace{-.3cm} \intens{$\abs{b_1b_2}{\app{\app{\app{\ifL}{b_1}}{b_2}}{\falseL}}$} \\
\intens{$\orL$} & \hspace{-.3cm} \intens{$\triangleq$} & \hspace{-.3cm} \intens{$\abs{b_1b_2}{\app{\app{\app{\ifL}{b_1}}{\trueL}}{b_2}}$} \\
\intens{$\notL$} & \hspace{-.3cm} \intens{$\triangleq$} & \hspace{-.3cm} \intens{$\abs{b_1}{\app{\app{\app{\ifL}{b_1}}{\falseL}}{\trueL}}$} \\
\end{tabular}

\end{center}


\intens{\textbf{Exercițiu.}} Aduceți la o formă normală următorii termenii:
\begin{itemize}
	\item \alert{$\app{\app{\andL}{\trueL}{\falseL}}$}
	\item \alert{$\app{\app{\orL}{\falseL}{\trueL}}$}
	\item \alert{$\app{\notL}{\trueL} $}
\end{itemize}
\end{frame}

%------------------------------------------------
\begin{frame}{Booleeni}

\begin{center}
 \intens{$\trueL \triangleq \abs{xy}{x}$} \hspace{.4cm} \intens{$\falseL \triangleq \abs{xy}{y}$ \hspace{.4cm} $\ifL \triangleq \abs{btf}{\app{\app{b}{t}}{f}}$}
 
 \begin{tabular}{rcl}
\intens{$\andL$} & \hspace{-.3cm} \intens{$\triangleq$} & \hspace{-.3cm} \intens{$\abs{b_1b_2}{\app{\app{\app{\ifL}{b_1}}{b_2}}{\falseL}}$} \\
\intens{$\orL$} & \hspace{-.3cm} \intens{$\triangleq$} & \hspace{-.3cm} \intens{$\abs{b_1b_2}{\app{\app{\app{\ifL}{b_1}}{\trueL}}{b_2}}$} \\
\intens{$\notL$} & \hspace{-.3cm} \intens{$\triangleq$} & \hspace{-.3cm} \intens{$\abs{b_1}{\app{\app{\app{\ifL}{b_1}}{\falseL}}{\trueL}}$} \\
\end{tabular}

\end{center}

\textbf{Soluții:}

$\app{\app{\andL}{\trueL}{\falseL}} = \app{\app{(\abs{b_1b_2}{\app{\app{\app{\ifL}{b_1}}{b_2}}{\falseL}})}{\trueL}{\falseL}} \twoheadrightarrow_\beta \app{\app{\app{\ifL}{\trueL}}{\falseL}}{\falseL} = \app{\app{\app{(\abs{btf}{\app{\app{b}{t}}{f}})}{\trueL}}{\falseL}}{\falseL}$\\
\hspace{1.2cm} $\twoheadrightarrow_\beta  \app{\app{\trueL}{\falseL}}{\falseL} = \app{\app{(\abs{xy}{x})}{\falseL}}{\falseL} \twoheadrightarrow_\beta \falseL$

$\app{\app{\orL}{\falseL}{\trueL}} = \app{\app{(\abs{b_1b_2}{\app{\app{\app{\ifL}{b_1}}{\trueL}}{b_2}})}{\falseL}{\trueL}} \twoheadrightarrow_\beta \app{\app{\app{\ifL}{\falseL}}{\trueL}}{\trueL} = \app{\app{\app{(\abs{btf}{\app{\app{b}{t}}{f}})}{\falseL}}{\trueL}}{\trueL}$ \\
\hspace{1.2cm} $\twoheadrightarrow_\beta \app{\app{\falseL}{\trueL}}{\trueL} =  \app{\app{(\abs{xy}{y})}{\trueL}}{\trueL} \twoheadrightarrow_\beta  \trueL$

$\app{\notL}{\trueL} = \app{(\abs{b_1}{\app{\app{\app{\ifL}{b_1}}{\falseL}}{\trueL}})}{\trueL} \twoheadrightarrow_\beta \app{\app{\app{\ifL}{\trueL}}{\falseL}}{\trueL} = \app{\app{\app{(\abs{btf}{\app{\app{b}{t}}{f}})}{\trueL}}{\falseL}}{\trueL}$ \\
\hspace{1.2cm} $\twoheadrightarrow_\beta \app{\app{\trueL}{\falseL}}{\trueL} = \app{\app{(\abs{xy}{x})}{\falseL}}{\trueL} \twoheadrightarrow_\beta \falseL $

\end{frame}

%================================================
\section{\color{section-color} \large Numere naturale}
%================================================

%------------------------------------------------
\begin{frame}{Numere naturale}

Vom reprezenta numerele naturale $\mathbb{N}$ folosind \alert{numeralii Church}.

Numeralul Church pentru numărul $n \in \mathbb{N}$ este notat \intens{$\overline{n}$}.

Numeralul Church $\overline{n}$ este $\lambda$-termenul \intens{$\abs{fx}{\app{f^n}{x}}$}, unde $f^n$ reprezintă compunerea lui $f$ cu ea însăși de $n$ ori:

\begin{center}
\begin{tabular}{rcccl}
\intens{$\overline{0}$} & $\triangleq$ & $\abs{fx}{\app{f^0}{x}}$ & $=$ & $\abs{fx}{x}$ \\
\intens{$\overline{1}$} & $\triangleq$ & $\abs{fx}{\app{f^1}{x}}$ & $=$ & $\abs{fx}{\app{f}{x}}$ \\
\intens{$\overline{2}$} & $\triangleq$ & $\abs{fx}{\app{f^2}{x}}$ & $=$ & $\abs{fx}{\app{f}{(\app{f}{x})}}$ \\
\intens{$\overline{3}$} & $\triangleq$ & $\abs{fx}{\app{f^3}{x}}$ & $=$ & $\abs{fx}{\app{f}{(\app{f}{(\app{f}{x})})}}$ \\
& $\vdots$ && \\
\intens{$\overline{n}$} & $\triangleq$ & $\abs{fx}{\app{f^n}{x}}$ & $=$ & $\abs{fx}{\underbrace{f(f(\ldots(f}_{n}\,x)\ldots))}$ \\
\end{tabular}
\end{center}


\end{frame}

%------------------------------------------------
\begin{frame}{Numere naturale}

\begin{center}
\intens{$\overline{n} \triangleq \abs{fx}{\app{f^n}{x}}$}
\end{center}

\vspace{-.4cm}
Acum putem defini funcția \alert{succesor} prin
\vspace{-.2cm}
\begin{center}
\intens{$\succL \triangleq \abs{nfx}{\app{f}{(\app{\app{n}{f}}{x})}}$}
\end{center}

\vspace{-.2cm}
Observați că $\succL$ pe argumentul $\overline{n}$ returnează o funcție care primește ca argument o funcție $f$, îi aplică $\overline{n}$ pentru a obține compunerea de $n$ ori a lui $f$ cu ea însăși, apoi aplică iar $f$ pentru a obține compunerea de $n+1$ ori a lui $f$ cu ea însăși.

\vspace{-.2cm}
\begin{center}
\begin{tabular}{rcl}
$\app{\succL}{\overline{n}}$ & $=$ & $\app{(\abs{nfx}{\app{f}{(\app{\app{n}{f}}{x})}})}{\overline{n}}$ \\
& $\twoheadrightarrow_\beta$ & $\abs{fx}{\app{f}{(\app{\app{\overline{n}}{f}}{x})}}$ \\
& $\twoheadrightarrow_\beta$ & $\abs{fx}{\app{f}{(\app{f^n}{x})}}$ \\
& $=$ & $\abs{fx}{\app{f^{n+1}}{x}}$ \\
& $=$ & $\overline{n+1}$ 
\end{tabular}
\end{center}

\end{frame}


%------------------------------------------------
\begin{frame}{Numere naturale}

\begin{center}
\intens{$\overline{n} \triangleq \abs{fx}{\app{f^n}{x}}$ \hspace{1cm} $\succL \triangleq \abs{nfx}{\app{f}{(\app{\app{n}{f}}{x})}}$}
\end{center}

\vspace{-.2cm}

Putem face operații aritmetice de bază cu numeralii Church.

Pentru \alert{adunare}, putem defini
\vspace{-.2cm}
\begin{center}
\intens{$\addL \triangleq \abs{mnfx}{\app{\app{m}{f}}{(\app{\app{n}{f}}{x})}}$}
\end{center}

\vspace{-.2cm}
Pentru argumentele $\overline{m}$ și $\overline{n}$, obținem:

\vspace{-.2cm}
\begin{center}
\begin{tabular}{rcl}
$\app{\app{\addL}{\overline{m}}}{\overline{n}}$ & $=$ & $\app{\app{(\abs{mnfx}{\app{\app{m}{f}}{(\app{\app{n}{f}}{x})}})}{\overline{m}}}{\overline{n}}$ \\
& $\twoheadrightarrow_\beta$ & $\abs{fx}{\app{\app{\overline{m}}{f}}{(\app{\app{\overline{n}}{f}}{x})}}$ \\
& $\twoheadrightarrow_\beta$ & $\abs{fx}{\app{f^m}{(\app{f^n}{x})}}$ \\
& $=$ & $\abs{fx}{\app{f^{m+n}}{x}}$ \\
& $=$ & $\overline{m+n}$ 
\end{tabular}
\end{center}

\vspace{-.2cm}
Am folosit compunerea lui $f^m$ cu $f^n$ pentru a obține $f^{m+n}$.
\end{frame}

%------------------------------------------------
\begin{frame}{Numere naturale}

\begin{center}
\intens{$\overline{n} \triangleq \abs{fx}{\app{f^n}{x}}$ \hspace{1cm} $\succL \triangleq \abs{nfx}{\app{f}{(\app{\app{n}{f}}{x})}}$}
\end{center}

\vspace{-.2cm}
Putem defini \alert{adunarea} și ca aplicarea repetată a funcției succesor:
\vspace{-.2cm}
\begin{center}
\intens{$\addL' \triangleq \abs{mn}{\app{\app{m}{\succL}}{n}}$}
\end{center}

\vspace{-.2cm}
\begin{center}
\begin{tabular}{rcl}
$\app{\app{\addL'}{\overline{m}}}{\overline{n}}$ & $=$ & $\app{\app{(\abs{mn}{\app{\app{m}{\succL}}{n}})}{\overline{m}}}{\overline{n}}$  \\
& $\twoheadrightarrow_\beta$ & $\app{\app{\overline{m}}{\succL}}{\overline{n}}$ \\
& $=$ & $\app{\app{(\abs{fx}{\app{f^m}{x}})}{\succL}}{\overline{n}}$ \\
& $\twoheadrightarrow_\beta$ & $\app{\succL^m}{\overline{n}}$ \\
& $=$ & $\underbrace{\succL(\succL(\ldots(\succL}_{m}\,\overline{n})\ldots))$ \\
& $\twoheadrightarrow_\beta$ & $\underbrace{\succL(\succL(\ldots(\succL}_{m-1}\,\overline{n+1})\ldots))$ \\
& $\twoheadrightarrow_\beta$ & $\overline{m+n}$
\end{tabular}
\end{center}

\end{frame}

%------------------------------------------------
\begin{frame}{Numere naturale}

\begin{center}
\intens{$\overline{n} \triangleq \abs{fx}{\app{f^n}{x}}$ \hspace{1cm} $\succL \triangleq \abs{nfx}{\app{f}{(\app{\app{n}{f}}{x})}}$} \\[.6em]
\intens{$\addL' \triangleq \abs{mn}{\app{\app{m}{\succL}}{n}}$}
\end{center}

\vspace{-.2cm}
Similar \alert{înmulțirea} este adunare repetată, iar ridicarea la putere este înmulțire repetată:
\vspace{-.2cm}
\begin{center}
\begin{tabular}{rcl}
\intens{$\mulL$} & \intens{$\triangleq$} & \intens{$\abs{mn}{\app{\app{m}{(\app{\addL}{n})}}{\overline{0}}}$}  \\[.6em]
\intens{$\expL$} & \intens{$\triangleq$} & \intens{$\abs{mn}{\app{\app{m}{(\app{\mulL}{n})}}{\overline{1}}}$}  
\end{tabular}
\end{center}

\end{frame}

%------------------------------------------------
\begin{frame}{Numere naturale}

Putem defini o funcție de la numere naturale la booleeni care verifică dacă un număr natural este $0$ sau nu
\begin{center}
\begin{tabular}{rcl}
$\isZero(0)$ & $=$ & $true$ \\
$\isZero(n)$ & $=$ & $false$ \hspace{.4cm} dacă $n \neq 0$ \\
\end{tabular}
\end{center}
O codare în lambda calcul a unei astfel de funcții este
\begin{center}
\intens{$\isZero \triangleq  \abs{nxy}{\app{\app{n}{(\abs{z}{y})}}{x}}$}
\end{center}

\intens{\textbf{Exercițiu.}}  Verificați afirmația de mai sus.

\medskip
Putem să definim și codarea  \intens{$\pred$} pentru predecesorul unui număr natural. Această codare nu este deloc ușoară și alegem să lucrăm cu ea ca cu o cutie neagră.
\end{frame}

%%------------------------------------------------
%\begin{frame}{Perechi}
%
%Vrem funcții pentru perechi  și  proiecții care să respecte legile:
%\vspace{-.2cm}
%\begin{center}
%\begin{tabular}{l}
%$\app{\firstL}{(\app{\app{\pairL}{e_1}}{e_2})} = e_1$ \\ 
%$\app{\secondL}{(\app{\app{\pairL}{e_1}}{e_2})} = e_2$ \\
% $\app{\app{\pairL}{(\app{\firstL}{p})}}{(\app{\secondL}{p})} = p$,
% \end{tabular}
%\end{center}
%\vspace{-.2cm}
%dacă $p$ este o pereche.
%
%Definim \intens{$\pairL \triangleq \abs{abf}{\app{\app{f}{a}}{b}}$}.
%
%Atunci $\app{\app{\pairL}{e_1}}{e_2} \twoheadrightarrow_\beta \abs{f}{\app{\app{f}{e_1}}{e_2}}$. Pentru a obține $e_1$, putem aplica acest termen lui $\trueL$:
%\vspace{-.2cm}
%\[
%\app{(\abs{f}{\app{\app{f}{e_1}}{e_2}})}{\trueL} \twoheadrightarrow_\beta \app{\app{\trueL}{e_1}}{e_2} = \app{\app{(\abs{xy}{x})}{e_1}}{e_2}  \twoheadrightarrow_\beta e_1
%\]
%
%\vspace{-.2cm}
%În concluzie, putem defini \intens{$\firstL \triangleq \abs{p}{\app{p}{\trueL}}$} și \intens{$\secondL \triangleq \abs{p}{\app{p}{\falseL}}$}.
%
%În definițiile de mai sus, dacă $p$ nu este un termen de forma $\app{\app{\pairL}{a}}{b}$, nu putem garanta nimic!
%
%
%\end{frame}

%------------------------------------------------
\begin{frame}{Putem exprima mai mult?}

Avem văzut  codari simple pentru booleeni și numere naturale. 

Totuși nu avem o metodă pentru a construi astfel de $\lambda$-termeni.

Ne trebuie un mecanism care să ne permită să construim funcții mai complicate din funcții mai simple.

De exemplu, să considerăm funcția factorial
\vspace{-.2cm}
\begin{center}
\begin{tabular}{rcl}
$0!$ & $=$ & $1$ \\
$n!$ & $=$ & $n \cdot (n-1)!$, \hspace{.4cm} dacă $n \neq 0$ 
\end{tabular}
\end{center}

\end{frame}

%================================================
\section{\color{section-color} \large Puncte fixe}
%================================================

%------------------------------------------------
\begin{frame}{Puncte fixe}

Fie $f$ o funcție. 
%
Spunem că $x$ este un \alert{punct fix} al lui $f$ dacă \intens{f(x) = x}.

În matematică, unele funcții au puncte fixe, altele nu au.

De exemplu, $f(x) = x^2$ are două puncte fixe $0$ și $1$, dar $f(x) = x+1$ nu are puncte fixe.

Mai mult, unele funcții au o infinitate de puncte fixe, cum ar fi $f(x)=x$.

\end{frame}

%------------------------------------------------
\begin{frame}{$\beta$-echivalență}


Am notat cu \intens{$M \twoheadrightarrow_\beta M'$} faptul că $M$ poate fi $\beta$-redus până la $M'$ în $0$ sau mai mulți pași  de $\beta$-reducție.

\intens{$\twoheadrightarrow_\beta$}  este închiderea reflexivă și tranzitivă a relației $\rightarrow_\beta$.

\medskip
Notăm cu \intens{$M =_\beta M'$} faptul că $M$ poate fi transformat în $M'$ în $0$ sau mai mulți pași de $\beta$-reducție, transformare în care pașii de reducție pot fi  și întorși. 

 \intens{$=_\beta$} este închiderea reflexivă, simetrică și tranzitivă a relației $\rightarrow_\beta$.

De exemplu, avem \intens{$\app{(\abs{y}{\app{y}{v}})}{z} =_\beta \app{(\abs{x}{\app{z}{x}})}{v}$} deoarece avem
\vspace{-.2cm}
\begin{center}
$\app{(\abs{y}{\app{y}{v}})}{z} \rightarrow_\beta \app{z}{v} \leftarrow_\beta \app{(\abs{x}{\app{z}{x}})}{v}$
\end{center}
\vspace{-.2cm}
Notăm cu $\leftarrow_\beta$ inversul relației $\rightarrow_\beta$.

\end{frame}

%------------------------------------------------
\begin{frame}{Puncte fixe în lambda-calcul}

Dacă $F$ și $M$ sunt $\lambda$-termeni, spunem că $M$ este un \alert{punct fix} al lui $F$ dacă \intens{$\app{F}{M} =_\beta M$}.

\medskip
\alert{\textbf{Thm.} În lambda calculul fără tipuri, orice termen are un punct fix.}

\pause
\smallskip \textbf{Dem.} Vrem să arătăm că pentru orice termen $F$ există un termen $M$ astfel încât $\app{F}{M} =_\beta M$.

Fie $F$ un termen. Considerăm  \intens{$M \triangleq \app{(\abs{x}{\app{F}{(\app{x}{x})}})}{(\abs{x}{\app{F}{(\app{x}{x})}})}$}. Avem
\vspace{-.2cm}
\begin{center}
\begin{tabular}{rcl}
$M$ & $=$ & $\app{(\abs{x}{\app{F}{(\app{x}{x})}})}{(\abs{x}{\app{F}{(\app{x}{x})}})}$ \\
& $\rightarrow_\beta$ & $\app{F}{(\app{(\abs{x}{\app{F}{(\app{x}{x})}})}{(\abs{x}{\app{F}{(\app{x}{x})}})})}$ \\
& $=$ & $\app{F}{M}$
\end{tabular}
\end{center}
\vspace{-.2cm}
Deci avem $\app{F M} =_\beta M$.
\end{frame}

%------------------------------------------------
\begin{frame}{Combinatori de punct fix}

\alert{Combinatorii de puncte fixe} sunt termeni închiși care "construiesc" un punct fix pentru un termen arbitrar. 

\medskip
Câteva exemple:
\begin{itemize}
	\item  \intens{Combinatorul de punct fix al lui Curry} \intens{$\mathbf{Y} \triangleq \abs{y}{\app{(\abs{x}{\app{y}{(\app{x}{x})}})}{(\abs{x}{\app{y}{(\app{x}{x})}})}}$} 

Pentru orice termen $F$, $\app{\mathbf{Y}{F}}$ este un punct fix al lui $F$ deoarece $\app{\mathbf{Y}{F}} \twoheadrightarrow_\beta \app{F}{(\app{\mathbf{Y}}{F})}$.

	\smallskip
	\item \intens{Combinatorul de punct fix al lui Turing} \intens{$\Theta \triangleq \app{(\abs{xy}{\app{y}{(\app{\app{x}{x}}{y})}})}(\abs{xy}{\app{y}{(\app{\app{x}{x}}{y})}}) $}
	
	Pentru orice termen $F$, $\app{\Theta{F}}$ este un punct fix al lui $F$ deoarece $\app{\Theta{F}} \twoheadrightarrow_\beta \app{F}{(\app{\Theta}{F})}$.
\end{itemize}
\end{frame}

%------------------------------------------------
\begin{frame}{Rezolvarea de ecuații în lambda calcul}

Punctele fixe ne permit să rezolvăm ecuații. A găsi un punct fix pentru $f$ este același lucru cu a rezolva o ecuație de forma
\vspace{-.2cm}
\begin{center}
$x = f(x)$
\end{center}
\vspace{-.2cm}

Am văzut că în lambda calcul există mereu o soluție pentru astfel de ecuații.
\end{frame}

%------------------------------------------------
\begin{frame}{Rezolvarea de ecuații în lambda calcul}

Să aplicăm această idee pentru funcția factorial. 

Cea mai naturală definiție a funcției factorial este cea recursivă și o putem scrie în lambda calcul prin
\begin{center}
\intens{$\app{\factL}{n} = \app
					{\ifL}
					{\app
						{(\app{\isZero}{n})}
						{\app{(\overline{1})}{(\app{\mulL}{\app{n}{(\app{\factL{(\app{\pred}{n})})}})}}}
					}
					$}
\end{center}

În ecuația de mai sus, $\factL$ apare și în stânga, și în dreapta. Pentru a găsi cine este $\factL$, trebuie să rezolvăm o ecuație.
\end{frame}

%------------------------------------------------
\begin{frame}{Rezolvarea de ecuații în lambda calcul}

Să rezolvăm ecuația de mai sus. Rescriem problema puțin

\begin{center}
\begin{tabular}{rcl}
$\factL$ & $=$ & $\abs{n}
			  {\app
					{\ifL}
					{\app
						{(\app{\isZero}{n})}
						{\app{(\overline{1})}{(\app{\mulL}{\app{n}{(\app{\factL{(\app{\pred}{n})})}})}}}
					}
			  }
			$ \\
$\factL$ & $=$ & $\app{
			(\abs{fn}
			  {\app
					{\ifL}
					{\app
						{(\app{\isZero}{n})}
						{\app{(\overline{1})}{(\app{\mulL}{\app{n}{(\app{f{(\app{\pred}{n})})}})}}}
					}
			  })}\factL
			$ \\
\end{tabular}
\end{center}
Notăm termenul $\abs{fn}
			  {\app
					{\ifL}
					{\app
						{(\app{\isZero}{n})}
						{\app{(\overline{1})}{(\app{\mulL}{\app{n}{(\app{f{(\app{\pred}{n})})}})}}}
					}
			  }$ cu $F$.
			  
Ultima ecuație devine $\factL = \app{F}\factL$, o ecuație de punct fix.	  

Am văzut că $\app{\mathbf{Y}}{F}$ este un punct fix pentru $F$ (adică $\app{\mathbf{Y}}{F} \twoheadrightarrow_\beta \app{F}{(\app{\mathbf{Y}}{F})}$), \\de aceea putem rezolva ecuația de mai sus luând 
\begin{center}
\begin{tabular}{rcl}
\intens{$\factL$} & \intens{$\triangleq$} & \intens{$\app{\mathbf{Y}}{F}$}  \\
\intens{$\factL$} & \intens{$\triangleq$} & \intens{$\app{\mathbf{Y}}{(\abs{fn}
			  {\app
					{\ifL}
					{\app
						{(\app{\isZero}{n})}
						{\app{(\overline{1})}{(\app{\mulL}{\app{n}{(\app{f{(\app{\pred}{n})})}})}}}
					}
			  })}$}   
\end{tabular}
\end{center}

Observați că $\factL$ a dispărut din partea dreaptă.

\intens{\textbf{Exercițiu.}} Evaluați $\app{\factL}{\overline{2}}$ ținând cont că $\factL \twoheadrightarrow_\beta \app{F}{\factL}$.
\end{frame}



%------------------------------------------------
\begin{frame}
  \vfill
  \centering

\textbf{\large \alert{Quiz time!}}

\includegraphics[scale=.35]{../Quiz/C04-Q1.png}

 \url{https://tinyurl.com/C04-Quiz1}
  \vfill
\end{frame}

%---------------------------------------------
\begin{frame}
  \vfill
  \centering

\textbf{Pe săptămâna viitoare!}

  \vfill
\end{frame}
\end{document}







