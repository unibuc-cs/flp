%\documentclass[handout,xcolor=x11names,compress,10pt]{beamer}

\documentclass[xcolor=x11names,compress,10pt]{beamer}
%\documentclass[xcolor=pdftex,romanian,colorlinks]{beamer}

\usepackage{../tslides}
%\usepackage{comment}

%% General document %%%%%%%%%%%%%%%%%%%%%%%%%%%%%%%%%%


%\newcommand{\plus}[1] {{\color{Green4} #1}}
%\newcommand{\intens}[1] {{\color{DeepSkyBlue3} #1}}
%\newcommand{\egf}[1]{\stackrel{\cdot}{=}_{#1}}
%\newcommand{\regR}{\mbox{R}}
%\newcommand{\regS}{\mbox{S}}
%\newcommand{\regT}{\mbox{T}}
%\newcommand{\regCS}{\mbox{C}\Sigma}
%\newcommand{\regSub}{\mbox{Sub}_\Gamma}
%\newcommand{\te}[1]{\mbox{\texttt{#1}}}
\newcommand{\Conf}[2]{\ensuremath{\langle #1\ ,\ #2\rangle}}
%\newcommand{\cc}[1]{\ensuremath{\langle #1 \rangle}}
%
%
%
%\newcommand{\parrow}{\rightharpoonup}
%\makeatletter
%\newcommand{\vo}{\vec{o}\@ifnextchar{^}{\,}{}}
%\makeatother
\renewcommand{\to}{}
%=========================================

\begin{document}
\title{\\Curs 10}
\author{Fundamentele limbajelor de programare} 
\date{2020-2021} 


\frame{\titlepage} 


\frame{\frametitle{Cuprins}\tableofcontents

} 

\begin{section}{Semantica Small-Step pentru Lambda Calcul}

  \begin{frame}[fragile]{Sintaxa limbajului LAMBDA}
    \begin{block}{BNF}
    \[
    \begin{array}{rcl}
      e & ::= & {\Id} \Smid {\Int} \Smid {\Strue} \Smid {\Sfalse} \\
        & \Smid & {e + e} \Smid {e < e} \Smid { \terminal{not}(e) } \\
        & \Smid & \Sif e \Sthen e \Selse e \\
        & \Smid & {\lambda \Id . e} \Smid {e \; e} \\
        & \Smid & \terminal{let} \Id \terminal{=} e \terminal{in} e
    \end{array}
    \]
    \end{block}

    \begin{block}{Verificarea sintaxei în Prolog}
      \begin{asciipl}
exp(Id) :- atom(Id).                      % identifier
exp(Lit) :- Lit = true ; Lit = false ; integer(Lit).
exp(E1 + E2) :- exp(E1), exp(E2).         % same for any op
exp(if(E1, E2, E3)) :- exp(E1), exp(E2), exp(E3).
exp(Id -> Exp) :- atom(Id), exp(Exp).     % lambda
exp(Exp1 $ Exp2) :- exp(Exp1), exp(Exp2). % application
exp(let(Id, Exp1, Exp2)) :- atom(Id), exp(Exp1), exp(Exp2).
      \end{asciipl}
    \end{block}
 \end{frame}

\begin{frame}[fragile]{Semantica small-step pentru Lambda}
\vspace*{0.5cm}
\begin{itemize}
  \item Definește cel mai mic pas de execuție ca o relație de tranziție  \^{\i}ntre expresii
    dată fiind o stare cu valori pentru variabilele libere
  \begin{center}
	$\rho \vdash \Ss{cod}{cod\,'}$ \hfill\intens{\texttt{step(Env, Cod1,Cod2)}}
  \end{center}
%  \item Fiecare pas de execuție este concluzia unei demonstrații

  \item Execuția se obține ca o succesiune de astfel de tranziții.
\end{itemize}
\end{frame}

\begin{frame}[fragile]{Semantica variabilelor}
	\[\reg{\rho \vdash \Ss{x}{v}}{}{\rho(x) = v}\]
  \begin{block}{Prolog}
    \begin{asciipl}
      step(Env, X, V) :- atom(X), get(Env, X, V).
    \end{asciipl}
  \end{block}

\end{frame}

  
\begin{frame}[fragile]{Semantica expresiilor aritmetice}
    \begin{itemize}
    \item \intens{Semantica adunării a două expresii aritmetice}	
     \item[] $\reg{\Ss{\Conf{i_1 + i_2}{\sigma}}{\Conf{i}{\sigma}}}{}{i = i_1 + i_2}$
    
    \item[] $\reg{\Ss{\Conf{a_1 + a_2}{\sigma}}{\Conf{a_1' + a_2}{\sigma'}}}{\Ss{\Conf{a_1}{\sigma}}{\Conf{a_1'}{\sigma'}}}{}$\hfill$\reg{\Ss{\Conf{a_1 + a_2}{\sigma}}\to{\Conf{a_1 + a_2'}{\sigma'}}}{\Ss{\Conf{a_2}{\sigma}}\to {\Conf{a_2'}{\sigma'}}}{}$
 

    \item Pentru alți operatori (artimetici, de comparație, booleeni, condițional)
    \begin{itemize}
        \item  Similar cu regulile din IMP
    \end{itemize}
    \end{itemize}
    
   \begin{block}{Prolog} 
  \begin{verbatim}
  step(_, I1 + I2,I):- integer(I1),integer(I2), 
                       I is I1 + I2.
                              
  step(Env, AE + AE1,AE + AE2):- step(Env, AE1,AE2).
  
  step(Env, AE1 + AE,AE2 + AE):- step(Env, AE1,AE2).
  \end{verbatim}
  \end{block}
  \end{frame}
  


\begin{frame}[fragile]{Semantica $\lambda$-abstracției}
	\[\reg{\rho \vdash \Ss{\lambda x. e}{{\it closure}(x, e, \rho)}}{}{}\]

  $\lambda$-abstracția se evaluează la o valoare specială numită closure
  care capturează valorile curente ale variabilelor pentru a se putea
  executa în acest mediu atunci când va fi aplicată.

  \begin{block}{Prolog}
    \begin{asciipl}
      step(Env, X -> E, closure(X, E, Env)).
    \end{asciipl}
  \end{block}

\end{frame}

\begin{frame}[fragile]{Semantica construcției $\terminal{let}$}
	\[\reg{\rho \vdash \Ss{\terminal{let} x \terminal{=} e_1 \terminal{in} e_2}{(\lambda x . e2)\; e_1}}{}{}\]

  A îi da lui $x$ valoarea lui $e_1$ în $e_2$ este același lucru cu a
  aplica funcția de $x$ cu corpul $e_2$ expresiei $e_1$.

  \begin{block}{Prolog}
    \begin{asciipl}
      step(_, let(X, E1, E2), (X -> E2) $ E1).
    \end{asciipl}
  \end{block}

\end{frame}


\begin{frame}[fragile]{Semantica operatorului de aplicare}
	\[\reg{\rho \vdash \Ss{{\it closure}(x, e, \rho_e) \; v}{{\it closure}(x, e', \rho_e) \; v}}{\rho_e[v/x]\vdash \Ss{e}{e'}}{v \mbox{ valoare}}\]

	\[\reg{\rho \vdash \Ss{{\it closure}(x, v, \rho_e) \; e}{v}}{}{v \mbox{ valoare}}\]

  \[\reg{\rho \vdash \Ss{e_1 \; e_2}{e'_1 \; e_2}}{\rho \vdash \Ss{e_1}{e'_1}}{}
    \hfill
    \reg{\rho \vdash \Ss{e_1 \; e_2}{e_1 \; e'_2}}{\rho \vdash \Ss{e_2}{e'_2}}{}
    \]

  \begin{block}{Prolog}
    \begin{asciipl}
      step(Env, E $ E1, E $ E2) :- step(Env, E1, E2).
      step(Env, E1 $ E, E2 $ E) :- step(Env, E1, E2).
      step(Env, closure(X, E, EnvE) $ V, Result) :-
          \+ step(Env, V, _),
          set(EnvE, X, V, EnvEX),
          step(EnvEX, E, E1)
          ->  Result = closure(X, E1, EnvE) $ V
          ;   Result = E.
    \end{asciipl}
  \end{block}

\end{frame}

\end{section}
\begin{section}{Determinarea tipurilor}

\begin{frame}{Problemă: Sintaxa este prea permisivă}
    \begin{block}{\alert{Problemă:} Mulți termeni acceptați de sintaxă nu pot fi evaluați}
        \begin{itemize}
        \item $2\;(\fun{x}{x})$\only<2>{--- expresia din stânga aplicației trebuie să reprezinte o funcție}
         \item $(\fun{x}{x}) + 1$\only<2>{  --- adunăm funcții cu numere}
        \item $(\fun{x}{x + 1})\; (\fun{x}{x})$\only<2>{ --- pot face o reducție, dar tot nu pot evalua}
        \end{itemize}
     \end{block}
     \onslide<2>
     \vfill\begin{block}{Soluție:  Identificarea (precisă) a programelor corecte}
     \begin{itemize}
     \item Definim tipuri pentru fragmente de program corecte (e.g., int, bool)
     \item Definim (recursiv) o relație care să lege fragmente de program de tipurile asociate
     \[((\fun{x}{x + 1})\; ((\fun{x}{x})\; 3)) : \mbox{int}\]
     \end{itemize}
     \end{block}
\end{frame}


    \begin{subsection}{Asociere de tipuri}
    \begin{frame}{Relația de asociere de tipuri}
    Definim (recursiv) o relație de forma $\T{\Gamma}{e}{\tau}$, unde
    \begin{itemize}
    \item $\tau$ este un tip
    
    \vspace{-2ex}
    \begin{syntaxBlock}{\structure{\tau}}
    \color<beamer|handout>{black!25!white}{
    \renewcommand{\syntaxKeyword}{}
    \syntax{int}{întregi}
    \syntaxCont{bool}{valori de adevăr}
    \syntaxCont{\structure{\tau} \rightarrow \structure{\tau}}{funcții}
    \syntaxCont{\structure{a}}{variabile de tip}
    }
    \vspace{-4ex}
    \end{syntaxBlock}
    \item $e$ este un termen (potențial cu variabile libere)
    \item $\Gamma$ este \structure{mediul de tipuri}, o funcție parțială finită care asociază tipuri variabilelor (libere ale lui $e$)
    \item Variabilele de tip sunt folosite pentru a indica polimorfismul
    \end{itemize}
    
    \begin{block}{Cum citim $\T{\Gamma}{e}{\tau}$?}
    Dacă variabila $x$ are tipul $\Gamma(x)$ pentru orice $x\in dom(\Gamma)$,\\ atunci termenul $e$ are tipul $\tau$.
    \end{block}
    \end{frame}
    
    \begin{frame}{Axiome}
    \begin{itemize}
    \item[] $\reg[:var]{\T{\Gamma}{x}{\tau}}{}{\Gamma(x) = \tau}$
    \vitem[] $\reg[:int]{\T{\Gamma}{n}{int}}{}{\mbox{$n$ întreg}}$
    \vitem[] $\reg[:bool]{\T{\Gamma}{b}{bool}}{}{b = \Strue \mathrel{or} b = \Sfalse}$
    \end{itemize}
    \end{frame}
    
    
    \begin{frame}
    {Expresii}
    \begin{itemize}
    \item[]
    $\reg[:iop]{\T{\Gamma}{e_1\mathrel{o} e_2}{int}}{\T{\Gamma}{e_1}{int} \si \T{\Gamma}{e_2}{int}}{o\in \{+,-,*,/\}}$
    
    \vitem[]
    $\reg[:cop]{\T{\Gamma}{e_1\mathrel{o} e_2}{bool}}{\T{\Gamma}{e_1}{int} \si \T{\Gamma}{e_2}{int}}{o\in \{\leq, \geq, <, >, =\}}$
    
    \vitem[]
    $\reg[:bop]{\T{\Gamma}{e_1\mathrel{o} e_2}{bool}}{\T{\Gamma}{e_1}{bool} \si \T{\Gamma}{e_2}{bool}}{o\in \{\Sand, \Sor\}}$
    
    \vitem[]
    $\reg[:if]{\T{\Gamma}{\Sif e_b \Sthen e_1 \Selse e_2}{\tau}}{\T{\Gamma}{e_b}{bool} \si \T{\Gamma}{e_1}{\tau} \si \T{\Gamma}{e_2}{\tau}}{}$
    \end{itemize}
    \end{frame}
    
    
    \begin{frame}{Fragmentul funcțional}
    \begin{itemize}
    \item[] 
    $\displaystyle\reg[:fn]{\T{\Gamma}{\fun{x}{e}}{\tau \rightarrow \tau'}}{\T{\Gamma'}{e}{\tau'}}{\Gamma' = \Gamma[x\mapsto \tau]}$
    
    \vitem[]
    $\displaystyle\reg[:app]{\T{\Gamma}{e_1\; e_2}{\tau}}{\T{\Gamma}{e_1}{\tau' \rightarrow \tau} \si \T{\Gamma}{e_2}{\tau'}}{}$
    \end{itemize}
    \end{frame}
    
    \end{subsection}
    
    \begin{subsection}{Proprietăți}
    
    \begin{frame}{Programe în execuție}
     \begin{alertblock}{Problemă:}
     \begin{itemize}
       \item În timpul execuției programul conține valori de tip closure
       \item Care este tipul lor?
     \end{itemize}
     \end{alertblock} 
    
     \begin{block}{Soluție}
     \begin{itemize}
     \item Adăugam regula $\reg[:cl]{\T{\Gamma}{{\it closure}(x, e, \rho)}{\tau}}{\T{\Gamma_\rho}{\lambda x. e}{\tau}}{} {\it unde}$ 
     \item Mediul de tipuri $\Gamma_\rho$ asociat unui mediu de execuție $\rho$ satisface:
     \begin{itemize}
     \item $\Dom(\Gamma_\rho) = \Dom(\rho)$
     \item Pentru orice variabilă $x\in \Dom(\rho)$, există $\tau$ tip și $v$ valoare astfel încât $\Gamma_\rho(x) = \tau$, $\rho(x) = v$ și $\T{}{v}{\tau}$
     \end{itemize}
     \end{itemize}
     \end{block}
    \end{frame}
    
    \begin{frame}{Proprietăți}{ale sistemului de tipuri pentru limbajul $\Sfun$-IMP}
    \begin{theorem}[Proprietatea de a progresa] Dacă $\T{\Gamma_\rho}{e}{\tau}$ atunci $e$ este valoare sau $e$ poate progresa în $\rho$: există $e'$ astfel încât $\rho \vdash \Ss{e}{e'}$.
    \end{theorem}
    \vfill
    
    \begin{theorem}[Proprietatea de conservare a tipului]
    Dacă $\T{\Gamma_\rho}{e}{\tau}$ și $\rho\vdash \Ss{e}{e'}$, atunci 
    $\T{\Gamma_\rho'}{e'}{\tau}$.
    \end{theorem}
    
    \vfill
    \begin{theorem}[Siguranță---programele bine formate nu se împotmolesc]
    Dacă $\T{\Gamma_\rho}{e}{\tau}$ și $\rho\vdash {e}\longrightarrow^\ast{e'}$, atunci $e'$ este valoare sau există $e''$, astfel încât $\rho \vdash \Ss{e'}{e''}$.
    \end{theorem}
    
    \end{frame}
    
    \begin{frame}{Probleme computaționale}
    \begin{block}{Verificarea tipului}
    Date fiind $\Gamma$, $e$ și $\tau$, verificați dacă $\tjud{e}{\tau}$.
    \end{block}
    
    \begin{block}{Determinarea (inferarea) tipului}
    Date fiind $\Gamma$ și $e$, găsiți (sau arătați ce nu există) un $\tau$ astfel încât $\tjud{e}{\tau}$.
    \end{block}
    
    \begin{itemize}
    \item A doua problemă e mai grea în general decât prima
    \item Algoritmi de inferare a tipurilor
    \begin{itemize}
    \item Colectează constrângeri asupra tipului
    \item Folosesc metode de rezolvare a constrângerilor (programare logică)
    \end{itemize}
    \end{itemize}
    \end{frame}
    
    \begin{frame}{Probleme computaționale}{Proprietăți}
    \begin{theorem}[Determinarea tipului este decidabilă]
    Date fiind $\Gamma$ și $e$, poate fi găsit (sau demonstrat că nu există) un $\tau$ astfel încât 
    $\tjud{e}{\tau}$.
    \end{theorem}
    \vfill
    
    \begin{theorem}[Verificarea tipului este decidabilă]
    Date fiind $\Gamma$, $e$ și $\tau$, problema $\tjud{e}{\tau}$ este decidabilă.
    \end{theorem}
    
    \vfill
    \begin{theorem}[Unicitatea tipului]
    Dacă $\tjud{e}{\tau}$ și $\tjud{e}{\tau'}$, atunci $\tau=\tau'$.
    \end{theorem}
    \end{frame}
    
    \end{subsection}

    \begin{subsection}{Exemplu}
      \begin{frame}{Exemplu}
        \begin{block}{Care este tipul expresiei următoare (dacă are)}
          \[ \lambda x.\lambda y. \lambda z. \Sif {y = 0} \Sthen z \Selse {x / y}\]
        \end{block}

        \begin{block}{Aplicăm regula 
    $\displaystyle\reg[:fn]{\T{\Gamma}{\fun{x}{e}}{\tau \rightarrow \tau'}}{\T{\Gamma'}{e}{\tau'}}{\Gamma' = \Gamma[x\mapsto \tau]}$
    }
          $\vdash  \lambda x.\lambda y. \lambda z. \Sif {y = 0} \Sthen z \Selse {x / y} : t_x \rightarrow t$ dacă
          $x \mapsto t_x \vdash \lambda y. \lambda z. \Sif {y = 0} \Sthen z \Selse {x / y} : t$

          \pause Mai departe:
          $x \mapsto t_x \vdash \lambda y. \lambda z. \Sif {y = 0} \Sthen z \Selse {x / y} : t_y \rightarrow t_0$ dacă
          $x \mapsto t_x, y\mapsto t_y \vdash \lambda z. \Sif {y = 0} \Sthen z \Selse {x / y} : t_0$ și, de mai sus,
          $t = t_y \rightarrow t_0$

          \pause Mai departe: 
          $x \mapsto t_x, y\mapsto t_y \vdash \lambda z. \Sif {y = 0} \Sthen z \Selse {x / y} : t_z\rightarrow t_1$ dacă
          $x \mapsto t_x, y\mapsto t_y, z \mapsto t_z \vdash \Sif {y = 0} \Sthen z \Selse {x / y} : t_1$ și, de mai sus,
          $t_0 = t_z \rightarrow t_1$
        \end{block}
     
      \end{frame}

      \begin{frame}{Exemplu}
      \begin{block}{Unde suntem}
        $\vdash  \lambda x.\lambda y. \lambda z. \Sif {y = 0} \Sthen z \Selse {x / y} : t_x \rightarrow t$ dacă
        $x \mapsto t_x, y\mapsto t_y, z \mapsto t_z \vdash \Sif {y = 0} \Sthen z \Selse {x / y} : t_1$ și
        $t_0 = t_z \rightarrow t_1$, $t = t_y \rightarrow t_0$.
      \end{block}
        \begin{block}{Aplicăm regula 
    $\reg[:if]{\T{\Gamma}{\Sif e_b \Sthen e_1 \Selse e_2}{\tau}}{\T{\Gamma}{e_b}{bool} \si \T{\Gamma}{e_1}{\tau} \si \T{\Gamma}{e_2}{\tau}}{}$
    }
        $x \mapsto t_x, y\mapsto t_y, z \mapsto t_z \vdash \Sif {y = 0} \Sthen z \Selse {x / y} : t_1$ dacă

        $x \mapsto t_x, y\mapsto t_y, z \mapsto t_z \vdash {y = 0} : \Sbool$ și
        $x \mapsto t_x, y\mapsto t_y, z \mapsto t_z \vdash {z} : t_1$ și
        $x \mapsto t_x, y\mapsto t_y, z \mapsto t_z \vdash {x / y} : t_1$


        \end{block}
      \end{frame}
      \begin{frame}{Exemplu}
        \begin{block}{Aplicăm regula 
    $\reg[:cop]{\T{\Gamma}{e_1\mathrel{o} e_2}{bool}}{\T{\Gamma}{e_1}{int} \si \T{\Gamma}{e_2}{int}}{o\in \{\leq, \geq, <, >, =\}}$
        }
        $x \mapsto t_x, y\mapsto t_y, z \mapsto t_z \vdash {y = 0} : \Sbool$ dacă

        $x \mapsto t_x, y\mapsto t_y, z \mapsto t_z \vdash y : \Sint$ \hfill și \hfill
        $x \mapsto t_x, y\mapsto t_y, z \mapsto t_z \vdash 0 : \Sint$
        \end{block}

        \begin{block}{Aplicăm regula 
     $\reg[:int]{\T{\Gamma}{n}{int}}{}{\mbox{$n$ întreg}}$
        }

        $x \mapsto t_x, y\mapsto t_y, z \mapsto t_z \vdash 0 : \Sint$ este adevărat
        \end{block}

        \begin{block}{Aplicăm regula 
    $\reg[:iop]{\T{\Gamma}{e_1\mathrel{o} e_2}{int}}{\T{\Gamma}{e_1}{int} \si \T{\Gamma}{e_2}{int}}{o\in \{+,-,*,/\}}$
        }

        $x \mapsto t_x, y\mapsto t_y, z \mapsto t_z \vdash {x / y} : \Sint$ dacă

        $x \mapsto t_x, y\mapsto t_y, z \mapsto t_z \vdash x : \Sint$ \hfill și \hfill
        $x \mapsto t_x, y\mapsto t_y, z \mapsto t_z \vdash y : \Sint$

        și, de mai sus, $t_1 = \Sint$
        \end{block}
        
      \end{frame}

      \begin{frame}{Exemplu}
      \begin{block}{Recapitulăm}
        $\vdash  \lambda x.\lambda y. \lambda z. \Sif {y = 0} \Sthen z \Selse {x / y} : t_x \rightarrow t$ dacă

        $x \mapsto t_x, y\mapsto t_y, z \mapsto t_z \vdash y : \Sint$ \hfill și \hfill
        $x \mapsto t_x, y\mapsto t_y, z \mapsto t_z \vdash {z} : t_1$

        $x \mapsto t_x, y\mapsto t_y, z \mapsto t_z \vdash x : \Sint$ \hfill și \hfill
        $x \mapsto t_x, y\mapsto t_y, z \mapsto t_z \vdash y : \Sint$

        și
        $t_0 = t_z \rightarrow t_1$, $t = t_y \rightarrow t_0$, $t_1 = \Sint$.
      \end{block}
        \begin{block}{Aplicăm regula 
     $\reg[:var]{\T{\Gamma}{x}{\tau}}{}{\Gamma(x) = \tau}$
    }

    $x \mapsto t_x, y\mapsto t_y, z \mapsto t_z \vdash y : t_y$ adevărat și, de mai sus $t_y = \Sint$

    $x \mapsto t_x, y\mapsto t_y, z \mapsto t_z \vdash {z} : t_z$ adevărat și, de mai sus, $t_1 = t_z$

    $x \mapsto t_x, y\mapsto t_y, z \mapsto t_z \vdash x : t_x$ adevărat și, de mai sus, $t_x = \Sint$

        \end{block}
      \end{frame}

      \begin{frame}{Exemplu}
      \begin{block}{Finalizăm}
        $\vdash  \lambda x.\lambda y. \lambda z. \Sif {y = 0} \Sthen z \Selse {x / y} : t_x \rightarrow t$ dacă

        $t_0 = t_z \rightarrow t_1$, $t = t_y \rightarrow t_0$, $t_1 = \Sint$, $t_y = \Sint$,
        $t_1 = t_z$ și $t_x = \Sint$.

      \end{block}
       \begin{block}{Rezolvăm constrângerile și obținem}

        $\vdash  \lambda x.\lambda y. \lambda z. \Sif {y = 0} \Sthen z \Selse {x / y} : {\Sint} \rightarrow {\Sint} \rightarrow {\Sint} \rightarrow {\Sint}$

        \end{block}
      \end{frame}

      
    \end{subsection}

    \begin{subsection}{Implementare în Prolog}



    \begin{frame}{Relația de asociere de tipuri în Prolog}
    Definim (recursiv) o relație de forma {\tt type(Gamma, E, T)}, unde
    \begin{itemize}
      \item {\tt Gamma} este o listă de perechi de forma {\tt (X, T)} unde
       {\tt X} este un identificator și {\tt T} este o expresie de tip cu variabile
      \item {\tt E} este o $\lambda$-expresie scrisă cu sintaxa descrisă mai sus
      \item {\tt T} este o expresie de tip cu variabile
    \end{itemize}
    \end{frame}

  \begin{frame}[fragile]{Sintaxa limbajului LAMBDA}
    \begin{block}{BNF}
    \[
    \begin{array}{rcl}
      e & ::= & {\Id} \Smid {\Int} \Smid {\Strue} \Smid {\Sfalse} \\
        & \Smid & {e + e} \Smid {e < e} \Smid { \terminal{not}(e) } \\
        & \Smid & \Sif e \Sthen e \Selse e \\
        & \Smid & {\lambda \Id . e} \Smid {e \; e} \\
        & \Smid & \terminal{let} \Id \terminal{=} e \terminal{in} e
    \end{array}
    \]
    \end{block}

    \begin{block}{Verificarea sintaxei în Prolog}
      \begin{asciipl}
exp(Id) :- atom(Id).                      % identifier
exp(Lit) :- Lit = true ; Lit = false ; integer(Lit).
exp(E1 + E2) :- exp(E1), exp(E2).         % same for any op
exp(if(E1, E2, E3)) :- exp(E1), exp(E2), exp(E3).
exp(Id -> Exp) :- atom(Id), exp(Exp).     % lambda
exp(Exp1 $ Exp2) :- exp(Exp1), exp(Exp2). % application
exp(let(Id, Exp1, Exp2)) :- atom(Id), exp(Exp1), exp(Exp2).
      \end{asciipl}
    \end{block}
 \end{frame}

  \begin{frame}[fragile]{Sintaxa tipurilor }
    \begin{block}{BNF}
 \begin{syntaxBlock}{\structure{\tau}}
 \color<beamer|handout>{black!25!white}{
 \renewcommand{\syntaxKeyword}{}
 \syntax{int}{întregi}
 \syntaxCont{bool}{valori de adevăr}
 \syntaxCont{\structure{\tau} \rightarrow \structure{\tau}}{funcții}
 \syntaxCont{\structure{a}}{variabile de tip}
 }
 \vspace{-4ex}
 \end{syntaxBlock}
\end{block}

    \begin{block}{Verificarea sintaxei tipurilor în Prolog}
      \begin{asciipl}
is_type(X) :- variable(X).     % variabile
is_type(int).                  % intregi
is_type(bool).                 % valori de adevar
is_type(T1 -> T2) :-           % functii
   is_type(T1), is_type(T2).
      \end{asciipl}
    \end{block}
\end{frame}

    \begin{frame}[fragile]{Axiome}
    \begin{itemize}
    \item[] $\reg[:var]{\T{\Gamma}{x}{\tau}}{}{\Gamma(x) = \tau}$

    \begin{asciipl}
      type(Gamma, X, T) :- atom(X), get(Gamma, X, T).
    \end{asciipl}

    \vitem[] $\reg[:int]{\T{\Gamma}{n}{int}}{}{\mbox{$n$ întreg}}$

    \begin{asciipl}
      type(_, I, int) :- integer(I).
    \end{asciipl}

    \vitem[] $\reg[:bool]{\T{\Gamma}{b}{bool}}{}{b = \Strue \mathrel{or} b = \Sfalse}$

    \begin{asciipl}
      type(_, true, bool).
      type(_, false, bool).
    \end{asciipl}
    \end{itemize}
    \end{frame}
    
    
    \begin{frame}[fragile]{Expresii}
    \begin{itemize}
    \item[]
    $\reg[:iop]{\T{\Gamma}{e_1\mathrel{o} e_2}{int}}{\T{\Gamma}{e_1}{int} \si \T{\Gamma}{e_2}{int}}{o\in \{+,-,*,/\}}$

    \begin{asciipl}
type(Gamma, E1 + E2, int) :-
    type(Gamma, E1, int), type(Gamma, E2, int).
    \end{asciipl}
    
    \vitem[]
    $\reg[:cop]{\T{\Gamma}{e_1\mathrel{o} e_2}{bool}}{\T{\Gamma}{e_1}{int} \si \T{\Gamma}{e_2}{int}}{o\in \{\leq, \geq, <, >, =\}}$

    \begin{asciipl}
type(Gamma, E1 < E2, bool) :-
    type(Gamma, E1, int), type(Gamma, E2, int).
    \end{asciipl}
    
    \vitem[]
    $\reg[:bop]{\T{\Gamma}{e_1\mathrel{o} e_2}{bool}}{\T{\Gamma}{e_1}{bool} \si \T{\Gamma}{e_2}{bool}}{o\in \{\Sand, \Sor\}}$

    \begin{asciipl}
type(Gamma, and(E1, E2), bool) :-
    type(Gamma, E1, bool), type(Gamma, E2, bool).
    \end{asciipl}
    \end{itemize}
    \end{frame}

    \begin{frame}[fragile]{Expresia condițională}
    \begin{itemize}
    
    \vitem[]
    $\reg[:if]{\T{\Gamma}{\Sif e_b \Sthen e_1 \Selse e_2}{\tau}}{\T{\Gamma}{e_b}{bool} \si \T{\Gamma}{e_1}{\tau} \si \T{\Gamma}{e_2}{\tau}}{}$

    \vitem[]
    \begin{asciipl}
type(Gamma, if(E, E1, E2), T) :-
    type(Gamma, E, bool),
    type(Gamma, E1, T),
    type(Gamma, E2, T).
    \end{asciipl}
    
    \end{itemize}
    \end{frame}
    
    \begin{frame}[fragile]{Fragmentul funcțional}
    \begin{itemize}
    \item[] 
    $\displaystyle\reg[:fn]{\T{\Gamma}{\fun{x}{e}}{\tau \rightarrow \tau'}}{\T{\Gamma'}{e}{\tau'}}{\Gamma' = \Gamma[x\mapsto \tau]}$
    
    \begin{asciipl}
type(Gamma, X -> E, TX -> TE) :-
    atom(X), set(Gamma, X, TX, GammaX), type(GammaX, E, TE).
    \end{asciipl}
 

    \vitem[]
    $\displaystyle\reg[:app]{\T{\Gamma}{e_1\; e_2}{\tau}}{\T{\Gamma}{e_1}{\tau' \rightarrow \tau} \si \T{\Gamma}{e_2}{\tau'}}{}$
    
    \begin{asciipl}
type(Gamma, E1 $ E2, T) :-
    type(Gamma, E, TE2 -> T), type(Gamma, E2, TE2).
    \end{asciipl}
     \end{itemize}
    \end{frame}

      
    \end{subsection}
  \end{section}

  \begin{section}{Funcții polimorfice}
    \begin{frame}{Tipurile variabile nu sunt suficiente}
      \begin{block}{Tipurile variabile sunt destul de flexibile}
      \begin{itemize}
        \item $\vdash {\lambda x . x} : t \rightarrow t$ pentru orice $t$
        \item $\vdash {\Sif (\lambda x.x)\; {\Strue} \Sthen (\lambda x. x)\; 3} \Selse 4 : \Sint$
      \end{itemize}
      \end{block}
      \pause 
      \begin{alertblock}{ Dar tipul unei expresii este fixat:}
       \[\not\vdash (\lambda {id}. {\Sif {id}\; {\Strue} \Sthen {id}\; 3} \Selse 4) (\lambda x. x) : \Sint\]
      \end{alertblock}
      \pause
      \begin{block}{Soluție}
        Pentru funcțiile cu nume, am vrea să fie ca și cum am calcula mereu tipul
       \[\vdash \Slet {id} = (\lambda x. x) \Sin {\Sif {id}\; {\Strue} \Sthen {id}\; 3} \Selse 4)  : \Sint\]

       \structure{Operațional:} redenumim variabilele de tip când instanțiem numele funcției

      \end{block}
      
    \end{frame}

    \begin{frame}{Scheme de tipuri}
      \begin{itemize}
        \item Numim schemă de tipuri o expresie de forma $\c{\tau}$,
         
        unde $\tau$ este e un expresie tip  cu variabile

        \item variabilele dintr-o schemă nu pot fi constrânse 
        
        e ca și cum ar fi cuantificate universal

        \item O schemă poate fi concretizată la un tip obișnuit
        substituindu-i fiecare variabilă cu orice tip (poate fi și variabilă)

        \begin{itemize}
          \item Pentru orice substituție $\theta$ de la variabile de tip la tipuri cu variabile
        \end{itemize}
      \end{itemize}
    \end{frame}


    \begin{frame}[fragile]{Reguli pentru scheme}
      
      \begin{block}{$\displaystyle \reg[:let]{\T{\Gamma}{\Slet {x = e_1} \Sin e2}{\tau}}{\T{\Gamma}{e_1}{\tau_1} \si \T{\Gamma_1}{e_2}{\tau}}{\Gamma_1 = \Gamma[\c{\tau_1} / x]}$}
\pause
        \begin{asciipl}
type(Gamma, let(X, E1, E2), T) :-
    type(Gamma, E1, T1),
    copy_term(T1, FreshT1),   % redenumeste variabilele
                              % ca sa nu poata fi constranse
    set(Gamma, X, scheme(FreshT1), GammaX),
    type(GammaX, E2, T).
        \end{asciipl}
      \end{block}

\pause
      \begin{block}{$\displaystyle \reg[:sch]{\T{\Gamma}{x}{\tau'}}{}{\Gamma(x) = \c{\tau} \mbox{ și } \tau' = \theta(\tau)}$}
        \begin{asciipl}
type(Gamma, X, T) :-
    atom(X), get(Gamma, X, T), is_type(T), !.         
type(Gamma, X, T) :-
    atom(X), get(Gamma, X, scheme(TX)),
    copy_term(TX, T).         % redenumeste variabilele
                              % ca sa poata fi constranse
        \end{asciipl}
      \end{block}
    \end{frame}



  \end{section}

    
%---------------------------------------------------------------------
 
\begin{frame}
\begin{center}
\intens{Pe săptămâna viitoare!}
\end{center}
\end{frame}

%------------------------------------------------------------------------

\end{document}




