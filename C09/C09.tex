%\documentclass[handout,xcolor=x11names,compress,10pt]{beamer}
\documentclass[xcolor=x11names,compress,10pt]{beamer}
%\documentclass[xcolor=pdftex,romanian,colorlinks]{beamer}

\usepackage{../tslides}
%\usepackage{comment}

%% General document %%%%%%%%%%%%%%%%%%%%%%%%%%%%%%%%%%


%\newcommand{\plus}[1] {{\color{Green4} #1}}
%\newcommand{\intens}[1] {{\color{DeepSkyBlue3} #1}}
%\newcommand{\egf}[1]{\stackrel{\cdot}{=}_{#1}}
%\newcommand{\regR}{\mbox{R}}
%\newcommand{\regS}{\mbox{S}}
%\newcommand{\regT}{\mbox{T}}
%\newcommand{\regCS}{\mbox{C}\Sigma}
%\newcommand{\regSub}{\mbox{Sub}_\Gamma}
%\newcommand{\te}[1]{\mbox{\texttt{#1}}}
\newcommand{\Conf}[2]{\ensuremath{\langle #1\ ,\ #2\rangle}}
%\newcommand{\cc}[1]{\ensuremath{\langle #1 \rangle}}
%
%
%
%\newcommand{\parrow}{\rightharpoonup}
%\makeatletter
%\newcommand{\vo}{\vec{o}\@ifnextchar{^}{\,}{}}
%\makeatother
\renewcommand{\to}{}
%=========================================

\begin{document}
\title{\\Curs 9}
\author{Fundamentele limbajelor de programare} 
\date{2020-2021} 


\frame{\titlepage} 


\frame{\frametitle{Cuprins}\tableofcontents

} 

%=========================================
\section{Limbajul IMP} \sectionframe
%=========================================


%------------------------------------------------------------------------


%------------------------------------------------------------------------
\begin{frame}{Limbajul IMP}
Vom implementa un limbaj care conține:

\medskip

\begin{itemize}
\item \intens{Expresii}
	\begin{itemize}
		  \item \intens{Aritmetice} \hfill \texttt{x + 3}
		  \item \intens{Booleene} \hfill \texttt{x >= 7}
	\end{itemize}
\item \intens{Instrucțiuni}
	\begin{itemize}
 		 \item \intens{De atribuire} \hfill \texttt{x = 5}
 		 \item \intens{Condiționale} \hfill \texttt{if(x >= 7,
 		  x =5, x = 0)}
		  \item \intens{De ciclare}  \hfill \texttt{while(x >= 7,x = x - 1)}
		  \end{itemize}
		  \item \intens{Compunerea instruțiunilor}
		  \hfill\texttt{x=7;while(x>=0,x=x-1)}
		  \item \intens{Blocuri de instrucțiuni}
		  \hfill\texttt{\{x=7;while(x>=0,x=x-1)\}}
\end{itemize}


\end{frame}


%------------------------------------------------------------------------
\begin{frame}[fragile]{Limbajul IMP}

\begin{example}
Un program în limbajul IMP

\begin{verbatim}
  {x = 10 ; sum = 0;
  while(0 =< x,
       {sum = sum + x; x = x-1}
       )},sum
       
\end{verbatim}

%x = 10 ; 
%sum = 0; 
%while(0 =< x) {
%               sum = sum + x;
%                 x =  x -1   
%               }

\end{example}

\begin{itemize}
\item \intens{Semantica}

după execuția programului, se evaluează  \texttt{sum} 
\end{itemize}

\end{frame}

%------------------------------------------------------------------------
%\begin{frame}{Sintaxa BNF a limbajului IMP}
%\vspace{-5ex}\begin{syntaxBlock}{\AExp}
%\alert{
%\begin{itemize}
%\item[]\renewcommand{\syntaxKeyword}{}
%\syntax{\Int\Smid\Id}{}
%\syntaxCont{\AExp\terminal{+}\AExp\Smid\AExp\terminal{*}\AExp\Smid\AExp\terminal{/}\AExp}{}
%\item[]\renewcommand{\defSort}{\BExp}
%\syntax{\Bool}{}
%\syntaxCont{\AExp\terminal{<=}\AExp}{}
%\syntaxCont{\Snot\BExp \Smid \BExp \terminal{\&\&} \BExp}{}
%\item[]\renewcommand{\defSort}{\Block}
%\syntax{\terminal{\{}\Stmt\terminal{\}} \Smid \terminal{\{}\terminal{\}}}{}
%\syntaxCont{\Stmt\terminal{}\Stmt}{}
%\syntaxCont{\Id\terminal{=}\AExp\terminal{;}}{}
%\syntaxCont{\Sif\;\terminal{(}\BExp\terminal{)}\Block\Selse\Block}{}
%\syntaxCont{\Swhile\;\terminal{(}\BExp\terminal{)}\Block}{}
%\item[]\renewcommand{\defSort}{\Pgm}
%\syntax{\Sint \Id \terminal{=} \Int \terminal{;} \Pgm\Smid \Stmt}{}
%\end{itemize}
%}
%\end{syntaxBlock}
%\end{frame}




  \begin{frame}{Sintaxa BNF a limbajului IMP}
  \vspace{-5ex}\begin{syntaxBlock}{\AExp}
  \alert{
  \begin{itemize}
  \item[]\renewcommand{\syntaxKeyword}{}
    \syntax{\Int\Smid\Id}{}
%  \syntaxCont{\terminal{++}(\AExp)}{}
  \syntaxCont{\AExp\terminal{+}\AExp\Smid\AExp\terminal{-}\AExp\Smid\AExp\terminal{*}\AExp}{}
     \item[]\renewcommand{\defSort}{\BExp}
  \syntax{\terminal{true}\Smid\terminal{false}}{}
  \syntaxCont{\AExp\terminal{=<}\AExp\Smid \AExp\terminal{>=}\AExp \Smid \AExp\terminal{==}\AExp}{}
  \syntaxCont{\texttt{not}(\BExp) \Smid \texttt{and}(\BExp \terminal{,}\BExp)\Smid \texttt{or}(\BExp \terminal{,}\BExp)}{}
  
  \item[]\renewcommand{\defSort}{\Block}
  \syntax{\terminal{skip}}{}
  \syntaxCont{\Id\terminal{=}\AExp}{}
  \syntaxCont{\Sif\terminal{(}\BExp\terminal{,}\Block\terminal{,}\Block\terminal{)}}{}
  \syntaxCont{\Swhile\terminal{(}\BExp\terminal{,}\Block\terminal{)}}{}
  \syntaxCont{\terminal{\{}\Block\terminal{\}}\Smid \Block\terminal{;}\Block}{}
  \item[]\renewcommand{\defSort}{\Pgm}
  \syntax{\terminal{\{}\Block\terminal{\}}\terminal{,}\AExp}{}
  \end{itemize}
  }
  \end{syntaxBlock}
  \end{frame}

%-------------------------------------------------------------
\section{O implementare a limbajului IMP \^{\i}n Prolog} \sectionframe
%-------------------------------------------------------------

%------------------------------------------------------------------------
\begin{frame}[fragile]{Decizii de implementare}
\vspace*{0.5cm}

\begin{itemize}
\item  \intens{\texttt{$\{\}$}} și \intens{\texttt{;}} sunt operatori
\begin{verbatim}
:- op(100, xf, {}).
:- op(1100, yf, ;).
\end{verbatim}
\medskip

\item definim un predicat pentru fiecare categorie 
sintactică

\begin{verbatim}
stmt(while(BE,St)) :- bexp(BE),  stmt(St).
\end{verbatim}
\medskip

\item \intens{\texttt{while}}, \intens{\texttt{if}}, 
\intens{\texttt{and}}, etc  sunt functori \^{\i}n Prolog 

{\texttt{ while(true,skip)}} este un termen compus

\medskip

\item \intens{,} are semnificația obișnuită
\medskip

\item  pentru valori numerice folosim \^{\i}ntregii din Prolog
\begin{verbatim}
aexp(I) :- integer(I).
\end{verbatim}
\medskip

\item  pentru identificatori folosim atomii din Prolog
\begin{verbatim}
aexp(X) :- atom(X).
\end{verbatim}

\end{itemize}


\end{frame}


\begin{frame}[fragile]{Expresiile aritmetice}

 \vspace{-5ex}\begin{syntaxBlock}{\AExp}
  \alert{
  \begin{itemize}
  \item[]\renewcommand{\syntaxKeyword}{}
  \syntax{\Int\Smid\Id}{}
%  \syntaxCont{\terminal{++}(\AExp)}{}
  \syntaxCont{\AExp\terminal{+}\AExp\Smid\AExp\terminal{-}\AExp\Smid\AExp\terminal{*}\AExp}{}
  \end{itemize}}
  \end{syntaxBlock}
  
\begin{block}{Prolog}

%aexp(++(X)):- atom(X).
\begin{verbatim}
aexp(I) :- integer(I).
aexp(X) :- atom(X).
aexp(A1 + A2) :- aexp(A1), aexp(A2).
aexp(A1 - A2) :- aexp(A1), aexp(A2).
aexp(A1 * A2) :- aexp(A1), aexp(A2).
\end{verbatim}
\end{block}
\end{frame}

\begin{frame}[fragile]{Expresiile aritmetice}
\begin{example}
?- aexp(1000).\\
true.\\

?- aexp(id).\\
true.\\

?- aexp(id + 1000).\\
true. \\


?- aexp(2 + 1000).\\
true. \\


?- aexp(x * y).\\
true.\\

?- aexp(- x).\\
\textcolor{red}{false.}
\end{example}
\end{frame}

\begin{frame}[fragile]{Expresiile booleene}

 \vspace{-5ex}\begin{syntaxBlock}{\AExp}
  \alert{
  \begin{itemize}
  \item[]\renewcommand{\syntaxKeyword}{}
   \item[]\renewcommand{\defSort}{\BExp}
  \syntax{\terminal{true}\Smid\terminal{false}}{}
  \syntaxCont{\AExp\terminal{=<}\AExp\Smid \AExp\terminal{>=}\AExp \Smid \AExp\terminal{==}\AExp}{}
  \syntaxCont{\texttt{not}(\BExp) \Smid \texttt{and}(\BExp \terminal{,}\BExp)\Smid \texttt{or}(\BExp \terminal{,}\BExp)}{}
  \end{itemize}}
  \end{syntaxBlock}
  
\begin{block}{Prolog}


\begin{verbatim}
bexp(true). bexp(false).
bexp(and(BE1,BE2)) :- bexp(BE1), bexp(BE2).
bexp(or(BE1,BE2)) :-  bexp(BE1), bexp(BE2).
bexp(not(BE)) :-  bexp(BE).

bexp(A1 =< A2) :- aexp(A1), aexp(A2).
bexp(A1 >= A2) :- aexp(A1), aexp(A2).
bexp(A1 == A2) :- aexp(A1), aexp(A2).
\end{verbatim}
\end{block}
\end{frame}

\begin{frame}[fragile]{Expresiile booleene}
\begin{example}
?- bexp(true).\\
true.\\

?- bexp(id).\\
\textcolor{red}{false.}\\


?- bexp(not(1 =< 2)).\\
true.\\

?- bexp(or(1 =< 2,true)).\\
true.\\

?- bexp(or(a =< b,true)).\\
true.\\


?- bexp(not(a)).\\
\textcolor{red}{false.}\\ 
?- bexp(!(a)).\\
\textcolor{red}{false.}
\end{example}
\end{frame}

\begin{frame}[fragile]{Instrucțiunile}

 \vspace{-5ex}\begin{syntaxBlock}{\AExp}
  \alert{
  \begin{itemize}
  \item[]\renewcommand{\syntaxKeyword}{}
  \item[]\renewcommand{\defSort}{\Block}
  \syntax{\terminal{skip}}{}
  \syntaxCont{\Id\terminal{=}\AExp\terminal{;}}{}
  \syntaxCont{\Sif\terminal{(}\BExp\terminal{)}\Block\Selse\Block}{}
  \syntaxCont{\Swhile\terminal{(}\BExp\terminal{)}\Block}{}
  \syntaxCont{\terminal{\{}\Block\terminal{\}}\Smid \Block\terminal{;}\Block}{}
  \end{itemize}}
  \end{syntaxBlock}
  
\begin{block}{Prolog}


\begin{verbatim}
stmt(skip).
stmt(X = AE) :- atom(X), aexp(AE).
stmt(St1;St2) :- stmt(St1), stmt(St2).  
stmt((St1;St2)) :- stmt(St1), stmt(St2).  
stmt({St}) :- stmt(St).
stmt(if(BE,St1,St2)) :-  bexp(BE), stmt(St1), stmt(St2).
stmt(while(BE,St)) :- bexp(BE), stmt(St).
\end{verbatim}
\end{block}
\end{frame}

\begin{frame}[fragile]{Instrucțiunile}
\begin{example}
?- stmt(id = 5).\\
true.\\

?- stmt(id = a).\\
true.\\

?- stmt(3 = 6).\\
false.\\

?- stmt(if(true, x=2;y=3, x=1;y=0)).\\
true.\\

?- stmt(while(x =< 0,skip)).\\
true.\\


?- stmt(while(x =< 0,{})).\\
\textcolor{red}{false.}\\

?- stmt(while(x =< 0,{skip})).\\
true .
\end{example}
\end{frame}


\begin{frame}[fragile]{Programele}

 \vspace{-5ex}\begin{syntaxBlock}{\AExp}
  \alert{
  \begin{itemize}
  \item[]\renewcommand{\syntaxKeyword}{}
  \item[]\renewcommand{\defSort}{\Pgm}
  \syntax{ \terminal{\{}\Block\terminal{\}}\terminal{,}\AExp}{}
  \end{itemize}}
  \end{syntaxBlock}
  
\begin{block}{Prolog}
\begin{verbatim}
program(St,AE) :- stmt(St), aexp(AE).
\end{verbatim}
\end{block}

\begin{example}
\begin{verbatim}
 test0 :- program( {x = 10 ; sum = 0; 
                   while(0 =< x, 
                          {sum = sum + x; x = x-1}
                         )}
                   , sum).
\end{verbatim}  
                  
?- test0.\\
true.                  

\end{example}

\end{frame}

%=========================================
\section{O implementare a semanticii small-step} \sectionframe
%=========================================

\begin{frame}{Semantica small-step}
  \medskip
  \begin{itemize}
  \item Introdusă de Gordon Plotkin (1981) 
  \smallskip
  \item Denumiri alternative: 
  \begin{itemize}
	\item  \structure{S}emantică \structure{O}perațională \structure{S}tructurală
	\item  semantică prin tranziții
	\item  semantică prin reducere
  \end{itemize}
   \smallskip
  \item Definește cel mai mic pas de execuție ca o relație „de tranziție” între configurații:
  \begin{center}
	$\Ss{\Conf{cod}{\sigma}}{\Conf{cod}{\sigma'}}$
  \end{center}
%  \item Fiecare pas de execuție este concluzia unei demonstrații
 \smallskip \pause
  \item Execuția se obține ca o succesiune de astfel de tranziții:
  
   \smallskip
  \begin{tabular}{rcl}
  $\Conf{\alert{\Sint x \terminal{=} 0 \terminal{;}}  x \terminal{=} x \terminal{+} 1 \terminal{;}}{\bot}$ & $\longrightarrow$ & $\Conf{x \terminal{=} \alert{x} \terminal{+} 1 \terminal{;}}{x \mapsto 0}$ \\ \pause
  &$\longrightarrow$& $\Conf{x \terminal{=} \alert{0 \terminal{+} 1} \terminal{;}}{x \mapsto 0}$ \\ \pause
  &$\longrightarrow$& $\Conf{\alert{x \terminal{=} 1 \terminal{;}}}{x \mapsto 0}$ \\  \pause
  &$\longrightarrow$& $\Conf{\terminal{\{}\terminal{\}}}{x \mapsto 1}$
  \end{tabular}
  
    \end{itemize}
  \end{frame}


\begin{frame}[fragile]{Semantica small-step}
\vspace*{0.5cm}
\begin{itemize}
  \item Definește cel mai mic pas de execuție ca o relație de tranziție  \^{\i}ntre configurații:
  \begin{center}
	$\Ss{\Conf{cod}{\sigma}}{\Conf{cod\,'}{\sigma'}}$ \hfill\intens{\texttt{step(Cod,S1,Cod',S2)}}
  \end{center}
%  \item Fiecare pas de execuție este concluzia unei demonstrații

  \item Execuția se obține ca o succesiune de astfel de tranziții.
  \item Starea executiei unui program IMP la un moment dat este o funcție parțială: $\sigma = n \mapsto 10, sum \mapsto 0$, etc.
\end{itemize}
 \begin{block}{Reprezentarea stărilor \^{\i}n Prolog}
\begin{verbatim} 
get(S,X,I) :- member(vi(X,I),S).
get(_,_,0).
set(S,X,I,[vi(X,I)|S1]) :- del(S,X,S1).

del(S,X,S1) :- select(vi(X,_), S, S1), !.
del(S, _, S).
\end{verbatim}
\end{block}
\end{frame}

  
\begin{frame}[fragile]{Semantica expresiilor aritmetice}
 
\begin{itemize}
  
   \item \intens{Semantica unei variabile}
	\item[] $\reg{\Ss{\Conf{x}{\sigma}}{\Conf{i}{\sigma}}}{}{i = \sigma(x)}$
   \end{itemize}
 \begin{block}{Prolog}
\begin{verbatim}  
step(X,S,I,S) :-
  atom(X),
  get(S,X,I).
\end{verbatim}
\end{block}   
  \end{frame}
  

  
\begin{frame}[fragile]{Semantica expresiilor aritmetice}
  \begin{itemize}
  \item \intens{Semantica adunării a două expresii aritmetice}	
   \item[] $\reg{\Ss{\Conf{i_1 + i_2}{\sigma}}{\Conf{i}{\sigma}}}{}{i = i_1 + i_2}$
  
  \item[] $\reg{\Ss{\Conf{a_1 + a_2}{\sigma}}{\Conf{a_1' + a_2}{\sigma'}}}{\Ss{\Conf{a_1}{\sigma}}{\Conf{a_1'}{\sigma'}}}{}$\hfill$\reg{\Ss{\Conf{a_1 + a_2}{\sigma}}\to{\Conf{a_1 + a_2'}{\sigma'}}}{\Ss{\Conf{a_2}{\sigma}}\to {\Conf{a_2'}{\sigma'}}}{}$
%  \vitem Ordine nespecificată de evaluare a argumentelor
  \end{itemize}
  
 \begin{block}{Prolog} 
\begin{verbatim}
step(I1 + I2,S,I,S):- integer(I1),integer(I2), 
                            I is I1 + I2.
                            
step(AE + AE1,S1,AE + AE2,S1):- step(AE1,S1,AE2,S2).

step(AE1 + AE,S1,AE2 + AE,S2):- step(AE1,S1,AE2,S2).
\end{verbatim}
\end{block}
\end{frame}



\begin{frame}{Semantica expresiilor aritmetice}
\begin{example}

?- step(a + b, [vi(a,1),vi(b,2)],AE, S).\\
AE = 1+b,\\
S = [vi(a, 1), vi(b, 2)] .\\

\medskip

?- step(1 + b, [vi(a,1),vi(b,2)],AE, S).\\
AE = 1+2,\\
S = [vi(a, 1), vi(b, 2)] .\\

\medskip

?- step(1 + 2, [vi(a,1),vi(b,2)],AE, S).\\
AE = 3,\\
S = [vi(a, 1), vi(b, 2)] 
\end{example}

\pause
  \begin{itemize}
   \item \intens{Semantica \texttt{*} și \texttt{-} se definesc similar.}
  \end{itemize}
\end{frame}

\begin{frame}[fragile]{Semantica expresiilor booleene}
%  {Expresii Booleene. Constante și operatorul de comparație.}
\vspace*{0.5cm}
 
  \begin{itemize}
 

 \item \intens{Semantica operatorului de comparație}
 
    $\reg{\Ss{\Conf{i_1 \terminal{=<} i_2}{\sigma}}\to{\Conf{\terminal{false}}{\sigma}}}{}{i_1 > i_2}$
    
   $\reg{\Ss{\Conf{i_1 \terminal{=<} i_2}{\sigma}}\to {\Conf{\terminal{true}}{\sigma}}}{}{i_1 \leq i_2}$

\medskip

 $\reg{\Ss{\Conf{a_1 \terminal{=<} a_2}{\sigma}}\to{\Conf{a_1' \terminal{=<} a_2}{\sigma'}}}{\Ss{\Conf{a_1}{\sigma}}\to{\Conf{a_1'}{\sigma'}}}{}\,\,\reg{\Ss{\Conf{a_1 \terminal{=<} a_2}{\sigma}}\to{\Conf{a_1 \terminal{=<} a_2'}{\sigma'}}}{\Ss{\Conf{a_2}{\sigma}}\to{\Conf{a_2'}{\sigma'}}}{}$
 
\end{itemize}
 
  \begin{block}{Prolog}  
  \begin{verbatim}
step(I1 =< I2,S,true,S):- integer(I1),integer(I2),
                                (I1 =< I2).
step(I1 =< I2,S,false,S):- integer(I1),integer(I2),
                                 (I1 > I2).
step(AE =< AE1,S1,AE =< AE2,S2):- step(AE1,S1,AE2,S2).
step(AE1 =< AE,S1,AE2 =< AE,S2):- step(AE1,S1,AE2,S2).
  \end{verbatim}
  \end{block}
  \end{frame}
  
\begin{frame}[fragile]{Semantica expresiilor Booleene}
%  {Expresii Booleene. Constante și operatorul de comparație.}
\vspace*{0.5cm}
 
 
    \begin{itemize}
  \item \intens{Semantica negației} 
   
  \item[] $\reg{\Ss{\Conf{\terminal{not(} \terminal{true)}}{\sigma}}\to{\Conf{\terminal{false}}{\sigma}}}{}{}$
    
  \item[] $\reg{\Ss{\Conf{\terminal{not(} \terminal{false)}}{\sigma}}\to{\Conf{\terminal{true}}{\sigma}}}{}{}$
   \medskip
  \item[] $\reg{\Ss{\Conf{\terminal{not}(a)}{\sigma}}\to{\Conf{\terminal{not}(a')}{\sigma'}}}{\Ss{\Conf{a}{\sigma}}\to{\Conf{a'}{\sigma'}}}{}$
\end{itemize}
 
  \begin{block}{Prolog}  
  \begin{verbatim}
step(not(true),S,false,S) .
step(not(false),S,true,S) .

step(not(BE1),S1,not(BE2),S2) :- step(BE1,S1,BE2,S2).
  \end{verbatim}
  \end{block}
  \end{frame}
  \begin{frame}[fragile]{Semantica compunerii și a blocurilor}
%  {Expresii Booleene. Constante și operatorul de comparație.}
\vspace*{0.5cm}
 
 
    \begin{itemize}
  \item \intens{Semantica blocurilor} 
  
  $\Ss{\Conf{\terminal{\{} s \terminal{\}}}{\sigma}}\to{\Conf{ s}{\sigma}}$
  
  
  
   \item \intens{Semantica compunerii secvențiale }
   
   $\reg{\Ss{\Conf{\terminal{\{}\terminal{\}} ; s_2}{\sigma}}\to{\Conf{s_2}{\sigma}}}{}{}\,\,\reg{\Ss{\Conf{s_1 ; s_2}{\sigma}}\to{\Conf{s_1' ; s_2}{\sigma'}}}{\Ss{\Conf{s_1}{\sigma}}\to{\Conf{s_1'}{\sigma'}}}{}$
\end{itemize}
 
  \begin{block}{Prolog}  
  \begin{verbatim}
step({E},S,E,S).

step((skip;St2),S,St2,S).
step((St1;St),S1,(St2;St),S2) :-
                              step(St1,S1,St2,S2) .
  \end{verbatim}
  \end{block}
  \end{frame}  

\begin{frame}[fragile]{Semantica atribuirii}
%  {Expresii Booleene. Constante și operatorul de comparație.}
\vspace*{0.5cm}
 
 
    \begin{itemize}
  \item \intens{Semantica atribuirii} 
    \item[] $\reg{\Ss{\Conf{x \terminal{=} i}{\sigma}}\to{\Conf{\terminal{\{\}}}{\sigma'}}}{}{\sigma'=\sigma[i/x]}$
   \medskip
  \item[] $\reg{\Ss{\Conf{x \terminal{=} a}{\sigma}}\to{\Conf{x \terminal{=} a' \terminal{;}}{\sigma'}}}{\Ss{\Conf{a}{\sigma}}\to{\Conf{a'}{\sigma'}}}{}$
  
\end{itemize}
 
  \begin{block}{Prolog}  
  \begin{verbatim}
step(X = I,S,skip,S1) :- integer(I),set(S,X,I,S1).

step(X = AE1,S1,X = AE2,S2) :- 
                                step(AE1,S1,AE2,S2).
  \end{verbatim}
  \end{block}
  \end{frame}  


\begin{frame}[fragile]{Semantica lui \texttt{if}}
%  {Expresii Booleene. Constante și operatorul de comparație.}
\vspace*{0.5cm}
 
 
    \begin{itemize}
  \item \intens{Semantica lui \texttt{if}} 
 \item[]
  $\reg{\Ss{\Conf{\Sif (\terminal{true},{\it bl}_1, {\it bl}_2)}{\sigma}}\to{\Conf{{\it bl}_1}{\sigma}}}{}{}$
   
  \item[]
  $\reg{\Ss{\Conf{\Sif (\terminal{false},{\it bl}_1,{\it bl}_2)}{\sigma}}\to{\Conf{{\it bl}_2}{\sigma}}}{}{}$
   \medskip   
  \item[]
  $\reg{
   \Ss{\Conf{\Sif ({b},{\it bl}_1, {\it bl}_2)}{\sigma}}\to{\Conf{\Sif ({b'}, {\it bl}_1, {\it bl}_2)}{\sigma'}}
  }{
    \Ss{\Conf{b}{\sigma}}\to{\Conf{b'}{\sigma'}}
  }
  {}$
  
\end{itemize}
 
  \begin{block}{Prolog}  
  \begin{verbatim}
step(if(true,St1,_),S,St1,S).
step(if(false,_,St2),S,St2,S).

step(if(BE1,St1,St2),S1,if(BE2,St1,St2),S2) :- 
                                  step(BE1,S1,BE2,S2) .
  \end{verbatim}
  \end{block}
  \end{frame}
  
 \begin{frame}[fragile]{Semantica lui \texttt{while}}
%  {Expresii Booleene. Constante și operatorul de comparație.}
\vspace*{0.5cm}
 
 
    \begin{itemize}
  \item \intens{Semantica lui \texttt{while}} 
  \item[]
  $\reg{
   \Ss{\Conf{\Swhile ({b},{\it bl})}{\sigma}}\to{\Conf{\Sif({b},  {\it bl}\terminal{;}\Swhile({b},{\it bl}),\terminal{skip})}{\sigma}}
  }{}
  {}$
  
\end{itemize}
 
  \begin{block}{Prolog}  
  \begin{verbatim}
step(while(BE,St),S,if(BE,(St;while(BE,St)),skip),S).
  \end{verbatim}
  \end{block}
  \end{frame}  
   
   
 \begin{frame}[fragile]{Semantica programelor}
%  {Expresii Booleene. Constante și operatorul de comparație.}
\vspace*{0.5cm}
 
 
    \begin{itemize}
  \item \intens{Semantica programelor} 
  \item[] $\reg{\Ss{\Conf{(\terminal{skip}, a_1)}{\sigma_1}}\to{\Conf{(\terminal{skip}, a_2)}{\sigma_2}}}{\Ss{\Conf{a_1}{\sigma_1}}\to{\Conf{a_2}{\sigma_2}}}{}$

 \item[] $\reg{\Ss{\Conf{(s_1, a)}{\sigma_1}}\to{\Conf{(s_2, a)}{\sigma_2}}}{\Ss{\Conf{s_1}{\sigma_1}}\to{\Conf{s_2}{\sigma_2}}}{}$  

  
  
\end{itemize}
 
  \begin{block}{Prolog}  
  \begin{verbatim}
step((skip,AE1),S1,(skip,AE2),S2) :- 
                           step(AE1,S1,AE2,S2) .
step((St1,AE),S1,(St2,AE),S2) :-
                           step(St1,S1,St2,S2).
  \end{verbatim}
  \end{block}
  \end{frame} 
  
  
 \begin{frame}[fragile]{Execuția programelor}
%  {Expresii Booleene. Constante și operatorul de comparație.}
%\vspace*{0.5cm}
 
 \begin{block}{Prolog}  
  \begin{verbatim}
all_steps(P1, S1, PF, SF) :-
    step(P1, S1, P2, S2)
    ->  all_steps(P2, S2, PF, SF)
    ;   PF = P1, SF = S1.
                     
run_program(Name) :- defpg(Name,{P},E),
                     all_steps((P,E), [],(skip, I), _),
                     write(I).
  \end{verbatim}
  \vspace*{-0.5cm}
  \end{block}
    \vspace*{-0.5cm}
     \begin{example}  
  \begin{verbatim}
defpg(pg2, {x = 10 ; sum = 0; while(0 =< x, {
                                  sum = sum + x;
                                  x = x - 1})},sum)
\end{verbatim}

?- run\_program(pg2).\\
55\\
true
  \end{example}
  \vspace*{0.3cm}
  \end{frame}  
  
    

 \begin{frame}[fragile]{Execuția programelor: \texttt{trace}}
 
 \medskip
 
 Putem defini o funcție care ne permite să urmărim execuția unui program \^{\i}n implementarea noastră?
 
 \medskip\pause
 
 \begin{block}{Prolog}
 
 \begin{verbatim}
all_steps(P1, S1, PF, SF, [(P1, S1)| Trace]) :-
    step(P1, S1, P2, S2)
    ->  all_steps(P2, S2, PF, SF, Trace)
    ;   PF = P1, SF = S1, Trace=[].

trace_program(Name) :- defpg(Name,{P},E), 
                       all_steps((P,E), [],_, _, Trace),
                       write(Trace).
\end{verbatim}
\end{block}
\end{frame}

 \begin{frame}{Execuția programelor: \texttt{trace\_program}}

 \begin{example}
?- trace\_program(pg2).

$\cdots$\\
{\small
((if(0=<x,({sum=sum+x;x=x-1};while(0=<x,{sum=sum+x;x=x-1})),skip), sum),
[vi(x,-1),vi(sum,55)]),

((if(0=<-1,({sum=sum+x;x=x-1};while(0=<x,{sum=sum+x;x=x-1})),skip), sum),
[vi(x,-1),vi(sum,55)]),

((if(false,({sum=sum+x;x=x-1};while(0=<x,{sum=sum+x;x=x-1})),skip), sum),
[vi(x,-1),vi(sum,55)]),

((skip, sum),
[vi(x,-1),vi(sum,55)]),

((skip, 55),
[vi(x,-1),vi(sum,55)]),
}
 \end{example}
 
 \end{frame}
%---------------------------------------------------------------------

\section{Semantica Small-Step pentru Lambda Calcul}

  \begin{frame}[fragile]{Sintaxa limbajului LAMBDA}
    \begin{block}{BNF}
    \[
    \begin{array}{rcl}
      e & ::= & \Id \\
        & \Smid & \lambda \Id . e \\
        & \Smid & e\; e \\
        & \Smid & \terminal{let} \Id \terminal{=} e \terminal{in} e
    \end{array}
    \]
    \end{block}

    \begin{block}{Verificarea sintaxei în Prolog}
      \begin{asciipl}
exp(Id) :- atom(Id).                      % identifier
exp(Id -> Exp) :- atom(Id), exp(Exp).     % lambda
exp(Exp1 $ Exp2) :- exp(Exp1), exp(Exp2). % application
exp(let(Id, Exp1, Exp2)) :- atom(Id), exp(Exp1), exp(Exp2).
      \end{asciipl}
    \end{block}
 \end{frame}

\begin{frame}[fragile]{Semantica small-step pentru Lambda}
\vspace*{0.5cm}
\begin{itemize}
  \item Definește cel mai mic pas de execuție ca o relație de tranziție  \^{\i}ntre expresii
    dată fiind o stare cu valori pentru variabilele libere
  \begin{center}
	$\rho \vdash \Ss{cod}{cod\,'}$ \hfill\intens{\texttt{step(Env, Cod1,Cod2)}}
  \end{center}
%  \item Fiecare pas de execuție este concluzia unei demonstrații

  \item Execuția se obține ca o succesiune de astfel de tranziții.
\end{itemize}
\end{frame}

\begin{frame}[fragile]{Semantica variabilelor}
	\[\reg{\rho \vdash \Ss{x}{v}}{}{\rho(x) = v}\]
  \begin{block}{Prolog}
    \begin{asciipl}
      step(Env, X, V) :- atom(X), get(Env, X, V).
    \end{asciipl}
  \end{block}

\end{frame}

\begin{frame}[fragile]{Semantica $\lambda$-abstracției}
	\[\reg{\rho \vdash \Ss{\lambda x. e}{{\it closure}(x, e, \rho)}}{}{}\]

  $\lambda$-abstracția se evaluează la o valoare specială numită closure
  care capturează valorile curente ale variabilelor pentru a se putea
  executa în acest mediu atunci când va fi aplicată.

  \begin{block}{Prolog}
    \begin{asciipl}
      step(Env, X -> E, closure(X, E, Env)).
    \end{asciipl}
  \end{block}

\end{frame}

\begin{frame}[fragile]{Semantica construcției $\terminal{let}$}
	\[\reg{\rho \vdash \Ss{\terminal{let} x \terminal{=} e_1 \terminal{in} e_2}{(\lambda x . e2)\; e_1}}{}{}\]

  A îi da lui $x$ valoarea lui $e_1$ în $e_2$ este același lucru cu a
  aplica funcția de $x$ cu corpul $e_2$ expresiei $e_1$.

  \begin{block}{Prolog}
    \begin{asciipl}
      step(_, let(X, E1, E2), (X -> E2) $ E1).
    \end{asciipl}
  \end{block}

\end{frame}


\begin{frame}[fragile]{Semantica operatorului de aplicare}
	\[\reg{\rho \vdash \Ss{{\it closure}(x, e, \rho_e) \; v}{{\it closure}(x, e', \rho_e) \; v}}{\rho_e[v/x]\vdash \Ss{e}{e'}}{v \mbox{ valoare}}\]

	\[\reg{\rho \vdash \Ss{{\it closure}(x, v, \rho_e) \; e}{v}}{}{v \mbox{ valoare}}\]

  \[\reg{\rho \vdash \Ss{e_1 \; e_2}{e'_1 \; e_2}}{\rho \vdash \Ss{e_1}{e'_1}}{}
    \hfill
    \reg{\rho \vdash \Ss{e_1 \; e_2}{e_1 \; e'_2}}{\rho \vdash \Ss{e_2}{e'_2}}{}
    \]

  \begin{block}{Prolog}
    \begin{asciipl}
      step(Env, E $ E1, E $ E2) :- step(Env, E1, E2).
      step(Env, E1 $ E, E2 $ E) :- step(Env, E1, E2).
      step(Env, closure(X, E, EnvE) $ V, Result) :-
          \+ step(Env, V, _),
          set(EnvE, X, V, EnvEX),
          step(EnvEX, E, E1)
          ->  Result = closure(X, E1, EnvE) $ V
          ;   Result = E.
    \end{asciipl}
  \end{block}

\end{frame}



\begin{frame}
\begin{center}
\intens{Pe săptămâna viitoare!}
\end{center}
\end{frame}
%---------------------------------------------------------------------



%------------------------------------------------------------------------

\end{document}



