\documentclass[xcolor=pdftex,romanian,colorlinks]{beamer}
%\documentclass[xcolor=pdftex,handout,romanian,colorlinks]{beamer}

\usepackage[export]{adjustbox}
\usepackage{../sty/tslides}
\usepackage[all]{xy}
\usepackage{pgfplots}
\usepackage{flowchart}
\usepackage{todonotes}
\usetikzlibrary{arrows,positioning,calc}
\lstset{language=Haskell}
\lstset{escapeinside={(*@}{@*)}}
\PrerenderUnicode{ăĂîÎȘșȚțâÂ}
\usepackage{amsmath}


%\usepackage{xcolor}
%\definecolor{IntensColor}{HTML}{2E86C1}
%\definecolor{StateTransition}{HTML}{D6EAF8}
%\definecolor{MedianLightOrange}{RGB}{216,178,92}
%\definecolor{Orchid}{HTML}{8E44AD}
%\definecolor{True}{HTML}{229954}
%\definecolor{False}{HTML}{CB4335}


\usepackage{proof}
\usepackage{multirow}
\usepackage{alltt}
\usepackage{mathpartir}
\usepackage{ulem}

\newcommand{\structured}[1]{#1}

\definecolor{IntensColor}{HTML}{2E86C1}
\definecolor{StateTransition}{HTML}{D6EAF8}
\definecolor{MedianLightOrange}{RGB}{216,178,92}
\definecolor{Orchid}{HTML}{8E44AD}
\definecolor{True}{HTML}{229954}
\definecolor{False}{HTML}{CB4335}

\newcommand{\cin}[1]{{\color{cobalt} #1}}
\newcommand{\sel}[1]{{\color{Orchid} #1}}

\newcommand{\intens}[1] {{\color{IntensColor} #1}}
\newcommand{\exe}[1] {{\color{True} #1}}

\newcommand{\la}{\lambda}

\setlength{\leftmargini}{0pt}

\newcommand{\app}[2]{#1\, #2}
\newcommand{\abs}[2]{\lambda #1.\,#2}

\newcommand{\type}[2]{{\color{True}#1\hspace{-.05cm}:}\,{\color{Orchid}#2}}

\newcommand{\sub}[3]{#1\langle#2/#3\rangle}
\newcommand{\subt}[3]{#1[#2/#3]}
\newcommand{\equiva}{=_\alpha}

\newcommand{\trueL}{\mathbf{T}}
\newcommand{\falseL}{\mathbf{F}}
\newcommand{\notL}{\mathbf{not}}
\newcommand{\andL}{\mathbf{and}}
\newcommand{\orL}{\mathbf{or}}
\newcommand{\ifL}{\mathbf{if}}
\newcommand{\boolL}{\mathbf{bool}}

\newcommand{\maybeL}{\mathbf{maybe}}
\newcommand{\nothingL}{\mathbf{Nothing}}
\newcommand{\justL}{\mathbf{Just}}
\newcommand{\Maybe}[1]{\mathop{\mathbf{Maybe}}{#1}}

\newcommand{\foldrL}{\mathbf{foldr}}
\newcommand{\nilL}{\mathbf{Nil}}
\newcommand{\consL}{\mathbf{Cons}}
\newcommand{\ListL}[1]{\mathop{\mathbf{List}}{#1}}

\newcommand{\unpairL}{\mathbf{uncons}}
\newcommand{\pairL}{\mathbf{Pair}}
\newcommand{\firstL}{\mathbf{first}}
\newcommand{\secondL}{\mathbf{second}}
\newcommand{\Pair}[2]{\mathop{\mathop{\mathbf{Pair}}{#1}}{#2}}

\newcommand{\succL}{\mathbf{Succ}}
\newcommand{\zeroL}{\mathbf{Zero}}
\newcommand{\iterateL}{\mathbf{iterate}}
\newcommand{\addL}{\mathbf{add}}
\newcommand{\mulL}{\mathbf{mul}}
\newcommand{\expL}{\mathbf{exp}}
\newcommand{\isZero}{\mathbf{isZero}}
\newcommand{\pred}{\mathbf{pred}}
\newcommand{\factL}{\mathbf{fact}}

\newcommand{\BoolT}{\ensuremath{\texttt{Bool}}}
%\newcommand{\BoolT}{\ensuremath{\texttt{Bool}}}
\newcommand{\ifT}[3]{\mathrm{if}\ #1\ \mathrm{then}\ #2\ \mathrm{else}\ #3}

\newcommand{\UnitT}{\ensuremath{\texttt{Unit}}}
\newcommand{\unit}{\mathrm{unit}}

\newcommand{\VoidT}{\ensuremath{\texttt{Void}}}
\newcommand{\void}{\mathrm{void}}


\newcommand{\ProductT}[2]{#1 \times #2}
\newcommand{\PairL}[2]{\langle #1,#2\rangle}
\newcommand{\ProjOne}[1]{fst\ #1}
\newcommand{\ProjTwo}[1]{snd\ #1}

\newcommand{\SumT}[2]{#1 + #2}
\newcommand{\Left}[1]{\mathrm{Left}\ #1}
\newcommand{\Right}[1]{\mathrm{Right}\ #1}
\newcommand{\Case}[3]{\mathrm{case}\ #1\ \mathrm{of}\ #2\ ;\ #3}

%----------------------------------------------

\newcommand{\SSnot}{\terminal{not}}

\newcommand{\Sand}{\terminal{and}}
\newcommand{\Sor}{\terminal{or}}
\newcommand{\Splus}{\terminal{+}}
\newcommand{\Smul}{\terminal{*}}
\newcommand{\Ssucc}{\terminal{S}}
\newcommand{\Spow}{\terminal{pow}}
\newcommand{\Spred}{\terminal{pred}}
\newcommand{\Seq}{\terminal{eq}}
\newcommand{\Sneq}{\terminal{neq}}

\newcommand{\SisZero}{\terminal{isZero}}
\newcommand{\Slte}{\terminal{<=}}
\newcommand{\Sgte}{\terminal{>=}}
\newcommand{\Slt}{\terminal{<}}
\newcommand{\Sgt}{\terminal{>}}
\newcommand{\Spair}{\terminal{pair}}
\newcommand{\Sfst}{\terminal{fst}}
\newcommand{\Ssnd}{\terminal{snd}}
\newcommand{\Sminus}{\terminal{-}}

\newcommand{\Snull}{\terminal{null}}
\newcommand{\Scons}{\terminal{cons}}
%\newcommand{\c sead}{\terminal{head}}
\newcommand{\SisNull}{\terminal{?null}}
\newcommand{\Stail}{\terminal{tail}}
\newcommand{\Ssum}{\terminal{sum}}
\newcommand{\Sfoldr}{\terminal{foldr}}
\newcommand{\Smap}{\terminal{map}}
\newcommand{\Sfilter}{\terminal{filter}}

\newcommand{\const}[1]{\triangleright {\color{False} #1}}

\newcommand{\egf}[1]{\stackrel{\cdot}{=}_{#1}}

\newcommand{\Conf}[2]{\ensuremath{\langle #1\ ,\ #2\rangle}}
\newcommand{\plus}[1] {{\color{True} #1}}
\newcommand{\te}[1]{\mbox{\texttt{#1}}}

\newcommand{\vexp}{\ensuremath{\mathbb{E}}}
\newcommand{\bexp}{\ensuremath{\mathbb{B}}}
\newcommand{\cmd}{\ensuremath{\mathbb{C}}}

\definecolor{section-color}{HTML}{23373b} %mDarkTeal
%\AtBeginSection[]{
%  \begin{frame}
%  \vfill
%  \centering
%  \begin{beamercolorbox}[sep=8pt,center,shadow=true,rounded=true]{title}
%    \usebeamerfont{title}\insertsectionhead\par%
%  \end{beamercolorbox}
%  \vfill
%  \end{frame}
%}



\title[FLP]{Fundamentele limbajelor de programare}
\subtitle{C07}
\date{}


\begin{document}
\begin{frame}
  \titlepage
\end{frame}

\setlength{\leftmargini}{12pt}


%================================================
\section{\color{section-color} Lambda calcul cu tipuri simple (cont.)}
%================================================

%%------------------------------------------------
%\begin{frame}{Tipuri simple}
%
%Mulțimea  \alert{tipurilor simple}  \hspace{.2cm}
% \intens{$\mathbb{T} = \mathbb{V}\ |\ \mathbb{T} \rightarrow \mathbb{T}$}
%\vspace{-.2cm}
%\begin{itemize}
%	\item \alert{(Tipul variabilă)} Dacă $\alpha \in \mathbb{V}$, atunci $\alpha \in \mathbb{T}$.
%	\item \alert{(Tipul săgeată)} Dacă $\sigma,\tau \in \mathbb{T}$, atunci  $(\sigma \rightarrow \tau) \in \mathbb{T}$.
%\end{itemize}
%
%
%Mulțimea \alert{$\lambda$-termenilor cu pre-tipuri $\Lambda_\mathbb{T}$} \hspace{.2cm}
%\intens{$\Lambda_\mathbb{T} = x\ |\ \app{\Lambda_\mathbb{T}}{\Lambda_\mathbb{T}}\ |\ \abs{\type{x}{\mathbb{T}}}{\Lambda_\mathbb{T}}$}
%
%
%\begin{itemize}
%	\item O \alert{afirmație} este o expresie de forma $\type{M}{\sigma}$, unde $M \in \Lambda_\mathbb{T}$ și $\sigma\in \mathbb{T}$.\\
%	\item O \alert{declarație} este o afirmație de forma $\type{x}{\sigma}$.
%	\item Un \alert{context} $\Gamma$ este o listă de declarații cu subiecți diferiți.
%	\item O \alert{judecată} este o expresie de forma $\Gamma \vdash \type{M}{\sigma}$.
%\end{itemize}
%
%\alert{Sistem de deducție pentru calculul Church $\lambda\hspace{-.1cm}\to$}
%\vspace{-.6cm}
%\begin{center}
%{\footnotesize
%\begin{tabular}{lll}
%\infer[(var)]
%	{\Gamma \vdash \type{x}{\sigma}}
%	{}
%%------------------
%&
%\infer[(\to_E)]
%	{\Gamma \vdash \type{\app{M}{N}}{\tau}}
%	 {\Gamma \vdash \type{M}{\sigma \to \tau} \hspace{.5cm} \Gamma \vdash \type{N}{\sigma}}
%%------------------
%&
%\infer[(\to_I)]
%	{\Gamma \vdash \type{(\abs{\type{x}{\sigma}}{M})}{\sigma \to \tau} }
%	 {\Gamma, \type{x}{\sigma} \vdash \type{M}{\tau}}\\
% dacă $\type{x}{\sigma}\in \Gamma$ 	 & &\\	 
%\end{tabular}
%}	 
%\end{center}
%
%Un termen $M$  în calculul $\lambda\hspace{-.1cm}\to$ este \alert{legal} dacă $\Gamma \vdash \type{M}{\rho}$.
%
%\end{frame}

%%------------------------------------------------
%\begin{frame}{Ce probleme putem să rezolvăm în teoria tipurilor?}
%
%\intens{\textit{Well-typedness (Typability/Type Inference)}}
%
%Se reduce la a verifica dacă un termen este \alert{legal}.
%Concret, trebuie să găsim un context și un tip dacă termenul este legal, altfel să arătăm de ce nu se poate.
%\vspace{-.3cm}
%\begin{center}
%$? \vdash \type{term}{?}$
%\end{center}
%O variațiune a problemei este \intens{\textit{Type Assignment}} în care contextul este dat și trebuie să găsim tipul.
%\vspace{-.3cm}
%\begin{center}
%$context \vdash \type{term}{?}$
%\end{center}
%
%\end{frame}

%------------------------------------------------
\begin{frame}{Ce problemă vrem să rezolvăm în cursul de astăzi?}

\intens{\textit{Type Inference}}

Pentru un lambda termen $M$ fără tipuri, vrem să adnotăm termenul $M$ cu tipuri obținând $\overline{M}$ și să rezolvăm problema
\vspace{-.3cm}
\begin{center}
$? \vdash \type{\overline{M}}{?}$
\end{center}
(să găsim un context și un tip, pentru a avea o judecată legală).

\vspace{.2cm}  
\textbf{\intens{Exemple:}}
\vspace{-.2cm}
\begin{itemize}
	\item Pentru termenul {\color{True}$\app{(\abs{z}{\abs{u}{z}})}{(\app{y}{x})}$}, am putea obține
	\begin{center}
	$\{\type{x}{\alpha, \type{y}{\alpha \to \beta}}\} \vdash \type{\app{(\abs{\type{z}{\beta}}{\abs{\type{u}{\gamma}}{z}})}{(\app{y}{x})}}{\gamma \to \beta}$
	\end{center}
	\vspace{.2cm}
	\item Pentru termenul {\color{False}$\app{x}{x}$} nu putem să rezolvăm problema.
\end{itemize}
\end{frame}

%------------------------------------------------
\begin{frame}{Tipuri simple}

%Mulțimea  \alert{tipurilor simple}  \hspace{.2cm}
% \intens{$\mathbb{T} = \mathbb{V}\ |\ \mathbb{T} \rightarrow \mathbb{T}$}

%Mulțimea \alert{$\lambda$-termenilor cu pre-tipuri $\Lambda_\mathbb{T}$} \hspace{.2cm}
%\intens{$\Lambda_\mathbb{T} = x\ |\ \app{\Lambda_\mathbb{T}}{\Lambda_\mathbb{T}}\ |\ \abs{\type{x}{\mathbb{T}}}{\Lambda_\mathbb{T}}$}

\vspace{.2cm}
\begin{minipage}{15cm}
\hspace{-1cm}
\begin{tabular}{cc}
\alert{Sistemul $\lambda\hspace{-.1cm}\to$} & 
\onslide<2->{\alert{Sistemul $\lambda\hspace{-.1cm}\to$ cu constrângeri}}
\\[.6em] 
$\Gamma \vdash \type{M}{\sigma}$
& 
\onslide<3->{$\Gamma \vdash \type{M}{\sigma} \const{C}$}
\\[1em] 
%------------------
\infer[(var)]
	{\Gamma \vdash \type{x}{\sigma}}
	{} dacă $\type{x}{\sigma} \in \Gamma$
 &  
 \onslide<4->{
\infer[(var^*)]
	{\Gamma \cup \{\type{x}{\tau}\}\vdash \type{x}{\sigma} \const{\{\sigma \egf{} \tau\}}}
	{} 
 }
\\[1em] 
%------------------
\infer[(\to_I)]
	{\Gamma \vdash \type{(\abs{\type{x}{\sigma}}{M})}{\sigma \to \tau} }
	 {\Gamma, \type{x}{\sigma} \vdash \type{M}{\tau}}
  & 
\onslide<5->{
 \infer[(\to_I^*)]
	{\Gamma \vdash \type{(\abs{\type{x}{\sigma}}{M})}{\tau} \const{C} }
	 {\Gamma, \type{x}{\sigma} \vdash \type{M}{\tau'} \const{C'} \vspace{.1cm}\\ {\color{False} C = C' \cup \{\tau \egf{} \sigma \to \tau'\}}}
	 }
\\[1em] 
 %------------------
\infer[(\to_E)]
	{\Gamma \vdash \type{\app{M}{N}}{\tau}}
	 {\Gamma \vdash \type{M}{\sigma \to \tau} \hspace{.5cm} \Gamma \vdash \type{N}{\sigma}}
  & 
\onslide<6->{
\hspace{.4cm}
\infer[(\to_E^*)]
	{\Gamma \vdash \type{\app{M}{N}}{\tau} \const{C}}
	 {\Gamma \vdash \type{M}{\tau_1}\const{C_1} \hspace{.5cm} \Gamma \vdash \type{N}{\tau_2} \const{C_2} \vspace{.1cm}\\  {\color{False} C = C_1 \cup C_2 \cup \{\tau_1 \egf{} \tau_2 \to \tau\}}}
}
\\[1em]
\onslide<7->{$\sigma, \tau$ variabile de tip}
& 
\onslide<7->{$\sigma, \tau, \tau', \tau_1, \tau_2$ variabile de tip}
\end{tabular}
\end{minipage}

\end{frame}

%%------------------------------------------------
%\begin{frame}{Exemplu}
%\begin{center}
%{\footnotesize
%\begin{minipage}{10cm}
%\begin{center}
%\begin{tabular}{cc}
%\infer[(var^*)]
%	{\Gamma \cup \{\type{x}{\tau}\}\vdash \type{x}{\sigma} \const{\{\sigma = \tau\}}}
%	{} 
%&
%% \infer[(\to_I^*)]
%%	{\Gamma \vdash \type{(\abs{\type{x}{\sigma}}{M})}{\sigma \to \tau} \const{C} }
%%	 {\Gamma, \type{x}{\sigma} \vdash \type{M}{\tau} \const{C}}
% \infer[(\to_I^*)]
%	{\Gamma \vdash \type{(\abs{\type{x}{\sigma}}{M})}{\tau} \const{C} }
%	 {\Gamma, \type{x}{\sigma} \vdash \type{M}{\tau'} \const{C'} \vspace{.1cm}\\ {\color{False} C = C' \cup \{\tau = \sigma \to \tau'\}}}
%\end{tabular}
%
%\vspace{.6cm}
%\begin{tabular}{c}
%\infer[(\to_E^*)]
%	{\Gamma \vdash \type{\app{M}{N}}{\tau} \const{C}}
%	 {\Gamma \vdash \type{M}{\tau_1}\const{C_1} \hspace{.5cm} \Gamma \vdash \type{N}{\tau_2} \const{C_2} \vspace{.1cm}\\  {\color{False} C = C_1 \cup C_2 \cup \{\tau_1 = \tau_2 \to \tau\}}}
%\end{tabular}
%\end{center}
%\end{minipage}
%}
%\end{center}
%
%\vspace{.2cm}
%{\footnotesize
%\begin{center}
%\begin{tabular}{c}
%\hspace{-.6cm}
%\infer[(\to_E^*)]
%	{
%	\onslide<1->{\underbrace{\{\type{x}{\alpha, \type{y}{\alpha \to \beta}}\}}_{\Gamma} \vdash \type{\app{(\abs{\type{z}{\beta}}{\abs{\type{u}{\gamma}}{z}})}{(\app{y}{x})}}{\gamma \to \beta} \const{C}}
%	}
%	{
%		\onslide<2->{
%	\infer[(\to_I^*)]
%	{\Gamma  \vdash 	\type{\abs{\type{z}{\beta}}{\abs{\type{u}{\gamma}}{z}}}{\tau_1} \const{C_1}}
%	{ 
%	\infer[(\to_I^*)]
%	{\Gamma \cup \{\type{z}{\beta}\} \vdash \type{\abs{\type{u}{\gamma}}{z}}{\tau_1'} \const{C_1'}}
%	{\Gamma \cup \{\type{z}{\beta}, \type{u}{\gamma}\} \vdash \type{z}{\delta} \const{D}  \vspace{.1cm}\\ 
%	{\color{False} C_1' = D \cup \{\tau_1' = (\gamma \to \delta)\}}
%	}
%	 \vspace{.1cm}\\ 
%	{\color{False} C_1 = C_1' \cup \{\tau_1 = (\beta \to \tau_1')\} }
%	}
%	}
%	\quad \quad
%		\onslide<2->{
%	\infer[(\to_E^*)]
%	{\Gamma  \vdash \type{\app{y}{x}}{\tau_2} \const{C_2}}
%	{
%	\Gamma  \vdash \type{y}{\sigma_1} \const{C_2'} \quad
%	\Gamma \vdash \type{x}{\sigma_2} \const{C_2''} \vspace{.1cm}\\ 
%	{\color{False} C_2 = C_2' \cup C_2'' \cup \{\sigma_1 = \sigma_2 \to \tau_2\}}	
%	}
%	 \vspace{.2cm}\\ 
%	{\color{False} C = C_1 \cup C_2 \cup \{\tau_1 = (\tau_2 \to (\gamma \to \beta))}\}}
%	}
%\end{tabular}
%\end{center}
%}
%\end{frame}
%
%%------------------------------------------------
%\begin{frame}{Exemplu}
%\vspace{-.2cm}
%\begin{center}
%{\footnotesize
%\begin{minipage}{10cm}
%\begin{center}
%\begin{tabular}{c}
%\infer[(var^*)]
%	{\Gamma \cup \{\type{x}{\tau}\}\vdash \type{x}{\sigma} \const{\{\sigma = \tau\}}}
%	{} 
%\end{tabular}
%\end{center}
%\end{minipage}
%}
%\end{center}
%
%\vspace{-.6cm}
%{\footnotesize
%\begin{center}
%\begin{tabular}{c}
%\hspace{-.6cm}
%\infer[(\to_E^*)]
%	{
%	\onslide<1->{\underbrace{\{\type{x}{\alpha, \type{y}{\alpha \to \beta}}\}}_{\Gamma} \vdash \type{\app{(\abs{\type{z}{\beta}}{\abs{\type{u}{\gamma}}{z}})}{(\app{y}{x})}}{\gamma \to \beta} \const{C}}
%	}
%	{
%		\onslide<1->{
%	\infer[(\to_I^*)]
%	{\Gamma  \vdash 	\type{\abs{\type{z}{\beta}}{\abs{\type{u}{\gamma}}{z}}}{\tau_1} \const{C_1}}
%	{ 
%	\infer[(\to_I^*)]
%	{\Gamma \cup \{\type{z}{\beta}\} \vdash \type{\abs{\type{u}{\gamma}}{z}}{\tau_1'} \const{C_1'}}
%	{\Gamma \cup \{\type{z}{\beta}, \type{u}{\gamma}\} \vdash \type{z}{\delta} \const{D}  \vspace{.1cm}\\ 
%	{\color{False} C_1' = D \cup \{\tau_1' = (\gamma \to \delta)\}}
%	}
%	 \vspace{.1cm}\\ 
%	{\color{False} C_1 = C_1' \cup \{\tau_1 = (\beta \to \tau_1')\} }
%	}
%	}
%	\quad \quad
%		\onslide<1->{
%	\infer[(\to_E^*)]
%	{\Gamma  \vdash \type{\app{y}{x}}{\tau_2} \const{C_2}}
%	{
%	\Gamma  \vdash \type{y}{\sigma_1} \const{C_2'} \quad
%	\Gamma \vdash \type{x}{\sigma_2} \const{C_2''} \vspace{.1cm}\\ 
%	{\color{False} C_2 = C_2' \cup C_2'' \cup \{\sigma_1 = \sigma_2 \to \tau_2\}}	
%	}
%	 \vspace{.2cm}\\ 
%	{\color{False} C = C_1 \cup C_2 \cup \{\tau_1 = (\tau_2 \to (\gamma \to \beta))}\}}
%	}
%\end{tabular}
%\end{center}
%
%\vspace{-.4cm}
%\begin{tabular}{cc}
%\hspace{-1cm}
%\begin{tabular}{l}
%$D= \{\delta = \beta\}$  \\
%$C_1' = \{\delta = \beta, \tau_1' = (\gamma \to \delta)\}$ \\
%$C_1 = \{\delta = \beta, \tau_1' = (\gamma \to \delta), \tau_1 = (\beta \to \tau_1')\}$ 
%\end{tabular}
%&
%\begin{tabular}{l}
%$C_2' = \{\sigma_1 = \alpha \to \beta\}$  \\
%$C_2'' = \{\sigma_2 = \alpha\}$ \\
%$C_2 = \{\sigma_1 = \alpha \to \beta, \sigma_2 = \alpha, \sigma_1 = \sigma_2 \to \tau_2\}$
%\end{tabular} 
%\end{tabular}
%\begin{center}
%$C = \{\delta = \beta, \tau_1' = (\gamma \to \delta), \tau_1 = (\beta \to \tau_1'), \sigma_1 = \alpha \to \beta, \sigma_2 = \alpha,$ \\ $\sigma_1 = \sigma_2 \to \tau_2, \tau_1 = (\tau_2 \to (\gamma \to \beta))\}$
%\end{center}
%
%  \vspace{-.2cm}
%%\hspace{-.4cm}\intens{Nu vă speriați!} Constrângerile de mai sus spun doar că ar trebui ca $\alpha = \sigma_2$ și $\beta = \delta$.
%\hspace{-.4cm}\intens{Nu vă speriați!} e mai simplu decât pare.
%}
%\end{frame}

%------------------------------------------------
\begin{frame}{Sistemul $\lambda\hspace{-.1cm}\to$ cu constrângeri}

O judecată de forma $\Gamma \vdash \type{M}{\sigma} \const{C}$ este \alert{legală} dacă constrângerile din $C$ au o "soluție".

\vspace{.2cm}
De exemplu, judecata de mai jos este legală
\vspace{-.2cm}
\begin{center}
$\{\type{x}{\alpha, \type{y}{\alpha \to \beta}}\} \vdash \type{\app{(\abs{\type{z}{\beta}}{\abs{\type{u}{\gamma}}{z}})}{(\app{y}{x})}}{\gamma \to \beta} \const{C}$, unde\\ 
$C = \{\delta \egf{} \beta, \tau_1' \egf{} (\gamma \to \delta), \tau_1 \egf{} (\beta \to \tau_1'), \sigma_1 \egf{} \alpha \to \beta, \sigma_2 \egf{} \alpha,$ \\ $\sigma_1 \egf{} \sigma_2 \to \tau_2, \tau_1 \egf{} (\tau_2 \to (\gamma \to \beta))\}$
\end{center}

$C$ are "soluție" și spune, de exemplu, că ar trebui ca $\alpha = \sigma_2$ și $\beta = \delta$. 

În slide-urile următoare dăm mai multe detalii despre ce înseamnă acest lucru.
\end{frame}

%------------------------------------------------
\begin{frame}{Type Inference}

Fie $M$ un lambda termen fără tipuri.

Construim un context \alert{$\Gamma_M$} pentru $M$:
\vspace{-.2cm}
\begin{center}
$\Gamma_M = \{\type{x}{X}\ |\ x \in FV(M)\}$
\end{center}
\vspace{-.2cm}
(toate variabilele de tip $X$ introduse mai sus sunt noi și distincte)


  \vspace{.2cm}
Adnotăm $M$ cu tipuri pentru variabilele legate obținând \alert{$\overline{M}$} prin inducție după structura lui $M$ astfel:
\vspace{-.2cm}
\begin{itemize}
	\item dacă $M = x$, atunci $\overline{M} = M$
	\item dacă $M = \app{M_1}{N_1}$, atunci $\overline{M} = \app{\overline{M_1}}{\overline{N_1}}$
	\item dacă $M = \abs{x}{N}$, atunci $\overline{M} = \abs{\type{x}{X}}{\overline{N}}$, unde $X$ este o variabilă de tip nouă
\end{itemize}
\end{frame}

%------------------------------------------------
\begin{frame}{Type Inference}

Fie $M$ un lambda termen fără tipuri.

Dacă există o constrângere de tipuri \alert{$C_M$} și o variabilă de tip nouă $V$ astfel  încât 
\vspace{-.2cm}
\begin{center}
$\Gamma_M \vdash \type{\overline{M}}{V} \const{C_M}$
\end{center}
\vspace{-.2cm}
este o judecată legală, atunci $M$ este \textit{typable}. \\
(soluția o găsim prin $C_M$)
\end{frame}

%------------------------------------------------
\begin{frame}{Type Inference - Exemplul 1}

{\footnotesize

Fie $M_1 = \app{(\abs{z}{\abs{u}{z}})}{(\app{y}{x})}$.  \\
Obținem $\Gamma_{M_1} = \{\type{x}{X}, \type{y}{Y}\}$ și $\overline{M_1} = \app{(\abs{\type{z}{Z}}{\abs{\type{u}{U}}{z}})}{(\app{y}{x})}$.

\vspace{-.4cm}
\begin{center}
\begin{tabular}{c}
\hspace{-.6cm}
\infer[(\to_E^*)]
	{
	\onslide<1->{\Gamma_{M_1} \vdash \type{\app{(\abs{\type{z}{Z}}{\abs{\type{u}{U}}{z}})}{(\app{y}{x})}}{V} \const{C_{M_1}}}
	}
	{
		\onslide<1->{
	\infer[(\to_I^*)]
	{\Gamma_{M_1}  \vdash 	\type{\abs{\type{z}{Z}}{\abs{\type{u}{U}}{z}}}{\tau_1} \const{C_1}}
	{ 
	\infer[(\to_I^*)]
	{\Gamma_{M_1} \cup \{\type{z}{Z}\} \vdash \type{\abs{\type{u}{U}}{z}}{\tau_1'} \const{C_1'}}
	{\Gamma_{M_1} \cup \{\type{z}{Z}, \type{u}{U}\} \vdash \type{z}{\delta} \const{D}  \vspace{.1cm}\\ 
	{\color{False} C_1' = D \cup \{\tau_1' \egf{} (U \to \delta)\}}
	}
	 \vspace{.1cm}\\ 
	{\color{False} C_1 = C_1' \cup \{\tau_1 \egf{} (Z \to \tau_1')\} }
	}
	}
	\quad \quad
		\onslide<1->{
	\infer[(\to_E^*)]
	{\Gamma_{M_1}  \vdash \type{\app{y}{x}}{\tau_2} \const{C_2}}
	{
	\Gamma_{M_1}  \vdash \type{y}{\sigma_1} \const{C_2'} \quad
	\Gamma_{M_1} \vdash \type{x}{\sigma_2} \const{C_2''} \vspace{.1cm}\\ 
	{\color{False} C_2 = C_2' \cup C_2'' \cup \{\sigma_1 \egf{} \sigma_2 \to \tau_2\}}	
	}
	 \vspace{.2cm}\\ 
	{\color{False} C_{M_1} = C_1 \cup C_2 \cup \{\tau_1 \egf{} (\tau_2 \to V)}\}}
	}
\end{tabular}
\end{center}

\vspace{-.4cm}
\begin{tabular}{cc}
\hspace{-1cm}
\begin{tabular}{l}
$D= \{\delta \egf{} Z\}$  \\
$C_1' = \{\delta \egf{} Z, \tau_1' \egf{} (U \to \delta)\}$ \\
$C_1 = \{\delta \egf{} Z, \tau_1' \egf{} (U \to \delta), \tau_1 \egf{} (Z \to \tau_1')\}$ 
\end{tabular}
&
\begin{tabular}{l}
$C_2' = \{\sigma_1 \egf{} Y\}$  \\
$C_2'' = \{\sigma_2 \egf{} X\}$ \\
$C_2 = \{\sigma_1 \egf{} Y, \sigma_2 \egf{} X, \sigma_1 \egf{} \sigma_2 \to \tau_2\}$
\end{tabular} 
\end{tabular}
\begin{center}
$C_{M_1} = \{\delta \egf{} Z, \tau_1' \egf{} (U \to \delta), \tau_1 \egf{} (Z \to \tau_1'), \sigma_1 \egf{} Y, \sigma_2 \egf{} X,$ \\ $\sigma_1 \egf{} \sigma_2 \to \tau_2, \tau_1 \egf{} (\tau_2 \to V)\}$
\end{center}

Constrângerile $C_{M_1}$ au "soluție". Ce înseamnă asta?
}
\end{frame}

%------------------------------------------------
\begin{frame}{Type Inference - Exemplul 2}

{\footnotesize

Fie $M_2 = \app{x}{x}$.  \\
Obținem $\Gamma_{M_2} = \{\type{x}{X}\}$ și $\overline{M_2} = M_2$.

\begin{center}
\begin{tabular}{c}
\infer[(\to_E^*)]
	{ \{\type{x}{X}\} \vdash \type{(\app{x}{x})}{V} \const{C_{M_2}}
	}
	{
	\{\type{x}{X}\} \vdash \type{x}{\tau_1}\const{C_1} \quad \{\type{x}{X}\} \vdash \type{x}{\tau_2} \const{C_2} \vspace{.1cm}\\  {\color{False} C_M = C_1 \cup C_2 \cup \{\tau_1 \egf{} \tau_2 \to V\}}
	}
\end{tabular}
\end{center}

\begin{tabular}{l}
$C_1 = \{\tau_1 \egf{} X\}$  \\
$C_2 = \{\tau_2 \egf{} X\}$ \\
$C_{M_2} = \{\tau_1 \egf{} X, \tau_2 \egf{} X, \tau_1 \egf{} \tau_2 \to V\}$
\end{tabular} 

Constrângerile $C_{M_2}$ nu au "soluție". Ce înseamnă asta?

\pause
Constrângerile au "soluție" dacă se pot \intens{unifica}.
}
\end{frame}

%================================================
\section{\color{section-color} Unificare}
%================================================

\begin{frame}{Termeni}

\intens{Alfabet:}
\vspace{-.2cm}
\begin{itemize}
	\item $\mathcal{F}$ o mulțime de simboluri de funcții de aritate cunoscută
	\item $\mathcal{V}$ o mulțime numărabilă de variabile 
	\item $\mathcal{F}$ și $\mathcal{V}$ sunt disjuncte
\end{itemize}

  \bigskip
\intens{Termeni peste $\mathcal{F}$ si $\mathcal{V}$:}
\vspace{-.2cm}
\begin{align*}
t \ ::= \ x\ |\ f(t_1,\ldots,t_n)
\end{align*}
\vspace{-1cm}
\begin{itemize}
	\item $n \geq 0$
	\item $x$ este o variabilă
	\item $f$ este un simbol de funcție de aritate $n$ 
\end{itemize}

  \bigskip
Pentru ușurință, considerăm funcțiile în forma prefixată.
\end{frame}

%---------------------------------------------------------------------
\begin{frame}{Termeni}
\intens{Notații:}
\vspace{-.2cm}
\begin{itemize}
	\item \intens{constante:} simboluri de funcții de aritate $0$
	\item $x,y,z,\ldots$ pentru variabile
	\item $a,b,c,\ldots$ pentru constante
	\item $f,g,h,\ldots$ pentru simboluri de funcții arbitrare
	\item $s,t,u,\ldots$ pentru termeni
	\item $var(t)$ mulțimea variabilelor care apar în $t$
	\item ecuații $s \egf{} t$ pentru o pereche de termeni
	\item $Trm_{\mathcal{F,\mathcal{V}}}$ mulțimea termenilor peste $\mathcal{F}$ și $\mathcal{V}$
\end{itemize}

\end{frame}

%---------------------------------------------------------------------
\begin{frame}{Termeni}

{\color{True} Exemple:}
\begin{itemize}
	\item $f(x,g(x,a),y)$ este un termen, unde $f$ are aritate $3$, $g$ are aritate $2$, $a$ este o constanta
	\item $var(f(x,g(x,a),y)) = \{x,y\}$
\end{itemize}
\end{frame}

%---------------------------------------------------------------------
\begin{frame}{Legătura cu teoria tipurilor}

Mulțimea  \alert{tipurilor simple}  \hspace{.2cm}
 \intens{$\mathbb{T} = \mathbb{V}\ |\ \mathbb{T} \rightarrow \mathbb{T}$}
 
În acest caz, avem alfabetul:
\vspace{-.2cm}
\begin{itemize}
	\item $\mathcal{F} = \{\to\}$, iar aritatea lui $\to$ este $2$
	\item $\mathcal{V} = \mathbb{V}$
\end{itemize}

\vspace{.4cm}
Dacă avem și alte tipuri, extindem $\mathcal{F}$ cu noi simboluri. De exemplu,
\vspace{-.2cm}
\begin{itemize}
	\item $\UnitT, \VoidT$ cu aritate $0$ (deci constante)
	\item $\BoolT, \texttt{Nat}$ cu aritate $0$ (deci constante)
	\item $\texttt{Maybe}, \texttt{List}$ cu aritate $1$
	\item $\times$ cu aritate $2$
	\item $\ldots$
\end{itemize}


\end{frame}

%---------------------------------------------------------------------
\begin{frame}{Substituții}

O \intens{substituție} $\Theta$ este o funcție (parțială) de la variabile la termeni,
\vspace{-.2cm}
	\begin{center}
	\intens{$\Theta: \mathcal{V} \to Trm_{\mathcal{F,\mathcal{V}}}$}
	\end{center}

\medskip
{\color{True} Exemplu:}

În notația uzuală, $\Theta = \{ a/x,\ g(w)/y, b/z\}$. 
%În notația uzuală, $\Theta = \{ x/a,\ y/g(w), z/b\}$. 

Substituția $\Theta$ este identitate pe restul variabilelor.

Notație alternativă $\Theta = \{x \mapsto a, y \mapsto g(w), z \mapsto b\}$.


\end{frame}
%---------------------------------------------------------------------

%---------------------------------------------------------------------
\begin{frame}{Substituții}

Substituțiile sunt o modalitate de a înlocui variabile cu alți termeni.

Substituțiile se aplică simultan pe toate variabilele.


\medskip
\intens{Aplicarea unei substituții} $\Theta$ unui termen $t$:
\begin{align*}
\Theta(t) = 
\left\{
	\begin{array}{ll}
		\Theta(x), \mbox{ dacă } t = x \\
		f(\Theta(t_1),\ldots,\Theta(t_n)), \mbox{ dacă } t = f(t_1,\ldots,t_n) \\
	\end{array}
\right. 
\end{align*}

 {\color{True} Exemplu:}
\begin{itemize}
	\item $\Theta = \{x \mapsto f(x,y), y \mapsto g(a) \}$
	\item $t = f(x,g(f(x,f(y,z))))$
	\item $\Theta(t) = f(f(x,y),g(f(f(x,y),f(g(a),z))))$
\end{itemize}
\end{frame}
%---------------------------------------------------------------------

%---------------------------------------------------------------------
\begin{frame}{Substituții}

\vspace{.2cm}
Două substituții $\Theta_1$ și $\Theta_2$ se pot compune \[\intens{\Theta_1 ; \Theta_2}\] \\(aplicăm întâi $\Theta_1$, apoi $\Theta_2$).

 
 {\color{True} Exemplu:}
	\begin{itemize}
		\item $\intens{t = h(u,v,x,y,z)}$
		 
		\item \intens{$\Theta_1 = \{ x \mapsto f(y),\ y \mapsto f(a),\ z \mapsto u \}$}
		\item \intens{$\Theta_2 = \{y \mapsto g(a),\ u \mapsto z,\ v \mapsto f(f(a))\}$}
		 	\medskip 
		\item $\intens{(\Theta_1 ; \Theta_2)(t) =}\  {\Theta_2}({\Theta_1}(t)) = {\Theta_2}( h(u, v, f(y), f(a), u)) = $ \\ \hspace{1.4cm}$\intens{= h(z,f(f(a)), f(g(a)), f(a), z)}$	
		  \medskip
		\item $\intens{(\Theta_2 ; \Theta_1)(t) = }\ {\Theta_1}({\Theta_2}(t)) = {\Theta_1}(h(z,f(f(a)),x, g(a), z))$ \\ \hspace{1.4cm} $\intens{= h(u,f(f(a)), f(y), g(a), u)}$
	\end{itemize}


\end{frame}
%---------------------------------------------------------------------

%---------------------------------------------------------------------
\begin{frame}{Unificare}

 Doi termeni $t_1$ și $t_2$ \intens{se unifică} dacă există o substituție $\Theta$ astfel încât
 \vspace{-.2cm}
	\begin{center}
	\intens{$\Theta(t_1) = \Theta(t_2)$}.
	\end{center}
	
În acest caz, $\Theta$ se numește un \intens{unificator} al termenilor $t_1$ și $t_2$.
	
	\medskip  
Un unificator $\Theta$ pentru $t_1$ și $t_2$ este \intens{cel mai general unificator} (\intens{cmgu,mgu}) dacă pentru orice alt unificator $\Theta'$ pentru $t_1$ și $t_2$, există o substituție $\Delta$ astfel încât
 \vspace{-.2cm}
	\begin{center}
	\intens{$\Theta' = \Theta ; \Delta$}.
	\end{center}
\end{frame}
%---------------------------------------------------------------------

%------------------------------------------------------------------------
\begin{frame}{Unificatori}

 {\color{True} Exemplu:}
\begin{itemize}
	\item $\intens{t = x + (y * y)} = +(x,*(y,y))$
	\vspace{.1cm}
	\item $\intens{t' = x + (y * x)} = +(x,*(y,x))$
	 
	\vspace{.1cm}
	\item $\Theta = \{x \mapsto y\}$
	\begin{itemize}
		\item $\Theta(t) = y + (y * y)$
		\item $\Theta(t') = y + (y * y)$
		\item $\Theta$ este \intens{cmgu}
	\end{itemize}
	 
	\vspace{.1cm}
	\item $\Theta'= \{x \mapsto 0, y \mapsto 0\}$
		\begin{itemize}
		\item $\Theta'(t) = 0 + (0 * 0)$
		\item $\Theta'(t') = 0 + (0 * 0)$
		 
		\item $\Theta' = \Theta ; \{y \mapsto 0\}$
		 
		\item $\Theta'$ este \intens{unificator}, dar nu este \intens{cmgu}
	\end{itemize}
\end{itemize}

\end{frame}

%%------------------------------------------------------------------------
%\begin{frame}{Algoritmul de unificare}
%\begin{itemize}
%	\item Pentru o mulțime finită de termeni \intens{$\{t_1,\ldots, t_n\}$, $n \geq 2$}, \intens{algoritmul de unificare }stabilește dacă există un cmgu.
%	\vspace{.2cm}
%	\item Există algoritmi mai eficienți, dar îl alegem pe acesta pentru simplitatea sa.
%	\vspace{.2cm}  
%	\item Algoritmul lucrează cu două liste:
%	\begin{itemize}
%		\item \intens{Lista soluție: $S$}
%		\item \intens{Lista de rezolvat: $R$}		
%	\end{itemize}
%	\vspace{.2cm}  
%	\item \intens{Inițial:}
%	\begin{itemize}
%		\item \intens{Lista soluție: $S = \emptyset$}
%		\item \intens{Lista de rezolvat: $R = \{t_1 \egf{} t_2, \ldots, t_{n-1} \egf{} t_n\}$}	 \\
%		\intens{$\egf{}$} este un simbol nou care ne ajută să formăm perechi de termeni ("ecuații")	
%	\end{itemize}
%\end{itemize}
%\end{frame}
%
%%------------------------------------------------------------------------
%\begin{frame}{Algoritmul de unificare}
%Algoritmul constă în aplicarea regulilor de mai jos:
%\begin{itemize}
%	\vspace{.2cm}  
%	\item \intens{SCOATE}
%		\begin{itemize}
%			\item orice ecuație de forma \intens{$t \egf{} t$} din \intens{$R$} este \intens{eliminată}.
%		\end{itemize}
%	\vspace{.2cm}  
%	\item \intens{DESCOMPUNE}
%		\begin{itemize}
%			\item orice ecuație de forma \intens{$f(t_1,\ldots,t_n) \egf{} f(t_1',\ldots,t_n')$} din \intens{$R$} este \intens{înlocuită} cu ecuațiile \intens{$t_1\egf{}t_1', \ldots, t_n \egf{}t_n'$}.
%		\end{itemize}
%	\vspace{.2cm}  
%	\item \intens{REZOLVĂ}
%	\begin{itemize}
%			\item orice ecuație de forma \intens{$x \egf{} t$} sau \intens{$t \egf{} x$} din \intens{$R$}, unde \intens{variabila $x$ nu apare în termenul $t$}, este \intens{mutată} sub forma \intens{$x \egf{} t$} în \intens{$S$}. \\\intens{\^In toate celelalte ecuații} (din $R$ și $S$), \intens{$x$ este înlocuit cu $t$}.	
%		\end{itemize}
%\end{itemize}
%\end{frame}
%
%%------------------------------------------------------------------------
%\begin{frame}{Algoritmul de unificare}
%Algoritmul \intens{se termină normal} dacă \intens{$R = \emptyset$}.\\
% În acest caz, \intens{$S$ conține cmgu}.
%
%\vspace{.5cm}  
%Algoritmul este oprit cu concluzia {\color{False} inexistenței unui unificator} dacă:
%\begin{enumerate}
%	\item În $R$ există o ecuație de forma
%	\begin{center}
%	{\color{False}$f(t_1,\ldots,t_n) \egf{} g(t_1',\ldots,t_k')$} cu {\color{False}$f \neq g$}.
%	\end{center}
%	\item În $R$ există o ecuație de forma {\color{False}$x \egf{} t$} sau {\color{False}$t \egf{} x$} și {\color{False}variabila $x$ apare în termenul $t$}.
%\end{enumerate}
%\end{frame}
%
%%------------------------------------------------------------------------
%\begin{frame}{Algoritmul de unificare - schemă}
%
%\begin{center}
%\begin{minipage}{15cm}
%\hspace{-.6cm}
%\begin{tabular}{|c|c|c|}
%\hline
%& \intens{Lista soluție} & \intens{Lista de rezolvat} \\
%& \intens{S} & \intens{R} \\  \hline \hline
%\intens{Inițial} & $\emptyset$ & $t_1 \egf{} t_1',\ldots, t_n \egf{} t_n'$ \\  \hline \hline
%\intens{SCOATE} & $S$ & $R'$, \intens{$t \egf{} t$} \\  \cline{2-3} 
% & $S$ & $R'$ \\ \hline \hline
% \intens{DESCOMPUNE} & $S$ & $R'$, \intens{$f(t_1,\ldots,t_n) \egf{} f(t_1',\ldots,t_n')$} \\ \cline{2-3}
% & $S$ & $R'$, \intens{$t_1\egf{}t_1', \ldots t_n \egf{}t_n'$} \\ \hline \hline
% \intens{REZOLVĂ} & $S$ & $R'$, \intens{$x \egf{} t$} sau \intens{$t \egf{} x$}, $x$ nu apare în $t$ \\ \cline{2-3}
% & \intens{$x \egf{} t$}, \intens{$S[x / t]$} & \intens{$R'[x / t]$} \\ \hline \hline
% \intens{Final} & $S$ & \intens{$\emptyset$} \\ \hline 
%\end{tabular}
%
%\vspace{.4cm}
%\intens{$S[x/ t]$}: în toate ecuațiile din $S$, $x$ este înlocuit cu $t$
%\end{minipage}
%\end{center}
%\end{frame}
%
%%------------------------------------------------------------------------
%\begin{frame}{Exemplul 1}
%\vspace{.4cm}
%
%
%{\color{True} Ecuațiile $\{g(y) \egf{}x,\ f(x,h(x),y) \egf{} f(g(z),w,z)\}$ au cmgu?}
%
%\pause
%{\footnotesize
%\begin{center}
%\begin{tabular}{|c|c|c|}
%\hline
%$S$ & $R$ & \\ \hline 
%$\emptyset$ & $g(y) \egf{} x,\ f(x,h(x),y) \egf{} f(g(z),w,z)$ & {\scriptsize   \intens{REZOLVĂ}}  \\ \hline  
%$\intens{x \egf{} g(y)}$ & $f(\intens{g(y)},h(\intens{g(y)}),y) \egf{} f(g(z),w,z)$   & {\scriptsize   \intens{DESCOMPUNE}}   \\ \hline  
%$x \egf{} g(y)$ & $g(y) \egf{} g(z),\ h(g(y)) \egf{} w,\ y \egf{} z$ & {\scriptsize   \intens{REZOLVĂ}}   \\ \hline  
%$\intens{w \egf{} h(g(y))},$ & $g(y) \egf{} g(z),\ y \egf{} z$ & {\scriptsize  \intens{REZOLVĂ}}   \\
%$\ x \egf{} g(y)$ & &   \\ \hline 
%$\intens{y \egf{} z}, x \egf{} g(\intens{z}),$ & $g(\intens{z}) \egf{} g(z)$ & {\scriptsize \intens{SCOATE}} \\
%$w \egf{} h(g(\intens{z}))$ & &   \\ \hline  
%$y \egf{} z, x \egf{} g(z), $ & $\emptyset$ & \\
%$w \egf{} h(g(z))$ & &   \\ \hline 
%\end{tabular}
%\end{center}
%}
%
%\intens{$\Theta = \{y \mapsto z,\ x \mapsto g(z),\ w \mapsto h(g(z)) \}$ este cmgu.} 
%
%
%\end{frame}
%
%%------------------------------------------------------------------------
%\begin{frame}{Exemplul 2}
%
%
%{\color{True} Ecuațiile $\{g(y) \egf{} x,\ f(x,h(y),y) \egf{} f(g(z),b,z)\}$ au cmgu?}
%
%
%\begin{center}
% {\footnotesize
%\begin{tabular}{|c|c|c|}
%\hline
%$S$ & $R$ & \\ \hline
%$\emptyset$ & $g(y) \egf{} x,\ f(x,h(y),y) \egf{} f(g(z),b,z)$ &{\scriptsize \intens{REZOLVĂ}} \\ \hline
%$\intens{x \egf{} g(y)}$ & $f(\intens{g(y)},h(y),y) \egf{} f(g(z),b,z)$   & {\scriptsize \intens{DESCOMPUNE}} \\ \hline
%$x \egf{} g(y)$ & $g(y) \egf{} g(z),\ {\color{False}h(y) \egf{} b},\ y \egf{} z$ & {\scriptsize {\color{False}- EȘEC - }} \\ \hline
%\end{tabular}
%}
%\end{center}
% 
% 
%\begin{itemize}
%	\item {\color{False}$h$ și $b$ sunt simboluri de funcții diferite!}
%	\item Nu există unificator pentru acești termeni.
%\end{itemize}
%
%\end{frame}
%
%%------------------------------------------------------------------------
%\begin{frame}{Exemplul 3}
%
%
%{\color{True} Ecuațiile $\{g(y) \egf{} x,\ f(x,h(x),y) \egf{} f(y,w,z)\}$ au cmgu?}
%
%
%\begin{center}
% {\footnotesize
%\begin{tabular}{|c|c|c|}
%\hline
%$S$ & $R$ & \\ \hline
%$\emptyset$ & $g(y) \egf{} x,\ f(x,h(x),y) \egf{} f(y,w,z)$ &{\scriptsize \intens{REZOLVĂ}} \\ \hline
%$\intens{x \egf{} g(y)}$ & $f(\intens{g(y)},h(\intens{g(y)}),y) \egf{} f(y,w,z)$   & {\scriptsize \intens{DESCOMPUNE}} \\ \hline
%$x \egf{} g(y)$ & ${\color{False}g(y) \egf{} y},\ h(g(y)) \egf{} w,\ y \egf{} z$ & {\scriptsize {\color{False}- EȘEC - }} \\ \hline
%\end{tabular}
%}
%\end{center}
% 
%\begin{itemize}
%	\item {\color{False}\^In ecuația ${g(y) \egf{} y}$, variabila $y$ apare în termenul $g(y)$.}
%	\item Nu există unificator pentru aceste ecuații.
%\end{itemize}
%\end{frame}
%
%
%%------------------------------------------------------------------------
%\begin{frame}{Exemplul 4}
%
%Înapoi la constrângerea obținută când am vorbit de \textit{type inference} pentru termenul {\color{True} $M_1 = \app{(\abs{z}{\abs{u}{z}})}{(\app{y}{x})}$}.
%
%Am obținut constrângerile
%\begin{center}
%\intens{$C_{M_1} = \{\delta \egf{} Z, \tau_1' \egf{} (U \to \delta), \tau_1 \egf{} (Z \to \tau_1'), \sigma_1 \egf{} Y, \sigma_2 \egf{} X,$ \\ $\sigma_1 \egf{} \sigma_2 \to \tau_2, \tau_1 \egf{} (\tau_2 \to V)\}$}
%\end{center}
%\begin{itemize}
%	\item $\to$ simbol de funcție de aritate $2$
%	\item $\delta, \tau_1, \tau_1', \tau_2, \sigma_1, \sigma_2, X,Y, Z, U, V$ variabile
%\end{itemize}
%
%
%\end{frame}
%
%%------------------------------------------------------------------------
%\begin{frame}{Exemplul 4 (cont.)}
%
%\vspace{-.4cm}
%\begin{center}
% {\footnotesize
% \begin{minipage}{15cm}
% \hspace{-.6cm}
%\begin{tabular}{|c|c|c|}
%\hline
%$S$ & $R$ & \\ \hline
%$\emptyset$ & $\intens{\delta} \egf{} Z,\ \tau_1' \egf{} (U \to \intens{\delta}),\ \tau_1 \egf{} (Z \to \tau_1'),\ \sigma_1 \egf{} Y$ &  {\scriptsize \intens{REZ.}} \\
%& $\sigma_2 \egf{} X,\ \sigma_1 \egf{} \sigma_2 \to \tau_2,\ \tau_1 \egf{} (\tau_2 \to V)$ & \\ \hline
%$\delta \egf{} Z$ & $\intens{\tau_1'} \egf{} (U \to Z),\ \tau_1 \egf{} (Z \to \intens{\tau_1'}),\ \sigma_1 \egf{} Y$ &  {\scriptsize \intens{REZ.}} \\
%& $\sigma_2 \egf{} X,\ \sigma_1 \egf{} \sigma_2 \to \tau_2,\ \tau_1 \egf{} (\tau_2 \to V)$ & \\ \hline
%$\delta \egf{} Z,\ \tau_1' \egf{} (U \to Z)$ & $\intens{\tau_1} \egf{} (Z \to (U \to Z)),\ \sigma_1 \egf{} Y$ &  {\scriptsize \intens{REZ.}} \\
%& $\sigma_2 \egf{} X,\ \sigma_1 \egf{} \sigma_2 \to \tau_2,\ \intens{\tau_1} \egf{} (\tau_2 \to V)$ & \\ \hline
%$\delta \egf{} Z,\ \tau_1' \egf{} (U \to Z),$ & $\sigma_1 \egf{} Y,\ \sigma_2 \egf{} X,\ \sigma_1 \egf{} \sigma_2 \to \tau_2,$ &  {\scriptsize \intens{DESC.}} \\
%$\tau_1 \egf{} (Z \to (U \to Z))$ & $(Z \to (U \to Z)) \egf{} (\tau_2 \to V)$ & \\ \hline
%$\delta \egf{} Z,\ \tau_1' \egf{} (U \to Z),$ & $\intens{\sigma_1} \egf{} Y,\ \sigma_2 \egf{} X,\ \intens{\sigma_1} \egf{} \sigma_2 \to \tau_2,$ &  {\scriptsize \intens{REZ.}} \\
%$\tau_1 \egf{} (Z \to (U \to Z))$ & $Z \egf{} \tau_2,\ U \to Z \egf{} V$ & \\ \hline
%$\delta \egf{} Z,\ \tau_1' \egf{} (U \to Z),$ & $\intens{\sigma_2} \egf{} X,\ Y \egf{} \intens{\sigma_2} \to \tau_2,$ &  {\scriptsize \intens{REZ.}} \\
%$\tau_1 \egf{} (Z \to (U \to Z)),$ & $Z \egf{} \tau_2,\ U \to Z \egf{} V$ & \\
%$\sigma_1 \egf{} Y$ & & \\ \hline
%$\delta \egf{} Z,\ \tau_1' \egf{} (U \to Z),$ & $Y \egf{} X \to \intens{\tau_2},$ &  {\scriptsize \intens{REZ.}} \\
%$\tau_1 \egf{} (Z \to (U \to Z)),$ & $Z \egf{} \intens{\tau_2},\ U \to Z \egf{} V$ & \\
%$\sigma_1 \egf{} Y, \sigma_2 \egf{} X$ & & \\ \hline
%
%\end{tabular}
%\end{minipage}
%}
%\end{center}
% 
%\end{frame}
%
%%------------------------------------------------------------------------
%\begin{frame}{Exemplul 4 (cont.)}
%
%\vspace{-.4cm}
%\begin{center}
% {\footnotesize
% \begin{minipage}{15cm}
% \hspace{-.6cm}
%\begin{tabular}{|c|c|c|}
%\hline
%$S$ & $R$ & \\ \hline
%$\delta \egf{} Z,\ \tau_1' \egf{} (U \to Z),$ & $\intens{Y} \egf{} X \to Z, \ U \to Z \egf{} V$ &  {\scriptsize \intens{REZ.}} \\
%$\tau_1 \egf{} (Z \to (U \to Z)),$ & $$ & \\
%$\sigma_1 \egf{} \intens{Y}, \sigma_2 \egf{} X,\ \tau_2 \egf{} Z$ & & \\ \hline
%$\delta \egf{} Z,\ \tau_1' \egf{} (U \to Z),$ & $U \to Z \egf{} V$ &  {\scriptsize \intens{REZ.}} \\
%$\tau_1 \egf{} (Z \to (U \to Z)),$ & $$ & \\
%$\sigma_1 \egf{} X \to Z, \sigma_2 \egf{} X,\ \tau_2 \egf{} Z$ & & \\
%$Y \egf{} X \to Z$ & & \\ \hline
%$\delta \egf{} Z,\ \tau_1' \egf{} (U \to Z),$ &  &   \\
%$\tau_1 \egf{} (Z \to (U \to Z)),$ & $$ & \\
%$\sigma_1 \egf{} X \to Z, \sigma_2 \egf{} X,\ \tau_2 \egf{} Z$ & & \\
%$Y \egf{} X \to Z,\ V \egf{} U \to Z $ & & \\ \hline
%
%\end{tabular}
%\end{minipage}
%}
%\end{center}
%
%{\footnotesize \intens{Constrângerile se pot unifica!}}
%\end{frame}
%
%%------------------------------------------------------------------------
%\begin{frame}{Exemplul 5}
%
%Înapoi la constrângerea obținută când am vorbit de \textit{type inference} pentru termenul {\color{False}$M_2 = \app{x}{x}$}.
%
%Am obținut constrângerile
%\begin{center}
%\intens{$C_{M_2} = \{\tau_1 \egf{} X, \tau_2 \egf{} X, \tau_1 \egf{} \tau_2 \to V\}$}
%\end{center}
%\vspace{-.2cm}
%\begin{itemize}
%	\item $\to$ simbol de funcție de aritate $2$
%	\item $\tau_1,  \tau_2, V$ variabile
%\end{itemize}
%
%
%\end{frame}
%
%%------------------------------------------------------------------------
%\begin{frame}{Exemplul 5 (cont.)}
%
%\vspace{-.4cm}
%
% {\footnotesize
% \begin{center}
%\begin{tabular}{|c|c|c|}
%\hline
%$S$ & $R$ & \\ \hline
%$\emptyset$ & $\intens{\tau_1} \egf{} X,\ \tau_2 \egf{} X,\ \intens{\tau_1} \egf{} \tau_2 \to V$ &  {\scriptsize \intens{REZ.}} \\  \hline
%$\tau_1 \egf{} X$ & $\intens{\tau_2} \egf{} X,\ X \egf{} \intens{\tau_2} \to V$ &  {\scriptsize \intens{REZ.}} \\  \hline
%$\tau_1 \egf{} X,\ \tau_2 \egf{} X$ & ${\color{False}X \egf{} X \to V}$ &  {\scriptsize {\color{False}- EȘEC -}} \\  \hline
%\end{tabular}
%\end{center}
%}
%
% \begin{itemize}
%	\item {\color{False}\^In ecuația ${X \egf{} X \to V}$, variabila $X$ apare în termenul $X \to V$.}
%	\item Nu există unificator pentru aceste ecuații.
%\end{itemize}
%\end{frame}
%

%------------------------------------------------
\begin{frame}
  \vfill
  \centering

\textbf{\large \alert{Quiz time!}}

\includegraphics[scale=.35]{../Quiz/C07-Q1.png}

 \url{https://tinyurl.com/C07-Quiz1}
  \vfill
\end{frame}

%---------------------------------------------
\begin{frame}
  \vfill
  \centering

\textbf{Vacanță plăcută!}

  \vfill
\end{frame}
\end{document}







