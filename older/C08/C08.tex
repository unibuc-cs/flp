\documentclass[xcolor=pdftex,romanian,colorlinks]{beamer}
%\documentclass[xcolor=pdftex,handout,romanian,colorlinks]{beamer}

\usepackage[export]{adjustbox}
\usepackage{../sty/tslides}
\usepackage[all]{xy}
\usepackage{pgfplots}
\usepackage{flowchart}
\usepackage{todonotes}
\usetikzlibrary{arrows,positioning,calc}
\lstset{language=Haskell}
\lstset{escapeinside={(*@}{@*)}}
\PrerenderUnicode{ăĂîÎȘșȚțâÂ}
\usepackage{amsmath}


%\usepackage{xcolor}
%\definecolor{IntensColor}{HTML}{2E86C1}
%\definecolor{StateTransition}{HTML}{D6EAF8}
%\definecolor{MedianLightOrange}{RGB}{216,178,92}
%\definecolor{Orchid}{HTML}{8E44AD}
%\definecolor{True}{HTML}{229954}
%\definecolor{False}{HTML}{CB4335}


\usepackage{proof}
\usepackage{multirow}
\usepackage{alltt}
\usepackage{mathpartir}
\usepackage{ulem}

\newcommand{\structured}[1]{#1}

\definecolor{IntensColor}{HTML}{2E86C1}
\definecolor{StateTransition}{HTML}{D6EAF8}
\definecolor{MedianLightOrange}{RGB}{216,178,92}
\definecolor{Orchid}{HTML}{8E44AD}
\definecolor{True}{HTML}{229954}
\definecolor{False}{HTML}{CB4335}

\newcommand{\cin}[1]{{\color{cobalt} #1}}
\newcommand{\sel}[1]{{\color{Orchid} #1}}

\newcommand{\intens}[1] {{\color{IntensColor} #1}}
\newcommand{\exe}[1] {{\color{True} #1}}

\newcommand{\la}{\lambda}

\setlength{\leftmargini}{0pt}

\newcommand{\app}[2]{#1\, #2}
\newcommand{\abs}[2]{\lambda #1.\,#2}

\newcommand{\type}[2]{{\color{True}#1\hspace{-.05cm}:}\,{\color{Orchid}#2}}

\newcommand{\sub}[3]{#1\langle#2/#3\rangle}
\newcommand{\subt}[3]{#1[#2/#3]}
\newcommand{\equiva}{=_\alpha}

\newcommand{\trueL}{\mathbf{T}}
\newcommand{\falseL}{\mathbf{F}}
\newcommand{\notL}{\mathbf{not}}
\newcommand{\andL}{\mathbf{and}}
\newcommand{\orL}{\mathbf{or}}
\newcommand{\ifL}{\mathbf{if}}
\newcommand{\boolL}{\mathbf{bool}}

\newcommand{\maybeL}{\mathbf{maybe}}
\newcommand{\nothingL}{\mathbf{Nothing}}
\newcommand{\justL}{\mathbf{Just}}
\newcommand{\Maybe}[1]{\mathop{\mathbf{Maybe}}{#1}}

\newcommand{\foldrL}{\mathbf{foldr}}
\newcommand{\nilL}{\mathbf{Nil}}
\newcommand{\consL}{\mathbf{Cons}}
\newcommand{\ListL}[1]{\mathop{\mathbf{List}}{#1}}

\newcommand{\unpairL}{\mathbf{uncons}}
\newcommand{\pairL}{\mathbf{Pair}}
\newcommand{\firstL}{\mathbf{first}}
\newcommand{\secondL}{\mathbf{second}}
\newcommand{\Pair}[2]{\mathop{\mathop{\mathbf{Pair}}{#1}}{#2}}

\newcommand{\succL}{\mathbf{Succ}}
\newcommand{\zeroL}{\mathbf{Zero}}
\newcommand{\iterateL}{\mathbf{iterate}}
\newcommand{\addL}{\mathbf{add}}
\newcommand{\mulL}{\mathbf{mul}}
\newcommand{\expL}{\mathbf{exp}}
\newcommand{\isZero}{\mathbf{isZero}}
\newcommand{\pred}{\mathbf{pred}}
\newcommand{\factL}{\mathbf{fact}}

\newcommand{\BoolT}{\ensuremath{\texttt{Bool}}}
%\newcommand{\BoolT}{\ensuremath{\texttt{Bool}}}
\newcommand{\ifT}[3]{\mathrm{if}\ #1\ \mathrm{then}\ #2\ \mathrm{else}\ #3}

\newcommand{\UnitT}{\ensuremath{\texttt{Unit}}}
\newcommand{\unit}{\mathrm{unit}}

\newcommand{\VoidT}{\ensuremath{\texttt{Void}}}
\newcommand{\void}{\mathrm{void}}


\newcommand{\ProductT}[2]{#1 \times #2}
\newcommand{\PairL}[2]{\langle #1,#2\rangle}
\newcommand{\ProjOne}[1]{fst\ #1}
\newcommand{\ProjTwo}[1]{snd\ #1}

\newcommand{\SumT}[2]{#1 + #2}
\newcommand{\Left}[1]{\mathrm{Left}\ #1}
\newcommand{\Right}[1]{\mathrm{Right}\ #1}
\newcommand{\Case}[3]{\mathrm{case}\ #1\ \mathrm{of}\ #2\ ;\ #3}

%----------------------------------------------

\newcommand{\SSnot}{\terminal{not}}

\newcommand{\Sand}{\terminal{and}}
\newcommand{\Sor}{\terminal{or}}
\newcommand{\Splus}{\terminal{+}}
\newcommand{\Smul}{\terminal{*}}
\newcommand{\Ssucc}{\terminal{S}}
\newcommand{\Spow}{\terminal{pow}}
\newcommand{\Spred}{\terminal{pred}}
\newcommand{\Seq}{\terminal{eq}}
\newcommand{\Sneq}{\terminal{neq}}

\newcommand{\SisZero}{\terminal{isZero}}
\newcommand{\Slte}{\terminal{<=}}
\newcommand{\Sgte}{\terminal{>=}}
\newcommand{\Slt}{\terminal{<}}
\newcommand{\Sgt}{\terminal{>}}
\newcommand{\Spair}{\terminal{pair}}
\newcommand{\Sfst}{\terminal{fst}}
\newcommand{\Ssnd}{\terminal{snd}}
\newcommand{\Sminus}{\terminal{-}}

\newcommand{\Snull}{\terminal{null}}
\newcommand{\Scons}{\terminal{cons}}
%\newcommand{\c sead}{\terminal{head}}
\newcommand{\SisNull}{\terminal{?null}}
\newcommand{\Stail}{\terminal{tail}}
\newcommand{\Ssum}{\terminal{sum}}
\newcommand{\Sfoldr}{\terminal{foldr}}
\newcommand{\Smap}{\terminal{map}}
\newcommand{\Sfilter}{\terminal{filter}}

\newcommand{\const}[1]{\triangleright {\color{False} #1}}

\newcommand{\egf}[1]{\stackrel{\cdot}{=}_{#1}}

\newcommand{\Conf}[2]{\ensuremath{\langle #1\ ,\ #2\rangle}}
\newcommand{\plus}[1] {{\color{True} #1}}
\newcommand{\te}[1]{\mbox{\texttt{#1}}}

\newcommand{\vexp}{\ensuremath{\mathbb{E}}}
\newcommand{\bexp}{\ensuremath{\mathbb{B}}}
\newcommand{\cmd}{\ensuremath{\mathbb{C}}}

\definecolor{section-color}{HTML}{23373b} %mDarkTeal
%\AtBeginSection[]{
%  \begin{frame}
%  \vfill
%  \centering
%  \begin{beamercolorbox}[sep=8pt,center,shadow=true,rounded=true]{title}
%    \usebeamerfont{title}\insertsectionhead\par%
%  \end{beamercolorbox}
%  \vfill
%  \end{frame}
%}



\title[FLP]{Fundamentele limbajelor de programare}
\subtitle{C08}
\date{}


\begin{document}
\begin{frame}
  \titlepage
\end{frame}

\setlength{\leftmargini}{12pt}


%================================================
\section{\color{section-color} Lambda calcul cu tipuri simple și unificare (recap)}
%================================================


%------------------------------------------------
\begin{frame}{Ce problemă am rezolvat în cursul trecut?}

\intens{\textit{Type Inference}}

Pentru un lambda termen $M$ fără tipuri, am adnotat termenul $M$ cu tipuri obținând $\overline{M}$ și am rezolvat (parțial) problema
\vspace{-.3cm}
\begin{center}
$? \vdash \type{\overline{M}}{?}$
\end{center}
(am găsit un context și un tip, pentru a avea o judecată legală).

\end{frame}

%------------------------------------------------
\begin{frame}{Type Inference}

Fie $M$ un lambda termen fără tipuri.

Construim un context \alert{$\Gamma_M$} pentru $M$:
\vspace{-.2cm}
\begin{center}
$\Gamma_M = \{\type{x}{X}\ |\ x \in FV(M)\}$
\end{center}
\vspace{-.2cm}
(toate variabilele de tip $X$ introduse mai sus sunt noi și distincte)


  \vspace{.2cm}
Adnotăm $M$ cu tipuri pentru variabilele legate obținând \alert{$\overline{M}$} prin inducție după structura lui $M$ astfel:
\vspace{-.2cm}
\begin{itemize}
	\item dacă $M = x$, atunci $\overline{M} = M$
	\item dacă $M = \app{M_1}{N_1}$, atunci $\overline{M} = \app{\overline{M_1}}{\overline{N_1}}$
	\item dacă $M = \abs{x}{N}$, atunci $\overline{M} = \abs{\type{x}{X}}{\overline{N}}$, unde $X$ este o variabilă de tip nouă
\end{itemize}
\end{frame}

%------------------------------------------------
\begin{frame}{Sistemul $\lambda\hspace{-.1cm}\to$ cu constrângeri}

\begin{center}
$\Gamma \vdash \type{M}{\sigma} \const{C}$

\vspace{.4cm}
\begin{tabular}{c}
\infer[(var^*)]
	{\Gamma \cup \{\type{x}{\tau}\}\vdash \type{x}{\sigma} \const{\{\sigma \egf{} \tau\}}}
	{} 
\end{tabular}

\vspace{.4cm}
\begin{tabular}{c}
 \infer[(\to_I^*)]
	{\Gamma \vdash \type{(\abs{\type{x}{\sigma}}{M})}{\tau} \const{C} }
	 {\Gamma, \type{x}{\sigma} \vdash \type{M}{\tau'} \const{C'} \vspace{.1cm}\\ {\color{False} C = C' \cup \{\tau \egf{} \sigma \to \tau'\}}}
\end{tabular}

\vspace{.4cm}
\begin{tabular}{c}
\infer[(\to_E^*)]
	{\Gamma \vdash \type{\app{M}{N}}{\tau} \const{C}}
	 {\Gamma \vdash \type{M}{\tau_1}\const{C_1} \hspace{.5cm} \Gamma \vdash \type{N}{\tau_2} \const{C_2} \vspace{.1cm}\\  {\color{False} C = C_1 \cup C_2 \cup \{\tau_1 \egf{} \tau_2 \to \tau\}}}
\end{tabular}

\vspace{.4cm}
$\sigma, \tau, \tau', \tau_1, \tau_2$ variabile de tip

\end{center}
\end{frame}

%------------------------------------------------
\begin{frame}{Sistemul $\lambda\hspace{-.1cm}\to$ cu constrângeri}

O judecată de forma $\Gamma \vdash \type{M}{\sigma} \const{C}$ este \alert{legală} dacă constrângerile din $C$ au o "soluție".


\vspace{.4cm}
Fie $M$ un lambda termen fără tipuri.
Dacă există o constrângere de tipuri \alert{$C_M$} și o variabilă de tip nouă $V$ astfel  încât 
\vspace{-.2cm}
\begin{center}
$\Gamma_M \vdash \type{\overline{M}}{V} \const{C_M}$
\end{center}
\vspace{-.2cm}
este o judecată legală, atunci $M$ este \textit{typable}. \\
(soluția o găsim prin $C_M$)

\end{frame}


%------------------------------------------------
\begin{frame}{Type Inference - Exemplul 1}

{\footnotesize

Fie $M_1 = \app{(\abs{z}{\abs{u}{z}})}{(\app{y}{x})}$.  \\
Obținem $\Gamma_{M_1} = \{\type{x}{X}, \type{y}{Y}\}$ și $\overline{M_1} = \app{(\abs{\type{z}{Z}}{\abs{\type{u}{U}}{z}})}{(\app{y}{x})}$.

\vspace{-.4cm}
\begin{center}
\begin{tabular}{c}
\hspace{-.6cm}
\infer[(\to_E^*)]
	{
	\onslide<1->{\Gamma_{M_1} \vdash \type{\app{(\abs{\type{z}{Z}}{\abs{\type{u}{U}}{z}})}{(\app{y}{x})}}{V} \const{C_{M_1}}}
	}
	{
		\onslide<1->{
	\infer[(\to_I^*)]
	{\Gamma_{M_1}  \vdash 	\type{\abs{\type{z}{Z}}{\abs{\type{u}{U}}{z}}}{\tau_1} \const{C_1}}
	{ 
	\infer[(\to_I^*)]
	{\Gamma_{M_1} \cup \{\type{z}{Z}\} \vdash \type{\abs{\type{u}{U}}{z}}{\tau_1'} \const{C_1'}}
	{\Gamma_{M_1} \cup \{\type{z}{Z}, \type{u}{U}\} \vdash \type{z}{\delta} \const{D}  \vspace{.1cm}\\ 
	{\color{False} C_1' = D \cup \{\tau_1' \egf{} (U \to \delta)\}}
	}
	 \vspace{.1cm}\\ 
	{\color{False} C_1 = C_1' \cup \{\tau_1 \egf{} (Z \to \tau_1')\} }
	}
	}
	\quad \quad
		\onslide<1->{
	\infer[(\to_E^*)]
	{\Gamma_{M_1}  \vdash \type{\app{y}{x}}{\tau_2} \const{C_2}}
	{
	\Gamma_{M_1}  \vdash \type{y}{\sigma_1} \const{C_2'} \quad
	\Gamma_{M_1} \vdash \type{x}{\sigma_2} \const{C_2''} \vspace{.1cm}\\ 
	{\color{False} C_2 = C_2' \cup C_2'' \cup \{\sigma_1 \egf{} \sigma_2 \to \tau_2\}}	
	}
	 \vspace{.2cm}\\ 
	{\color{False} C_{M_1} = C_1 \cup C_2 \cup \{\tau_1 \egf{} (\tau_2 \to V)}\}}
	}
\end{tabular}
\end{center}

\vspace{-.4cm}
\begin{tabular}{cc}
\hspace{-1cm}
\begin{tabular}{l}
$D= \{\delta \egf{} Z\}$  \\
$C_1' = \{\delta \egf{} Z, \tau_1' \egf{} (U \to \delta)\}$ \\
$C_1 = \{\delta \egf{} Z, \tau_1' \egf{} (U \to \delta), \tau_1 \egf{} (Z \to \tau_1')\}$ 
\end{tabular}
&
\begin{tabular}{l}
$C_2' = \{\sigma_1 \egf{} Y\}$  \\
$C_2'' = \{\sigma_2 \egf{} X\}$ \\
$C_2 = \{\sigma_1 \egf{} Y, \sigma_2 \egf{} X, \sigma_1 \egf{} \sigma_2 \to \tau_2\}$
\end{tabular} 
\end{tabular}
\begin{center}
$C_{M_1} = \{\delta \egf{} Z, \tau_1' \egf{} (U \to \delta), \tau_1 \egf{} (Z \to \tau_1'), \sigma_1 \egf{} Y, \sigma_2 \egf{} X,$ \\ $\sigma_1 \egf{} \sigma_2 \to \tau_2, \tau_1 \egf{} (\tau_2 \to V)\}$
\end{center}

Constrângerile $C_{M_1}$ au "soluție". Ce înseamnă asta?
}
\end{frame}

%------------------------------------------------
\begin{frame}{Type Inference - Exemplul 2}

{\footnotesize

Fie $M_2 = \app{x}{x}$.  \\
Obținem $\Gamma_{M_2} = \{\type{x}{X}\}$ și $\overline{M_2} = M_2$.

\begin{center}
\begin{tabular}{c}
\infer[(\to_E^*)]
	{ \{\type{x}{X}\} \vdash \type{(\app{x}{x})}{V} \const{C_{M_2}}
	}
	{
	\{\type{x}{X}\} \vdash \type{x}{\tau_1}\const{C_1} \quad \{\type{x}{X}\} \vdash \type{x}{\tau_2} \const{C_2} \vspace{.1cm}\\  {\color{False} C_M = C_1 \cup C_2 \cup \{\tau_1 \egf{} \tau_2 \to V\}}
	}
\end{tabular}
\end{center}

\begin{tabular}{l}
$C_1 = \{\tau_1 \egf{} X\}$  \\
$C_2 = \{\tau_2 \egf{} X\}$ \\
$C_{M_2} = \{\tau_1 \egf{} X, \tau_2 \egf{} X, \tau_1 \egf{} \tau_2 \to V\}$
\end{tabular} 

Constrângerile $C_{M_2}$ nu au "soluție". Ce înseamnă asta?

Constrângerile au "soluție" dacă se pot \intens{unifica}.
}
\end{frame}


\begin{frame}{Termeni}

\intens{Alfabet:}
\vspace{-.2cm}
\begin{itemize}
	\item $\mathcal{F}$ o mulțime de simboluri de funcții de aritate cunoscută
	\item $\mathcal{V}$ o mulțime numărabilă de variabile 
	\item $\mathcal{F}$ și $\mathcal{V}$ sunt disjuncte
\end{itemize}

  \bigskip
\intens{Termeni peste $\mathcal{F}$ si $\mathcal{V}$:}
\vspace{-.2cm}
\begin{align*}
t \ ::= \ x\ |\ f(t_1,\ldots,t_n)
\end{align*}
\vspace{-1cm}
\begin{itemize}
	\item $n \geq 0$
	\item $x$ este o variabilă
	\item $f$ este un simbol de funcție de aritate $n$ 
\end{itemize}

\end{frame}

%---------------------------------------------------------------------
\begin{frame}{Termeni}
\intens{Notații:}
\vspace{-.2cm}
\begin{itemize}
	\item \intens{constante:} simboluri de funcții de aritate $0$
	\item $x,y,z,\ldots$ pentru variabile
	\item $a,b,c,\ldots$ pentru constante
	\item $f,g,h,\ldots$ pentru simboluri de funcții arbitrare
	\item $s,t,u,\ldots$ pentru termeni
	\item $var(t)$ mulțimea variabilelor care apar în $t$
	\item ecuații $s \egf{} t$ pentru o pereche de termeni
	\item $Trm_{\mathcal{F,\mathcal{V}}}$ mulțimea termenilor peste $\mathcal{F}$ și $\mathcal{V}$
\end{itemize}

\end{frame}

%---------------------------------------------------------------------
\begin{frame}{Legătura cu teoria tipurilor}

Mulțimea  \alert{tipurilor simple}  \hspace{.2cm}
 \intens{$\mathbb{T} = \mathbb{V}\ |\ \mathbb{T} \rightarrow \mathbb{T}$}
 
În acest caz, avem alfabetul:
\vspace{-.2cm}
\begin{itemize}
	\item $\mathcal{F} = \{\to\}$, iar aritatea lui $\to$ este $2$
	\item $\mathcal{V} = \mathbb{V}$
\end{itemize}

\vspace{.4cm}
Dacă avem și alte tipuri, extindem $\mathcal{F}$ cu noi simboluri. De exemplu,
\vspace{-.2cm}
\begin{itemize}
	\item $\UnitT, \VoidT$ cu aritate $0$ (deci constante)
	\item $\BoolT, \texttt{Nat}$ cu aritate $0$ (deci constante)
	\item $\texttt{Maybe}, \texttt{List}$ cu aritate $1$
	\item $\times$ cu aritate $2$
	\item $\ldots$
\end{itemize}


\end{frame}

%---------------------------------------------------------------------
\begin{frame}{Substituții}

O \intens{substituție} $\Theta$ este o funcție de la variabile la termeni,
\vspace{-.2cm}
	\begin{center}
	\intens{$\Theta: \mathcal{V} \to Trm_{\mathcal{F,\mathcal{V}}}$}
	\end{center}
\intens{Notație} pentru substituții care schimbă un număr finit de variabile:
$$[u_1 / x_1, u_2 / x_2, \ldots, u_n / x_n]$$

\medskip
\intens{Aplicarea unei substituții} $\Theta$ unui termen $t$:
\begin{align*}
\Theta(t) = 
\left\{
	\begin{array}{ll}
		\Theta(x), \mbox{ dacă } t = x \\
		f(\Theta(t_1),\ldots,\Theta(t_n)), \mbox{ dacă } t = f(t_1,\ldots,t_n) \\
	\end{array}
\right. 
\end{align*}

\intens{Notație} pentru $\Theta(t)$: $t [u_1 / x_1, u_2 / x_2, \ldots, u_n / x_n]$
\end{frame}
%---------------------------------------------------------------------



%---------------------------------------------------------------------
\begin{frame}{Unificare}

 Doi termeni $t_1$ și $t_2$ \intens{se unifică} dacă există o substituție $\Theta$ astfel încât
 \vspace{-.2cm}
	\begin{center}
	\intens{$\Theta(t_1) = \Theta(t_2)$}.
	\end{center}
	
În acest caz, $\Theta$ se numește un \intens{unificator} al termenilor $t_1$ și $t_2$.
	
	\medskip  
Un unificator $\Theta$ pentru $t_1$ și $t_2$ este \intens{cel mai general unificator} (\intens{cmgu,mgu}) dacă pentru orice alt unificator $\Theta'$ pentru $t_1$ și $t_2$, există o substituție $\Delta$ astfel încât
 \vspace{-.2cm}
	\begin{center}
	\intens{$\Theta' = \Theta ; \Delta$}.
	\end{center}

De exemplu, dacă $\Theta$ este  $[u_1 / x_1, u_2 / x_2, \ldots, u_n / x_n]$, atunci
$\Delta$ este de forma $[v_1 / y_1, v_2 / y_2, \ldots, v_n / y_m]$ cu $x_i \neq y_j$ pentru orice
alegere a lui $i$ și $j$, și
$$\Theta' = [\Delta(u_1) / x_1, \ldots \Delta(u_n) / x_n, v_1 / y_1, \ldots, v_m / y_m]$$
\end{frame}
%---------------------------------------------------------------------

%------------------------------------------------------------------------
\begin{frame}{Unificatori}

 {\color{True} Exemplu:}
\begin{itemize}
	\item $\intens{t = x + (y * y)} = +(x,*(y,y))$
	\vspace{.1cm}
	\item $\intens{t' = x + (y * x)} = +(x,*(y,x))$
	 
	\vspace{.1cm}
	\item $\Theta = \{x \mapsto y\}$
	\begin{itemize}
		\item $\Theta(t) = y + (y * y)$
		\item $\Theta(t') = y + (y * y)$
		\item $\Theta$ este \intens{cmgu}
	\end{itemize}
	 
	\vspace{.1cm}
	\item $\Theta'= \{x \mapsto 0, y \mapsto 0\}$
		\begin{itemize}
		\item $\Theta'(t) = 0 + (0 * 0)$
		\item $\Theta'(t') = 0 + (0 * 0)$
		 
		\item $\Theta' = \Theta ; \{y \mapsto 0\}$
		 
		\item $\Theta'$ este \intens{unificator}, dar nu este \intens{cmgu}
	\end{itemize}
\end{itemize}

\end{frame}

%================================================
\section{\color{section-color} Algoritmul de unificare}
%================================================

%------------------------------------------------------------------------
\begin{frame}{Algoritmul de unificare}
\begin{itemize}
	\item Pentru o mulțime finită de termeni \intens{$\{t_1,\ldots, t_n\}$, $n \geq 2$}, \\\intens{algoritmul de unificare }stabilește dacă există un cmgu.
	\vspace{.2cm}
	\item Există algoritmi mai eficienți, \\dar îl alegem pe acesta pentru simplitatea sa.
	\vspace{.2cm}  
	\item Algoritmul lucrează cu două liste:
	\begin{itemize}
		\item \intens{Lista soluție: $S$}
		\item \intens{Lista de rezolvat: $R$}		
	\end{itemize}
	\vspace{.2cm}  
	\item \intens{Inițial:}
	\begin{itemize}
		\item \intens{Lista soluție: $S = \emptyset$}
		\item \intens{Lista de rezolvat: $R = \{t_1 \egf{} t_2, \ldots, t_{n-1} \egf{} t_n\}$}	 \\
		\intens{$\egf{}$} este un simbol nou care ne ajută să formăm perechi de termeni ("ecuații")	
	\end{itemize}
\end{itemize}
\end{frame}

%------------------------------------------------------------------------
\begin{frame}{Algoritmul de unificare}
Algoritmul constă în aplicarea regulilor de mai jos:
\begin{itemize}
	\vspace{.2cm}  
	\item \intens{SCOATE}
		\begin{itemize}
			\item orice ecuație de forma \intens{$t \egf{} t$} din \intens{$R$} este \intens{eliminată}.
		\end{itemize}
	\vspace{.2cm}  
	\item \intens{DESCOMPUNE}
		\begin{itemize}
			\item orice ecuație de forma \intens{$f(t_1,\ldots,t_n) \egf{} f(t_1',\ldots,t_n')$} din \intens{$R$} este \intens{înlocuită} cu ecuațiile \intens{$t_1\egf{}t_1', \ldots, t_n \egf{}t_n'$}.
		\end{itemize}
	\vspace{.2cm}  
	\item \intens{REZOLVĂ}
	\begin{itemize}
			\item orice ecuație de forma \intens{$x \egf{} t$} sau \intens{$t \egf{} x$} din \intens{$R$}, unde \intens{variabila $x$ nu apare în termenul $t$}, este \intens{mutată} sub forma \intens{$x \egf{} t$} în \intens{$S$}. \\\intens{\^In toate celelalte ecuații} (din $R$ și $S$), \intens{$x$ este înlocuit cu $t$}.	
		\end{itemize}
\end{itemize}
\end{frame}

%------------------------------------------------------------------------
\begin{frame}{Algoritmul de unificare}
Algoritmul \intens{se termină normal} dacă \intens{$R = \emptyset$}.\\
 În acest caz, \intens{$S$ conține cmgu}.

\vspace{.5cm}  
Algoritmul este oprit cu concluzia {\color{False} inexistenței unui unificator} dacă:
\begin{enumerate}
	\item În $R$ există o ecuație de forma
	\begin{center}
	{\color{False}$f(t_1,\ldots,t_n) \egf{} g(t_1',\ldots,t_k')$} cu {\color{False}$f \neq g$}.
	\end{center}
	\item În $R$ există o ecuație de forma {\color{False}$x \egf{} t$} sau {\color{False}$t \egf{} x$} și {\color{False}variabila $x$ apare în termenul $t$}.
\end{enumerate}
\end{frame}

%------------------------------------------------------------------------
\begin{frame}{Algoritmul de unificare - schemă}

\begin{center}
\begin{minipage}{15cm}
\hspace{-.6cm}
\begin{tabular}{|c|c|c|}
\hline
& \intens{Lista soluție} & \intens{Lista de rezolvat} \\
& \intens{S} & \intens{R} \\  \hline \hline
\intens{Inițial} & $\emptyset$ & $t_1 \egf{} t_1',\ldots, t_n \egf{} t_n'$ \\  \hline \hline
\intens{SCOATE} & $S$ & $R'$, \intens{$t \egf{} t$} \\  \cline{2-3} 
 & $S$ & $R'$ \\ \hline \hline
 \intens{DESCOMPUNE} & $S$ & $R'$, \intens{$f(t_1,\ldots,t_n) \egf{} f(t_1',\ldots,t_n')$} \\ \cline{2-3}
 & $S$ & $R'$, \intens{$t_1\egf{}t_1', \ldots t_n \egf{}t_n'$} \\ \hline \hline
 \intens{REZOLVĂ} & $S$ & $R'$, \intens{$x \egf{} t$} sau \intens{$t \egf{} x$}, $x$ nu apare în $t$ \\ \cline{2-3}
 & \intens{$x \egf{} t$}, \intens{$S[t / x]$} & \intens{$R'[t / x]$} \\ \hline \hline
 \intens{Final} & $S$ & \intens{$\emptyset$} \\ \hline 
\end{tabular}

\vspace{.4cm}
\intens{$S[t / x]$}: în toate ecuațiile din $S$, $x$ este înlocuit cu $t$
\end{minipage}
\end{center}
\end{frame}

%------------------------------------------------------------------------
\begin{frame}{Exemplul 1}
\vspace{.4cm}


{\color{True} Ecuațiile $\{g(y) \egf{}x,\ f(x,h(x),y) \egf{} f(g(z),w,z)\}$ au cmgu?}

\pause
{\footnotesize
\begin{center}
\begin{tabular}{|c|c|c|}
\hline
$S$ & $R$ & \\ \hline 
$\emptyset$ & $g(y) \egf{} x,\ f(x,h(x),y) \egf{} f(g(z),w,z)$ & {\scriptsize   \intens{REZOLVĂ}}  \\ \hline  
$\intens{x \egf{} g(y)}$ & $f(\intens{g(y)},h(\intens{g(y)}),y) \egf{} f(g(z),w,z)$   & {\scriptsize   \intens{DESCOMPUNE}}   \\ \hline  
$x \egf{} g(y)$ & $g(y) \egf{} g(z),\ h(g(y)) \egf{} w,\ y \egf{} z$ & {\scriptsize   \intens{REZOLVĂ}}   \\ \hline  
$\intens{w \egf{} h(g(y))},$ & $g(y) \egf{} g(z),\ y \egf{} z$ & {\scriptsize  \intens{REZOLVĂ}}   \\
$\ x \egf{} g(y)$ & &   \\ \hline 
$\intens{y \egf{} z}, x \egf{} g(\intens{z}),$ & $g(\intens{z}) \egf{} g(z)$ & {\scriptsize \intens{SCOATE}} \\
$w \egf{} h(g(\intens{z}))$ & &   \\ \hline  
$y \egf{} z, x \egf{} g(z), $ & $\emptyset$ & \\
$w \egf{} h(g(z))$ & &   \\ \hline 
\end{tabular}
\end{center}
}

\intens{$\Theta = \{y \mapsto z,\ x \mapsto g(z),\ w \mapsto h(g(z)) \}$ este cmgu.} 


\end{frame}

%------------------------------------------------------------------------
\begin{frame}{Exemplul 2}


{\color{True} Ecuațiile $\{g(y) \egf{} x,\ f(x,h(y),y) \egf{} f(g(z),b,z)\}$ au cmgu?}

\pause
\begin{center}
 {\footnotesize
\begin{tabular}{|c|c|c|}
\hline
$S$ & $R$ & \\ \hline
$\emptyset$ & $g(y) \egf{} x,\ f(x,h(y),y) \egf{} f(g(z),b,z)$ &{\scriptsize \intens{REZOLVĂ}} \\ \hline
$\intens{x \egf{} g(y)}$ & $f(\intens{g(y)},h(y),y) \egf{} f(g(z),b,z)$   & {\scriptsize \intens{DESCOMPUNE}} \\ \hline
$x \egf{} g(y)$ & $g(y) \egf{} g(z),\ {\color{False}h(y) \egf{} b},\ y \egf{} z$ & {\scriptsize {\color{False}- EȘEC - }} \\ \hline
\end{tabular}
}
\end{center}
 
 
\begin{itemize}
	\item {\color{False}$h$ și $b$ sunt simboluri de funcții diferite!}
	\item Nu există unificator pentru acești termeni.
\end{itemize}

\end{frame}

%------------------------------------------------------------------------
\begin{frame}{Exemplul 3}


{\color{True} Ecuațiile $\{g(y) \egf{} x,\ f(x,h(x),y) \egf{} f(y,w,z)\}$ au cmgu?}

\pause
\begin{center}
 {\footnotesize
\begin{tabular}{|c|c|c|}
\hline
$S$ & $R$ & \\ \hline
$\emptyset$ & $g(y) \egf{} x,\ f(x,h(x),y) \egf{} f(y,w,z)$ &{\scriptsize \intens{REZOLVĂ}} \\ \hline
$\intens{x \egf{} g(y)}$ & $f(\intens{g(y)},h(\intens{g(y)}),y) \egf{} f(y,w,z)$   & {\scriptsize \intens{DESCOMPUNE}} \\ \hline
$x \egf{} g(y)$ & ${\color{False}g(y) \egf{} y},\ h(g(y)) \egf{} w,\ y \egf{} z$ & {\scriptsize {\color{False}- EȘEC - }} \\ \hline
\end{tabular}
}
\end{center}
 
\begin{itemize}
	\item {\color{False}\^In ecuația ${g(y) \egf{} y}$, variabila $y$ apare în termenul $g(y)$.}
	\item Nu există unificator pentru aceste ecuații.
\end{itemize}
\end{frame}


%------------------------------------------------------------------------
\begin{frame}{Exemplul 4}

Înapoi la constrângerea obținută când am vorbit de \textit{type inference} pentru termenul {\color{True} $M_1 = \app{(\abs{z}{\abs{u}{z}})}{(\app{y}{x})}$}.

Am obținut constrângerile
\begin{center}
\intens{$C_{M_1} = \{\delta \egf{} Z, \tau_1' \egf{} (U \to \delta), \tau_1 \egf{} (Z \to \tau_1'), \sigma_1 \egf{} Y, \sigma_2 \egf{} X,$ \\ $\sigma_1 \egf{} \sigma_2 \to \tau_2, \tau_1 \egf{} (\tau_2 \to V)\}$}
\end{center}
\begin{itemize}
	\item $\to$ simbol de funcție de aritate $2$
	\item $\delta, \tau_1, \tau_1', \tau_2, \sigma_1, \sigma_2, X,Y, Z, U, V$ variabile
\end{itemize}


\end{frame}

%------------------------------------------------------------------------
\begin{frame}{Exemplul 4 (cont.)}

\vspace{-.4cm}
\begin{center}
 {\footnotesize
 \begin{minipage}{15cm}
 \hspace{-.6cm}
\begin{tabular}{|c|c|c|}
\hline
$S$ & $R$ & \\ \hline
$\emptyset$ & $\intens{\delta} \egf{} Z,\ \tau_1' \egf{} (U \to \intens{\delta}),\ \tau_1 \egf{} (Z \to \tau_1'),\ \sigma_1 \egf{} Y$ &  {\scriptsize \intens{REZ.}} \\
& $\sigma_2 \egf{} X,\ \sigma_1 \egf{} \sigma_2 \to \tau_2,\ \tau_1 \egf{} (\tau_2 \to V)$ & \\ \hline
$\delta \egf{} Z$ & $\intens{\tau_1'} \egf{} (U \to Z),\ \tau_1 \egf{} (Z \to \intens{\tau_1'}),\ \sigma_1 \egf{} Y$ &  {\scriptsize \intens{REZ.}} \\
& $\sigma_2 \egf{} X,\ \sigma_1 \egf{} \sigma_2 \to \tau_2,\ \tau_1 \egf{} (\tau_2 \to V)$ & \\ \hline
$\delta \egf{} Z,\ \tau_1' \egf{} (U \to Z)$ & $\intens{\tau_1} \egf{} (Z \to (U \to Z)),\ \sigma_1 \egf{} Y$ &  {\scriptsize \intens{REZ.}} \\
& $\sigma_2 \egf{} X,\ \sigma_1 \egf{} \sigma_2 \to \tau_2,\ \intens{\tau_1} \egf{} (\tau_2 \to V)$ & \\ \hline
$\delta \egf{} Z,\ \tau_1' \egf{} (U \to Z),$ & $\sigma_1 \egf{} Y,\ \sigma_2 \egf{} X,\ \sigma_1 \egf{} \sigma_2 \to \tau_2,$ &  {\scriptsize \intens{DESC.}} \\
$\tau_1 \egf{} (Z \to (U \to Z))$ & $(Z \to (U \to Z)) \egf{} (\tau_2 \to V)$ & \\ \hline
$\delta \egf{} Z,\ \tau_1' \egf{} (U \to Z),$ & $\intens{\sigma_1} \egf{} Y,\ \sigma_2 \egf{} X,\ \intens{\sigma_1} \egf{} \sigma_2 \to \tau_2,$ &  {\scriptsize \intens{REZ.}} \\
$\tau_1 \egf{} (Z \to (U \to Z))$ & $Z \egf{} \tau_2,\ U \to Z \egf{} V$ & \\ \hline
$\delta \egf{} Z,\ \tau_1' \egf{} (U \to Z),$ & $\intens{\sigma_2} \egf{} X,\ Y \egf{} \intens{\sigma_2} \to \tau_2,$ &  {\scriptsize \intens{REZ.}} \\
$\tau_1 \egf{} (Z \to (U \to Z)),$ & $Z \egf{} \tau_2,\ U \to Z \egf{} V$ & \\
$\sigma_1 \egf{} Y$ & & \\ \hline
$\delta \egf{} Z,\ \tau_1' \egf{} (U \to Z),$ & $Y \egf{} X \to \intens{\tau_2},$ &  {\scriptsize \intens{REZ.}} \\
$\tau_1 \egf{} (Z \to (U \to Z)),$ & $Z \egf{} \intens{\tau_2},\ U \to Z \egf{} V$ & \\
$\sigma_1 \egf{} Y, \sigma_2 \egf{} X$ & & \\ \hline

\end{tabular}
\end{minipage}
}
\end{center}
 
\end{frame}

%------------------------------------------------------------------------
\begin{frame}{Exemplul 4 (cont.)}

\vspace{-.4cm}
\begin{center}
 {\footnotesize
 \begin{minipage}{15cm}
 \hspace{-.6cm}
\begin{tabular}{|c|c|c|}
\hline
$S$ & $R$ & \\ \hline
$\delta \egf{} Z,\ \tau_1' \egf{} (U \to Z),$ & $\intens{Y} \egf{} X \to Z, \ U \to Z \egf{} V$ &  {\scriptsize \intens{REZ.}} \\
$\tau_1 \egf{} (Z \to (U \to Z)),$ & $$ & \\
$\sigma_1 \egf{} \intens{Y}, \sigma_2 \egf{} X,\ \tau_2 \egf{} Z$ & & \\ \hline
$\delta \egf{} Z,\ \tau_1' \egf{} (U \to Z),$ & $U \to Z \egf{} V$ &  {\scriptsize \intens{REZ.}} \\
$\tau_1 \egf{} (Z \to (U \to Z)),$ & $$ & \\
$\sigma_1 \egf{} X \to Z, \sigma_2 \egf{} X,\ \tau_2 \egf{} Z$ & & \\
$Y \egf{} X \to Z$ & & \\ \hline
$\delta \egf{} Z,\ \tau_1' \egf{} (U \to Z),$ &  &   \\
$\tau_1 \egf{} (Z \to (U \to Z)),$ & $$ & \\
$\sigma_1 \egf{} X \to Z, \sigma_2 \egf{} X,\ \tau_2 \egf{} Z$ & & \\
$Y \egf{} X \to Z,\ V \egf{} U \to Z $ & & \\ \hline

\end{tabular}
\end{minipage}
}
\end{center}

{\footnotesize \intens{Constrângerile se pot unifica!}}
\end{frame}

%------------------------------------------------------------------------
\begin{frame}{Exemplul 5}

Înapoi la constrângerea obținută când am vorbit de \textit{type inference} pentru termenul {\color{False}$M_2 = \app{x}{x}$}.

Am obținut constrângerile
\begin{center}
\intens{$C_{M_2} = \{\tau_1 \egf{} X, \tau_2 \egf{} X, \tau_1 \egf{} \tau_2 \to V\}$}
\end{center}
\vspace{-.2cm}
\begin{itemize}
	\item $\to$ simbol de funcție de aritate $2$
	\item $\tau_1,  \tau_2, V$ variabile
\end{itemize}


\end{frame}

%------------------------------------------------------------------------
\begin{frame}{Exemplul 5 (cont.)}

\vspace{-.4cm}

 {\footnotesize
 \begin{center}
\begin{tabular}{|c|c|c|}
\hline
$S$ & $R$ & \\ \hline
$\emptyset$ & $\intens{\tau_1} \egf{} X,\ \tau_2 \egf{} X,\ \intens{\tau_1} \egf{} \tau_2 \to V$ &  {\scriptsize \intens{REZ.}} \\  \hline
$\tau_1 \egf{} X$ & $\intens{\tau_2} \egf{} X,\ X \egf{} \intens{\tau_2} \to V$ &  {\scriptsize \intens{REZ.}} \\  \hline
$\tau_1 \egf{} X,\ \tau_2 \egf{} X$ & ${\color{False}X \egf{} X \to V}$ &  {\scriptsize {\color{False}- EȘEC -}} \\  \hline
\end{tabular}
\end{center}
}

 \begin{itemize}
	\item {\color{False}\^In ecuația ${X \egf{} X \to V}$, variabila $X$ apare în termenul $X \to V$.}
	\item Nu există unificator pentru aceste ecuații.
\end{itemize}
\end{frame}


%------------------------------------------------
\begin{frame}{Exerciții}

Considerăm 
\vspace{-.2cm}
\begin{itemize}
	\item $x,y,z,u,v$ variabile,
	\item $a,b,c$ simboluri de constantă,
	\item $h,g$ simboluri de funcție de aritate 1,
	\item $f$ simbol de funcție de aritate 2,
	\item $p$ simbol de funcție de aritate 3.
\end{itemize}

\vspace{.2cm}
{\color{True}Aplicați algoritmul de unificare de mai sus pentru termenii:}
\vspace{-.2cm}
\begin{enumerate}
	\item $p(a,x,h(g(y)))$ și $p(z,h(z), h(u))$
	\item $f(h(a),g(x))$ și $f(y,y)$
	\item $p(a,x,g(x))$ și $p(a,y,y)$
	\item $p(x,y,z)$ și $p(u,f(v,v), u)$
\end{enumerate}

\end{frame}


%------------------------------------------------
\begin{frame}{Exerciții - rezolvări}

{\footnotesize
1.

\begin{tabular}{|c|c|c|}
\hline
$S$ & $R$ & \\ \hline 
$\emptyset$ & $p(a, x, h(g(y)))= p(z, h(z), h(u))$ & DESCOMPUNE \\ \hline  
$\emptyset$ & $a \egf{} z, x \egf{} h(z), h(g(y)) \egf{} h(u)$ & REZOLVĂ\\ \hline  
$z \egf{} a$ & $x \egf{} h(a), h(g(y)) \egf{} h(u)$ & REZOLVĂ\\ \hline
$z \egf{} a, x \egf{} h(a)$ & $h(g(y)) \egf{} h(u)$ & DESCOMPUNE\\ \hline
$z \egf{} a, x \egf{} h(a)$ & $g(y) \egf{} u$ & REZOLVĂ\\ \hline
$z \egf{} a, x \egf{} h(a), u \egf{} g(y)$ & $\emptyset$ & \\ \hline
\end{tabular}

\vspace{.2cm}
$\Theta = \{z / a,\ x / h(a),\ u / g(y) \}$ este cmgu.

}
\end{frame}


%------------------------------------------------
\begin{frame}{Exerciții - rezolvări}

{\footnotesize
2.

\begin{tabular}{|c|c|c|}
\hline
$S$ & $R$ & \\ \hline 
$\emptyset$ & $f(h(a), g(x)) \egf{} f(y,y)$ & DESCOMPUNE \\ \hline  
$\emptyset$ & $y \egf{} h(a), y \egf{} g(x)$ & REZOLVĂ \\ \hline  
$y \egf{} h(a)$ & $g(x) \egf{} h(a)$ & EȘEC \\ \hline  
\end{tabular}

 Nu există unificator!

}
\end{frame}

%------------------------------------------------
\begin{frame}{Exerciții - rezolvări}

{\footnotesize
3.

\begin{tabular}{|c|c|c|}
\hline
$S$ & $R$ & \\ \hline 
$\emptyset$ & $p(a, x, g(x)) \egf{} p(a, y, y)$ & DESCOMPUNE \\ \hline  
$\emptyset$ & $a \egf{} a, x \egf{} y, y \egf{} g(x)$ & SCOATE\\ \hline  
$\emptyset$ & $x \egf{} y, y \egf{} g(x)$ & REZOLVĂ\\ \hline  
$x \egf{} y$ & $y \egf{} g(y)$ & EȘEC\\ \hline  
\end{tabular}

Nu există unificator!

}
\end{frame}

%------------------------------------------------
\begin{frame}{Exerciții - rezolvări}

{\footnotesize
4.

\begin{tabular}{|c|c|c|}
\hline
$S$ & $R$ & \\ \hline 
$\emptyset$ & $p(x, y, z) \egf{} p(u, f(v,v), u)$ & DESCOMPUNE \\ \hline  
$\emptyset$ & $x \egf{} u, y \egf{} f(v,v),  z \egf{} u$ & REZOLV\u A \\ \hline  
$x \egf{} u$ & $y \egf{} f(v,v),  z \egf{} u$ & REZOLV\u A \\ \hline  
$y \egf{} f(v,v), x \egf{} u$ & $z \egf{} u$ & REZOLV\u A \\ \hline  
$ z \egf{} u, y \egf{} f(v,v), x \egf{} u$ &  &  \\ \hline  
\end{tabular}

$\Theta = \{ z / u,\ y / f(v,v), \ x / u \}$ este cmgu.

}
\end{frame}

%---------------------------------------------
\begin{frame}
  \vfill
  \centering

\textbf{Pe data viitoare!}

  \vfill
\end{frame}
\end{document}







